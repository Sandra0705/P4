Under udarbejdelsen af problemet er der blevet foretaget nogle antagelser om forbrugeren og hans forbrugstilbøjeligheder. Disse antagelser er simple, og da de har indflydelse på, hvordan forbrugeren vælger at forbruge, påvirker de resultatet. De forskellige antagelser vil blive diskuteret i dette afsnit.

Antagelsen om, at pandemier og finanskriser ikke påvirker forbrugerens økonomi, er usandsynlig. Dette skyldes, at globale kriser kan have stor indflydelse på sandsynligheden for at blive arbejdsløs og dermed også sandsynligheden for at komme i arbejde. En uforudsigelig krise vil ofte medføre en stigning i arbejdsløsheden, hvorved der kommer større konkurrence på arbejdsmarkedet. Det vil derfor blive svære at beholde sit arbejde og komme i arbejde. Derudover vil en krise også påvirke en forbrugers forventninger til fremtiden og dermed forbrugerens forbrugstilbøjeligheder således, at der forbruges mindre og spares mere op. På samme måde som en krise kan påvirke ens forbrug, kan en stor fremgang i økonomien også påvirke arbejdsløsheden og forbrugstilbøjeligheden. 

Da der i dette projekt kun tages udgangspunkt i en tidsperiode på et år, er det usandsynligt, at der opstår en økonomisk krise eller en stor fremgang i økonomien. Derfor kan der argumenteres for, at det ikke er nødvendigt at tage højde for dette. Af denne grund kan det antages, at en forbrugers arbejdsstatus kan beskrives som en homogen Markov-kæde i begrænsede tidsperioder. Hvis tidsperioden var længere, ville man kunne lade $\alpha$ og $\beta$ være tidsafhængige, og på denne måde kan der tages højde for ændringer i arbejdsløshed forårsaget af blandt andet kriser, arbejdskontrakter eller arbejdsskader.

Det er antaget, at lønnen er fast, og at der ikke kan forekomme skatteændringer. Dermed er den disponible indkomst fast i de beslutningstidspunkter, han er i arbejde. Typisk vil forbrugerens disponible indkomst variere som følge af løn- og skatteændringer. Derfor vil forbrugsproblemet være mere realistisk, hvis man tilføjede en varierende disponibel indkomst. Dette kan gøres ved at lade lønnen være tidsafhængig eller implementere skatteændringer.

I basis-problemet, når indlånsrenten varieres, kan det aflæses på resultaterne, at forbrugeren forbruger mere ved højere indlånsrenter. Ved at sammenligne $\gamma_i=0$ og $\gamma_i=0.1$ i \autoref{fig:renter} er dette tydeligt, hvor forbruget omtrent er tre gange så stort ved $\gamma_i = 0.1$ til sidste beslutningstidspunkt. For forbrugeren er dette den optimale strategi, når han har en belønningsfunktion givet ved $r_t(a) = \sqrt{a}$. Hvis denne belønningsfunktion anvendes ved en længere tidsperiode, vil forbruget fortsætte med at stige på samme måde som i \autoref{fig:renter}. Dette er kun realistisk til et vist beslutningstidspunkt, da enhver forbruger har en øvre grænse for, hvor meget de vil forbruge. Når forbruget nærmer sig denne grænse, vil forbrugeren begynde at spare mere op. Dette vil ikke være tilfældet med belønningsfunktionen $r_t(a) = \sqrt{a}$, og derfor er denne belønningsfunktion kun realistisk i en begrænset tidperiode.

En diskretiseringsfaktor på $\Delta = 0.1$ anvendes i basis-problemet, og i det udvidede problem er $\Delta = 1$. Det gælder, at en lavere diskretiseringsfaktor vil give bedre resultater, som følge af mere præcise tal. Da problemet vokser eksplosivt, når $\Delta$ bliver mindre, er det for det udvidede problem ikke muligt at simulere problemet for et $\Delta$ mindre end $1$. Dette har resulteret i, at der er spring i dataet for det udvidede problem, hvilket eksempelvis bemærkes i dataet for \autoref{fig:lån_gamma_u}, hvor kurverne ikke er strengt voksende som i \autoref{fig:renter}. Disse spring ville forsvinde ved en lavere diskretiseringsfaktor. Da dataet er upræcist som et resultat af en høj diskretiseringfaktor, vanskeliggøres analysen heraf. %:)

Da problemet vokser eksplosivt i dets størrelse, er det nødvendigt at begrænse faktorerne, hvilket ikke er ideelt. I det udvidede problem analyseres det blandt andet, hvordan forbrugeren vil låne, når lånebegrænsningen stiger. Det er her kun muligt at simulere op til $L=4$ før problemet bliver for stort,
når de andre faktorer fastholdes. Det kan i \autoref{tab:ind_ud_lån_var} aflæses, at for $L \leq 4$ vil forbrugeren låne mest muligt til første beslutningstidspunkt. Ved at $L$ er begrænset, er det dermed ikke muligt ud fra det givne data at konkludere, om der til første beslutningstidspunkt er et maksimalt beløb, forbrugeren er villig til at låne. Ud fra antagelserne vil forbrugeren ikke låne det maksimale, hvis ikke han har mulighed for at betale lånet tilbage. Da det i det udvidede problem antages, at forbrugeren til ethvert beslutningstidspunkt skal forbruge, hvad han låner, er det sikkert, at der vil være en øvre grænse for, hvor meget han vil låne. Denne grænse afhænger af opsparingen, lønnen og renterne og kunne ikke bestemmes, da det kun var muligt at simulere problemet op til $L = 4$.

Der er altså flere antagelser og begrænsede faktorer, der er med til at gøre modellen mere urealistisk.

