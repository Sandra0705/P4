Under udarbejdelsen af problemet er der blevet foretaget nogle antagelser om forbrugeren og hans forbrugstilbøjeligheder. Disse antagelser er simple, og da de har indflydelse på, hvordan forbrugeren vælger at forbruge, kan de have påvirket resultatet. De forskellige antagelser vil blive diskuteret i dette afsnit.

Antagelsen om, at pandemier og finanskriser ikke påvirker forbrugerens økonomi, er usandsynlig. Dette skyldes, at globale kriser kan have stor indflydelse på sandsynligheden for at blive arbejdsløs og dermed også sandsynligheden for at komme i arbejde. En uforudsigelig krise vil ofte medføre en stigning i arbejdsløsheden, hvorved der kommer større konkurrence på arbejdsmarkedet. Det vil derfor blive svære at beholde sit arbejde og komme i arbejde. Derudover vil en krise også påvirke ens forventninger til fremtiden, og dermed ens forbrugstilbøjeligheder således, at der forbruges mindre og spares mere op. På samme måde som en krise kan påvirke ens forbrug, kan en stor fremgang i økonomien også påvirke arbejdsløsheden og forbrugstilbøjeligheden. 

Da der i dette projekt kun tages udgangspunkt i en tidsperiode på et år, er det usandsynligt, at der opstår en økonomisk krise eller en stor fremgang i økonomien. Derfor kan der argumenteres for, at det ikke er nødvendigt at tage højde for dette. Af denne grund kan det antages, at en forbrugers arbejdsstatus kan beskrives med en homogen Markov-kæde i begrænsede tidsperioder. Hvis tidsperioden var længere, ville man kunne lade $\alpha$ og $\beta$ være tidsafhængige og på denne måde kan der tages højde for ændringer i arbejdsløshed under en krise. 

Det er antaget, at lønnen er fast og at der ikke kan forekomme skatteændringer. Dermed er hans disponible indkomst fast i de beslutningstidspunkter, han er i arbejde. Typisk vil forbrugerens disponible indkomst variere afhængig af varierende løn og skat. Derfor ville problemet have været mere realistisk hvis man havde tilføjet en varierende disponibel indkomst.  

I basis-problemet, når indlånsrenten varieres, kan det aflæses på resultaterne, at forbrugeren forbruger mere, når indlånsrenten stiger. Dette ses især, hvis man sammenligner $\gamma_i=0$ og $\gamma_i=0.1$ i \autoref{fig:renter}, da forbruget cirka er tre gange så stort ved den høje indlånsrente til sidste beslutningstidspunkt. For forbrugeren er dette den optimale strategi, når han har en belønningsfunktion givet ved $r_t(a) = \sqrt{a}$. Hvis denne belønningsfunktion anvendes ved en længere tidsperiode, vil forbruget på samme måde, som det kan ses i \autoref{fig:renter}, fortsætte med at stige. Dette er kun realistisk til et vist beslutningstidspunkt, da enhver forbruger har en øvre grænse for, hvor meget de vil forbruge. Når forbrugeren kommer tættere på denne grænse, vil han begynde at spare mere op. Dette ville ikke være tilfældet med belønningsfunktionen $r_t(a) = \sqrt{a}$, og derfor ville denne belønningsfunktion kun være realistisk i en begrænset tidperiode.

I basisproblemet anvendes en diskretiseringsfaktor på $\Delta = 0.1$, og i det udvidede problem er $\Delta = 1$. Det gælder, at en lavere diskretiseringsfaktor vil give bedre resultater, som resultat af mere præcise tal. Da problemet vokser eksplosivt, når $\Delta$ bliver mindre, var det for det udvidede problem ikke muligt at simulere problemet for et $\Delta$ mindre end $1$. Dette har resulteret i, at der er spring i dataet for det udvidede problem. Dette ses for eksempel i dataet for \autoref{fig:lån_gamma_u}, hvor kurverne ikke er strengt voksende som i \autoref{fig:renter}. Disse spring ville forsvinde ved en lavere diskretiseringsfaktor.


Da problemet vokser eksplosivt i dets størrelse, er det nødvendigt at begrænse faktorerne, hvilket ikke er ideelt. I det udvidede problem analyseres det blandt andet, hvordan forbrugeren vil låne, når hans mulige maksimale lån hver måned stiger. Det er her kun muligt at analysere op til $L=4$, hvis de andre faktorer skal forblive det samme. Det kan aflæses, at for $L \leq 4$ vil forbrugeren låne mest muligt til første beslutningstidspunkt. Ved at $L$ er begrænset, er det dermed ikke muligt at kunne konkludere, om der ville være et maksimalt beløb for forbrugeren at låne til første beslutningstidspunkt. Forbrugeren vil ikke låne det maksimale hvis han forventede at han ikke kunne betale det tilbage. Da det i det udvidede problem antages, at forbrugeren til ethvert beslutningstidspunkt skal forbruge, hvad han låner, er det sikkert, at der vil være en øvre grænse for, hvor meget han vil låne. Denne grænse afhænger af opsparingen, lønnen og renterne og kunne ikke bestemmes, da det kun var muligt at simulere problemet op til $L = 4$.

Der er altså flere antagelser og begrænsede faktorer, der er med til at gøre modellen mere urealistisk.

