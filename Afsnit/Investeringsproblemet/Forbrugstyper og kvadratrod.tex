%I dette afsnit redegøres for nogle af de forbrugstilbøjeligheder, der kan forekomme, dette i forhold til den problemstilling, der tidligere er blevet stillet.

For at analysere forbrugstilbøjelighederne kræver det at anvende psykologiske aspekter i økonomiske sammenhænge. Altså er det muligt at beskrive en forbrugers adfærd som en belønningsfunktion. 

%På denne måde kan man beskrive en forbrugers adfærd matematisk.
%For at gå mere i dybden med tilbøjeligheder, kræver det dog at kigge på psykologiske aspekter, men der tages udgangspunkt i de økonomiske sammenhænge.


\subsection{Utålmodig forbruger}
En utålmodig forbruger foretrækker at forbruge nu fremfor senere. Den utålmodige forbruger er derfor ikke interesseret i den belønning, han kan opnå i fremtiden. Derfor vil en indlånsrente ikke have en indflydelse på forbrugstilbøjeligheden for en utålmodig forbruger. Herudover vil forbrugstilbøjeligheden ikke blive påvirket af forbrugerens arbejdsstatus. Hvis det antages, at der ikke er mulighed for at låne, risikerer den utålmodige forbruger i nogle beslutningstidspunkter dermed ikke at have noget at forbruge. Der kan således være relativt store udsving i forbruget i forhold til, om forbrugeren er i arbejde eller er arbejdsløs. 

%Da den utålmodige forbrugers forbrugstilbøjelighed er den samme uafhængig af arbejdsstatus og renter i banken, kan han risikere i nogle beslutningstidspunkter ikke at have noget at forbruge, hvis der ikke er mulighed for at låne. Der kan således være relativt store udsving i forbruget i forhold til, om forbrugeren er i arbejde eller er arbejdsløs. 



%Hvis den utålmodige forbruger bliver arbejdsløs og ikke har mulighed for at låne, vil hans forbrugstilbøjeligheder altså ikke blive ændret. Der er derfor mulighed for, at den utålmodige forbruger ikke har forbrug i nogle beslutningstidspunkter.
%Det er således muligt at beregne den nytte, som en utålmodig forbruger vil gå glip af.
%Hvis marginalnytten er konstant for hvert forbrug, vil forbrugeren være utålmodig. 

%%Dette motiverer begrebet "cost-benefit". Marginal nytte er ændringen i belønning som funktion af forbruget. Jo mindre marginal nytte, jo mere utålmodig siges forbrugeren at være, da forbrugerens marginal omkostninger først vil overstige marginal nytten efter et tilstrækkelig stort forbrug.

%Dette skyldes, at forbrugeren får mere nytte af at forbruge én enhed mere end en forbruger der ikke er utålmodig. 

%%Det kan ikke altid svare sig for en forbruger at spare op, eksempelvis hvis renten er relativt lav.

Belønningsfunktionen for en utålmodig forbruger kan eksempelvis være givet ved følgende
\begin{align}\label{eq:utolmodig}
    r_t(S, a) = \frac{a}{(S+1)t} \ ,
\end{align}
hvor $S$ er en opsparing, $a$ er en beslutning, og $t$ er et beslutningstidspunkt. Hvis belønningen er givet ved \eqref{eq:utolmodig} er det forventeligt ud fra \eqref{eq:u_t_max}, at alt forbruges til hvert beslutningstidspunkt, da dette giver den største belønning. Dette skyldes, at tælleren maksimeres og nævneren minimeres ved at forbruge alt til hvert beslutningstidspunkt. 

%Da belønningen er givet ved beslutningen divideret med opsparingen multipliceret med beslutningstidspunktet vil det gælde, at det samme forbrug vil give en større belønning i første beslutningsperiode end i de følgende. Derfor er det forventeligt, at alt forbruges i hver tidperiode, da dette giver den største belønning.  

%Ved at anvende denne belønningsfunktion, vil den optimale strategi, jævnfør \eqref{eq:u_t_max}, være at forbruge alt i hver tidsperiode. - 

%$r_t > r_{t+1}$
%Det ses jævnfør \eqref{eq:u_t_max}, at den optimale strategi vil være at forbruge alt i hver tidsperiode. Det gælder desuden, at beløningen er beskrevet direkte af forbruget. 

\subsection{Tålmodig forbruger}
En tålmodig forbruger er indifferent i forhold til, om han skal forbruge nu eller senere. Hvis indlånsrenten i banken er nul, vil den tålmodige forbruger derved opnå den samme belønning uafhængigt af strategien. Hvis renten i banken er positiv, vil han forbruge det hele til sidste beslutningstidspunkt. En tålmodig forbruger er derfor kun interesseret i det samlede forbrug over hele tidsperioden. Dermed vil en tålmodig forbrugers arbejdsstatus ikke påvirke forbrugstilbøjeligheden. Det vil derudover gælde, at hvis der er mulighed for at låne, så vil den tålmodige forbruger låne mest muligt så længe indlånsrenten er højere end udlånsrenten. 

Følgende belønningsfunktion er et eksempel på en tålmodig forbruger
\begin{align}\label{eq:tolmodig}
    r_t(a) = a,
\end{align}
hvor $a$ er en beslutning, og $t$ er et beslutningstidspunkt. Eftersom belønningen er givet ved \eqref{eq:tolmodig}, er det forventeligt ud fra \eqref{eq:u_t_max}, at såfremt indlånsrenten i banken er nul, vil han opnå samme belønning uafhængig af strategien, han vælger.
Hvis renten derimod er større end nul, vil det gælde ud fra \eqref{eq:u_t_max}, at den maksimale belønning opnås ved at forbruge alt til sidste beslutningstidspunkt. 

\subsection{Fornuftig forbruger}
Eftersom alle forbrugere i virkeligheden har et basisforbrug, vil det være urealistisk kun at forbruge til sidste beslutningstidspunkt. Samtidig vil de fleste forbrugeres forbrugstilbøjeligheder påvirkes af renten. Dermed er det usandsynligt, at en forbruger enten er helt utålmodig eller tålmodig. Det antages derfor, at den gennemsnitlige forbruger vil fordele sit forbrug mere ligeligt mellem beslutningstidspunkterne. På denne måde vil der blandt andet ikke være måneder, hvor der ikke forbruges noget. Denne slags forbruger betegnes som en fornuftig forbruger. En belønningsfunktion for en fornuftig forbruger kan eksempelvis være givet ved følgende
\begin{align}\label{eq:belønning_fornuftig_forbruger}
    r_t(a) = \sqrt{a}
\end{align}
hvor $a$ er en beslutning, og $t$ er et beslutningstidspunkt. I det følgende afsnit analyseres det, hvordan renten, sandsynligheden for at være arbejdsløs og muligheden for lån påvirker forbruget og den samlede belønning for en fornuftig forbruger.




%En forbruger med en mindre grad af disse forbrugstilbøjeligheder vil derfor stadig kaldes tålmodig.
%De ovenstående er ret yderlige derfor vil der analyseres på en forbruger som er "mere" fornuftig. 

%En fornuftig forbruger foretrækker at fordele sit forbrug ligeligt mellem alle beslutningstidspunkter. Hvis renten i banken er nul, vil den fornuftige forbruger opnå den største belønning, hvis forbruget er det samme til hvert beslutningstidspunkt. Såfremt renten er positiv forventes det, at den mest ligelige fordeling af forbruget derimod er stigende, eftersom den fornuftige forbruger er tvunget til at forbruge alt til sidste beslutningstidspunkt.

%Dermed vil en fornuftig forbruger blive mere tålmodig, jo højere renten er. 

%Dermed forventes en fornuftig forbruger at have mere tålmodige forbrugstilbøjeligheder ved en høj rente.

%En fornuftig forbruger foretrækker at fordele sit forbrug ligeligt mellem alle beslutningstidspunkter. 

%Den fornuftige forbruger er derfor interesseret i have have det samme forbrug til hver beslutningstidspunkt.

%hvis han fordeler forbruget ligeligt.


%Hvis renten er positiv, vil forbrugeren være mere tålmodig som funktion af rentens størrelse. 


%Hvis renten er positiv, vil forbrugeren ikke længere distribuere sit forbrug ligeligt. Dette skyldes, at den mest ligelige 


%hvis renten er positiv, vil den mest ligelige fordeling af forbruget være stigende, eftersom den fornuftige forbruger er tvunget til at forbruge alt til sidste beslutningstidspunkt. --- 

% % % % % %

% For en fornuftig forbruger har renten en indflydelse på forbrugstilbøjelighederne, men i mindre grad end for den tålmodige forbruger. 






