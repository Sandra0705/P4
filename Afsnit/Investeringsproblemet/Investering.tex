En investor ønsker at maksimere sin nytte over en given periode. For at maksimere sin nytte skal investoren beslutte, hvornår det er henholdvis bedst at spare op og forbruge. Investoren kan til hvert beslutningstidspunkt vælge at bruge af sin opsparing eller lade pengene stå i banken. Disse beslutninger afhænger derudover også af arbejdsstatus, opsparingen og renten i banken.

Investoren skal tage en beslutning hver måned og derfor er indeksmængden, $T$, givet ved $t = 0,1,\ldots, N$, hvor hvert $t$ repræsenterer en måned. Mængden af mulige beslutninger, $a_t$, til beslutningstidspunkt, $t$, kaldes beslutningsrummet, $A_t$, hvor hver beslutning angiver et forbrug. Lad $S$ være opsparingen, $S_t$ være opsparingen til beslutningstidspunkt $t$ samt $S_0$ være kendt. Derudover gælder det at opsparingen til beslutningstidspunktet, $t+1$, er givet ved
\begin{align*}
    S_{t+1}=(1+r)(S_t-a_t)+I \cdot \xi_t,
\end{align*}
hvor $r$ er renten i banken og $I>0$ er investorerens faste indkomst. Den stokastiske proces $t\mapsto \xi_t$ beskriver investorens arbejdsstatus, hvor $\xi_t=1$, hvis investoren er i arbejde og $\xi_t=0$, hvis han er arbejdsløs. Det antages, at investorens arbejdsstatus kan beskrives som følgende Markov kæde
\begin{align*}
    &P(\xi_{t+1}=0|\xi_t=1)=1-P(\xi_{t+1}=1|\xi_t=1)=p \text{ og }\\
    &P(\xi_{t+1}=0|\xi_t=0)=1-P(\xi_{t+1}=1|\xi_t=0)=q,
\end{align*}
hvor $p$ angiver sandsynligheden for, at investoren mister sit arbejde i næste måned givet, at han er i arbejde i den nuværende måned. Derudover angiver $q$ sandsynligeden for at han er arbejdsløs i næste måned givet, at han er ikke er i arbejde i den nuværende måned. 


% \begin{table}[!hbt] \centering
% \caption{Variabler for investeringsproblemet}
% \begin{tabular}{@{}l|l@{}}
% \toprule
% Variabler & Forklaring                     \\ \midrule
% $t$         & Startidspunkt                \\
% $N$         & Sluttidspunkt                \\
% $S_t$       & Opsparing til tiden t        \\
% $S$         & Beslutningsrum               \\
% $r$         & Rente                        \\
% $A_t$       & Forbrug                      \\
% $I$         & Indkomst                     \\
% $\xi_t$     & $\begin{cases}1\ \text{Hvis i arbejde}\\0\ \text{Ellers}\end{cases}$ \\ \bottomrule
% \end{tabular}
% \end{table}

\section{Basis Problemet}
Ovenstående system kan beskrives ved følgende overgangsgraf\\
\begin{tikzpicture}[node distance=2cm and 1cm,>=stealth',auto, every place/.style={draw}]
    \node [place] (S1) {Ikke i arbejde};
    \coordinate[node distance=1.1cm,left of=S1] (left-S1);
    \coordinate[node distance=1.1cm,right of=S1] (right-S1);


    \node [place] (S2) [right=of S1] {\phantom{---}I arbejde \phantom{---}};

    \path[->] (S1) edge [bend left] node {1-q} (S2);
    \path[->] (S1) edge [loop left] node {q} (s1);
    \path[->] (S2) edge [bend left] node {p} (S1);
    \path[->] (S2) edge [loop right] node {1-p} ();
\end{tikzpicture}

Det vil sige, at overgangsmatricen er givet ved:\\
\begin{align*}
    P=\begin{bmatrix}q & 1-q\\ p & 1-p\end{bmatrix}
\end{align*}

Det antages, at nytten er ækvivalent med forbruget. Det vil sige, at markedet er perfekt. Altså er der et vist forbrug til tidspunkter mellem 0 og $N-1$ og der forbruges resten ($S_N$) til tidspunkt $N$. Med andre ord, så er nytten, $r_t(S_t,A_t)=A_t$ og $r_N(S_N)=S_N$.


\begin{align*}
    W=\sum_{t=1}^{N-1} r_t(s_t,a_t)+ r_N(s_N)
\end{align*}

Lad nyttefunktionen være givet ved
\begin{align*}
    \{U\}_t^N=\begin{cases}A_t\ \text{ for } t<N\\ S_N \text{ for } t=N \end{cases}.
\end{align*}
Det antages, desuden, at indkomsten kommer sidst på måneden og den kan dermed ikke bruges i den indeværende måned. Beslutningsrummet er givet ved:
\begin{align*}
    S=[0,S_t]
\end{align*}

Dermed ønskes at finde strategi ($\pi$), der maksimerer
\begin{align*}
    E^\pi \left[\sum_{t=1}^N U_t(A_t)\right]
\end{align*}





