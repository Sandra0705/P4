En investor ønsker at maksimere sin \textit{nytte} over en given periode.
Lad tidshorisonten være givet ved $N$. Investorens opsparing til tiden $t$ betegnes ved $S_t$, sådan at $S_0$ er givet. Investoren kan enten forbruge eller lade pengene akkumulere til en rente, $r$. Lad $A_t$ være forbrug og indkomst være $I>0$. I dette afsnit konstrueres en MDP over disse beslutninger, med henblik på maksimering af nytte. Følgende tabel opsummerer disse variabler.


\begin{table}[!hbt]
\caption{Variabler for investeringsproblemet}
\begin{tabular}{@{}l|l@{}}
\toprule
Variabler & Forklaring                     \\ \midrule
$N,t$       & Tidspunkter                  \\
$S_t$       & Opsparing til tiden t        \\
$S$         & Beslutningsrum               \\
$r$         & Rente                        \\
$A_t$       & Forbrug                      \\
$I$         & Indkomst                     \\
$\xi_t$     & $\begin{cases}1\ \text{Hvis i arbejde}\\0\ \text{Ellers}\end{cases}$ \\ \bottomrule
\end{tabular}
\end{table}




Opsparing til tiden $t+1$ er dermed givet ved
\begin{align*}
    S_{t+1}=(1+r)(S_t-A_t)+I\dot \xi_t
\end{align*}


\section{Basis Problemet}
Lad $p$ angive den betingede sandsynlighed for, at han mister sit arbejde i næste måned og lad $q$ være den betingede sandsynlighed for, at han ikke er i arbejde næste måned, givet han ikke er i arbejde. 
Det vil sige:
\begin{align*}
    P(\xi_{t+1}=0|\xi_t=1)=1-P(\xi_{t+1}=1|\xi_t=1)=p\\
    P(\xi_{t+1}=0|\xi_t=0)=1-P(\xi_{t+1}=1|\xi_t=0)=q
\end{align*}



Dette medfører følgende overgangsgraf\\
\begin{tikzpicture}[node distance=2cm and 1cm,>=stealth',auto, every place/.style={draw}]
    \node [place] (S1) {Ikke i arbejde};
    \coordinate[node distance=1.1cm,left of=S1] (left-S1);
    \coordinate[node distance=1.1cm,right of=S1] (right-S1);


    \node [place] (S2) [right=of S1] {\phantom{---}I arbejde \phantom{---}};

    \path[->] (S1) edge [bend left] node {1-q} (S2);
    \path[->] (S1) edge [loop left] node {q} (s1);
    \path[->] (S2) edge [bend left] node {p} (S1);
    \path[->] (S2) edge [loop right] node {1-p} ();
\end{tikzpicture}

Det vil sige, at overgangsmatricen er givet ved:\\
\begin{align*}
    P=\begin{bmatrix}q & 1-q\\ p & 1-p\end{bmatrix}
\end{align*}

Lad os først antage, at nytten er ækvivalent med forbruget. Det vil sige, at markedet er perfekt. Det vil sige, at man har et vist forbrug for tidspunkter mellem 0 og $N-1$ og man forbruger resten ($S_N$) til tidspunkt $N$. Med andre ord, så er nytten, $r_t(S_t,A_t)=A_t$ og $r_N(S_N)=S_N$.


\begin{align*}
    W=\sum_{t=1}^{N-1} r_t(s_t,a_t)+ r_N(s_N)
\end{align*}

Lad nyttefunktionen være givet ved
\begin{align*}
    \{U\}_t^N=\begin{cases}A_t\ \text{ for } t<N\\ S_N \text{ for } t=N \end{cases}.
\end{align*}
Det antages, desuden, at indkomsten kommer sidst på måneden og den kan dermed ikke bruges i den indeværende måned. Beslutningsrummet er givet ved:
\begin{align*}
    S=[0,S_t]
\end{align*}

Dermed ønskes at finde strategi ($\pi$), der maksimerer
\begin{align*}
    E^\pi \left[\sum_{t=1}^N U_t(A_t)\right]
\end{align*}






\subsection{Udvidelse(Lån)}
Det antages, at forbrugeren i en lang periode har været foruden job (eller i form af et markant højere forbrug ift. opsparing/indkomst), derved kræves et lån.
Lad nu beslutningsrummet udvides, så det kan antage negative værdier i form af et lån.
Beslutningsrummet bliver derved
\begin{align*}
    S = [-L_t, S_t],
\end{align*}
hvor $L_t$ er det maksimale beløb, som investoren kan låne. Der gælder således følgende bibetingelser,
\begin{align*}
    \sum_{t=1}^{T-1} L_t \leq L_{max}
\end{align*}
og
\begin{align*}
    \sum_{t=1}^{T-1} L_t \leq S_T
\end{align*}

\textit{Når udlåns- og indlånsrenten er lig med hinanden? Hvad sker hvis de er forskellige? Hvordan påvirker processen $\xi's$ parametre, hvordan lån optages?}