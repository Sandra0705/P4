Dette kapitel er inspireret af \cite[s. 5-6]{"investeringsproblem"}.

I de forrige kapitler blev der redegjort for den nødvendige teori for at kunne opstille og analysere forbrugsproblemet. Forbrugsproblemet vil blive opstillet som en Markov-beslutningsproces, hvori der indgår en Markov-kæde. I dette afsnit anvendes Maple og Python til samtlige beregninger.


% I kapitlet om sandsynlighedteori blev der introduceret nogle grundlæggende begreber, som er blevet anvendt til at definere og analysere Markov-kæder i kapitel 3. Ved at kombinere teorien fra disse kapitler er det muligt at kunne beskrive den Markov-kæde, der senere vil blive opstillet. Derudover er der i kapitel 4 blevet redegjort for Markov-beslutningsprocesser, og hvordan der bestemmes en løsning til disse. Ved at anvende teorien om Markov-beslutningsprocesser er det muligt at opstille og "løse" forbrugsproblemet. Derudover vil Bellman-ligningerne og algoritme 4.4.1 blive anvendt til at bestemme den forventede maksimale belønning. For at kunne beskrive forbrugerens forbrugstilbøjelighed og forventede opførsel vil der i dette kapitel blive introduceret forskellige typer af forbrugere.

En forbruger vil maksimere sin belønning over en given tidsperiode. For at maksimere sin belønning skal forbrugeren beslutte, hvornår det er bedst at forbruge eller spare op. Forbrugeren kan til ethvert beslutningstidspunkt vælge at forbruge af sin opsparing eller lade sine penge stå i banken. 

Forbrugeren skal tage en beslutning hver måned, og derfor er indeksmængden givet ved $T=\{0, 1, \ldots, N\}$, hvor hvert beslutningstidspunkt, $t\in T$, repræsenterer en måned. 
Hver beslutning til beslutningstidspunkt, $t$, betegnes $a_t$ og angiver et forbrug. Mængden af mulige beslutninger til $t$ kaldes beslutningsrummet og betegnes $A_t$.
%Mængden af mulige beslutninger, $a_t$, til beslutningstidspunkt, $t$, kaldes beslutningsrummet, $A_t$, hvor hver beslutning angiver et forbrug.
Lad $S$ være opsparingen og $S_t$ være opsparingen til beslutningstidspunkt, $t$. Lad derudover startopsparingen, $S_0$, være kendt.

Den stokastiske proces $\bm \xi= (\xi_t : t \in T)$ beskriver forbrugerens arbejdsstatus, hvor $\xi_t=1$, hvis han er i arbejde og $\xi_t=0$, hvis han er arbejdsløs til beslutningstidspunktet, $t$. Det antages, at forbrugerens arbejdsstatus kan beskrives som en Markov-kæde med følgende overgangssandsynligheder
\begin{align}\label{eq:markov_kæde_problem}
\begin{split}
    &P(\xi_{t+1}=0|\xi_t=1)=1-P(\xi_{t+1}=1|\xi_t=1)=\alpha \text{ og }\\
    &P(\xi_{t+1}=0|\xi_t=0)=1-P(\xi_{t+1}=1|\xi_t=0)=\beta.
    \end{split}
\end{align}
Heraf angiver $\alpha$ sandsynligheden for, at forbrugeren mister sit arbejde i næste måned givet, at han er i arbejde i den nuværende måned. Derudover angiver $\beta$ sandsynligeden for, at han er arbejdsløs i næste måned givet, at han ikke er i arbejde i den nuværende måned. Det følger af \autoref{def:markov-egenskab}, at denne Markov-kæde er homogen, da overgangssandsynlighederne ikke afhænger af $t$. De ovenstående overgangssandsynligheder til Markov-kæden kan illustreres ved følgende overgangsgraf
\begin{figure}[H]
\begin{tikzpicture} [node distance=2cm and 1cm,>=stealth',auto, every place/.style={draw}]
    \node [place] (S1) {\phantom{..}$\xi_t = 0$\phantom{..}};
    \coordinate[node distance=1.1cm,left of=S1] (left-S1);
    \coordinate[node distance=1.1cm,right of=S1] (right-S1);


    \node [place] (S2) [right=of S1] {\phantom{..}$\xi_t = 1$\phantom{..}};5

    \path[->] (S1) edge [bend left] node {$1-\beta$} (S2);
    \path[->] (S1) edge [loop left] node {$\beta$} (s1);
    \path[->] (S2) edge [bend left] node {$\alpha$} (S1);
    \path[->] (S2) edge [loop right] node {$1-\alpha$} ();
\end{tikzpicture}
    \centering
    \caption{Overgangsgraf for arbejdsstatus.}
    \label{fig:overgangsgraf_for_arbejdsstatus}
\end{figure}

% \begin{center}
%     \begin{tikzpicture}
%       \draw[color=purple, fill=purple] (0.5, -8) circle (1.5);
%       \draw[color=yellow, fill=yellow] (-1,0) rectangle ++ (3, -8);
%       \draw[color=blue, fill=blue] (-1, 1) circle (2);
%       \draw[color=blue, fill=blue] (2,1) circle (2);
%     \end{tikzpicture}
% \end{center}
hvor den tilhørende overgangsmatrice er givet ved
\begin{align*}
    \P=\begin{bmatrix}\beta & 1-\beta\\ \alpha & 1-\alpha\end{bmatrix}.
\end{align*}
Ud fra overgangssandsynlighederne, \eqref{eq:markov_kæde_problem}, er det muligt for forbrugeren at vide, hvor sandsynligt det er at være i arbejde, og han kan dermed tage beslutninger ud fra disse sandsynligheder.

Forbrugerens maksimering af belønning er en Markov beslutningsproces, hvor forbrugeren er beslutningstageren, der skal tage bestemte beslutninger til hvert beslutningstidspunkt. Ved at tage disse beslutninger opnår forbrugeren nogle bestemte tilstande, som er givet ved en opsparing. Processen er Markov, da den fremtidige opsparing afhænger af den nuværende opsparing. Forbrugeren skal derudover tage beslutninger ud fra sandsynligheden for at være i arbejde, som er beskrevet ved den homogene Markov-kæde, muligheden for lån samt indlåns- og udlånsrenten. %Da det er en Markov beslutningsproces vides det fra \textbf{ref afsnit om optimering}, at investoren kan maksimere sin belønning ved \textbf{ref algo og bellman evt}.  
Da dette er en Markov beslutningsproces, vil forbrugeren modtage en belønning ud fra de beslutninger, han tager. Belønningsfunktionen er givet forskelligt afhængig af, hvilken type forbruger han er. Derfor vil der nu redegøres for de forbrugstilbøjeligheder, som vil blive anvendt. 


\section{Forbrugstilbøjeligheder}
%I dette afsnit redegøres for nogle af de forbrugstilbøjeligheder, der kan forekomme, dette i forhold til den problemstilling, der tidligere er blevet stillet.

For at analysere forbrugstilbøjelighederne kræver det at anvende psykologiske aspekter i økonomiske sammenhænge. Altså er det muligt at beskrive en forbrugers adfærd som en belønningsfunktion. 

%På denne måde kan man beskrive en forbrugers adfærd matematisk.
%For at gå mere i dybden med tilbøjeligheder, kræver det dog at kigge på psykologiske aspekter, men der tages udgangspunkt i de økonomiske sammenhænge.


\subsection{Utålmodig forbruger}
En utålmodig forbruger foretrækker at forbruge nu fremfor senere. Den utålmodige forbruger er derfor ikke interesseret i den mulige fremtidige belønning. Derfor vil en indlånsrente ikke have en indflydelse på forbrugstilbøjeligheden for en utålmodig forbruger. Herudover vil forbrugstilbøjeligheden ikke blive påvirket af forbrugerens arbejdsstatus. Hvis det antages, at der ikke er mulighed for at låne, risikerer den utålmodige forbruger i nogle beslutningstidspunkter dermed ikke at have noget at forbruge. Der kan således være relativt store udsving i forbruget i forhold til, om forbrugeren er i arbejde eller er arbejdsløs. 

%Da den utålmodige forbrugers forbrugstilbøjelighed er den samme uafhængig af arbejdsstatus og renter i banken, kan han risikere i nogle beslutningstidspunkter ikke at have noget at forbruge, hvis der ikke er mulighed for at låne. Der kan således være relativt store udsving i forbruget i forhold til, om forbrugeren er i arbejde eller er arbejdsløs. 



%Hvis den utålmodige forbruger bliver arbejdsløs og ikke har mulighed for at låne, vil hans forbrugstilbøjeligheder altså ikke blive ændret. Der er derfor mulighed for, at den utålmodige forbruger ikke har forbrug i nogle beslutningstidspunkter.
%Det er således muligt at beregne den nytte, som en utålmodig forbruger vil gå glip af.
%Hvis marginalnytten er konstant for hvert forbrug, vil forbrugeren være utålmodig. 

%%Dette motiverer begrebet "cost-benefit". Marginal nytte er ændringen i belønning som funktion af forbruget. Jo mindre marginal nytte, jo mere utålmodig siges forbrugeren at være, da forbrugerens marginal omkostninger først vil overstige marginal nytten efter et tilstrækkelig stort forbrug.

%Dette skyldes, at forbrugeren får mere nytte af at forbruge én enhed mere end en forbruger der ikke er utålmodig. 

%%Det kan ikke altid svare sig for en forbruger at spare op, eksempelvis hvis renten er relativt lav.

Belønningsfunktionen for en utålmodig forbruger kan eksempelvis være givet ved følgende
\begin{align}\label{eq:utolmodig}
    r_t(S_t, a_t) = \frac{a_t}{(S_t+1)^t} \ ,
\end{align}
hvor $S_t$ er en opsparing, $a_t$ en beslutning og $t$ et beslutningstidspunkt. Hvis belønningen er givet ved \eqref{eq:utolmodig} er det forventeligt ud fra \eqref{eq:u_t_max}, at alt forbruges til hvert beslutningstidspunkt. Dette forventes, da det er denne forbrugstendens, der giver den forventede maksimale belønning, hvilket skyldes, at tælleren maksimeres og nævneren minimeres, når alt forbruges til hvert beslutningstidspunkt.

%da dette giver den største belønning. Dette skyldes, at tælleren maksimeres og nævneren minimeres ved at forbruge alt til hvert beslutningstidspunkt. 

%Da belønningen er givet ved beslutningen divideret med opsparingen multipliceret med beslutningstidspunktet vil det gælde, at det samme forbrug vil give en større belønning i første beslutningsperiode end i de følgende. Derfor er det forventeligt, at alt forbruges i hver tidperiode, da dette giver den største belønning.  

%Ved at anvende denne belønningsfunktion, vil den optimale strategi, jævnfør \eqref{eq:u_t_max}, være at forbruge alt i hver tidsperiode. - 

%$r_t > r_{t+1}$
%Det ses jævnfør \eqref{eq:u_t_max}, at den optimale strategi vil være at forbruge alt i hver tidsperiode. Det gælder desuden, at beløningen er beskrevet direkte af forbruget. 

\subsection{Tålmodig forbruger}
En tålmodig forbruger er indifferent i forhold til, om han skal forbruge nu eller senere. Hvis indlånsrenten i banken er nul, vil den tålmodige forbruger derved opnå den samme forventede maksimale belønning uafhængigt af strategien. Hvis indlånsrenten i banken er positiv, vil han forbruge det hele til sidste beslutningstidspunkt. En tålmodig forbruger er derfor kun interesseret i det samlede forbrug over hele tidsperioden. Dermed vil en tålmodig forbrugers arbejdsstatus ikke påvirke forbrugstilbøjeligheden. Det vil derudover gælde, at hvis der er mulighed for at låne, så vil den tålmodige forbruger låne mest muligt, så længe indlånsrenten er højere end udlånsrenten. 

Følgende belønningsfunktion er et eksempel på en tålmodig forbruger
\begin{align}\label{eq:tolmodig}
    r_t(a_t) = a_t,
\end{align}
hvor $a_t$ er en beslutning, og $t$ et beslutningstidspunkt. Eftersom belønningen er givet ved \eqref{eq:tolmodig}, er det forventeligt ud fra \eqref{eq:u_t_max}, at såfremt indlånsrenten i banken er nul, vil han opnå samme forventede maksimale belønning uafhængig af strategien.
Hvis indlånsrenten derimod er større end nul, vil det gælde ud fra \eqref{eq:u_t_max}, at den maksimale belønning opnås ved at forbruge alt til sidste beslutningstidspunkt. 

\subsection{Fornuftig forbruger}
Eftersom alle forbrugere har et basisforbrug, vil det være urealistisk kun at forbruge til sidste beslutningstidspunkt. Samtidig vil de fleste forbrugere have en forbrugstilbøjelighed, der påvirkes af indlånsrenten. Dermed er det usandsynligt, at en forbruger enten er helt utålmodig eller tålmodig. Det antages derfor, at den gennemsnitlige forbruger fordeler sit forbrug mere ligeligt mellem beslutningstidspunkterne. På denne måde vil der blandt andet ikke være måneder, hvor der ikke forbruges. Denne type af forbruger betegnes som en fornuftig forbruger. En belønningsfunktion for en fornuftig forbruger kan eksempelvis være givet ved følgende
\begin{align}\label{eq:belønning_fornuftig_forbruger}
    r_t(a_t) = \sqrt{a_t}
\end{align}
hvor $a_t$ er en beslutning, og $t$ et beslutningstidspunkt. I det følgende afsnit analyseres det, hvordan renten, sandsynligheden for at være arbejdsløs og muligheden for lån påvirker forbruget og den forventede maksimale belønning for en fornuftig forbruger.




%En forbruger med en mindre grad af disse forbrugstilbøjeligheder vil derfor stadig kaldes tålmodig.
%De ovenstående er ret yderlige derfor vil der analyseres på en forbruger som er "mere" fornuftig. 

%En fornuftig forbruger foretrækker at fordele sit forbrug ligeligt mellem alle beslutningstidspunkter. Hvis renten i banken er nul, vil den fornuftige forbruger opnå den største belønning, hvis forbruget er det samme til hvert beslutningstidspunkt. Såfremt renten er positiv forventes det, at den mest ligelige fordeling af forbruget derimod er stigende, eftersom den fornuftige forbruger er tvunget til at forbruge alt til sidste beslutningstidspunkt.

%Dermed vil en fornuftig forbruger blive mere tålmodig, jo højere renten er. 

%Dermed forventes en fornuftig forbruger at have mere tålmodige forbrugstilbøjeligheder ved en høj rente.

%En fornuftig forbruger foretrækker at fordele sit forbrug ligeligt mellem alle beslutningstidspunkter. 

%Den fornuftige forbruger er derfor interesseret i have have det samme forbrug til hver beslutningstidspunkt.

%hvis han fordeler forbruget ligeligt.


%Hvis renten er positiv, vil forbrugeren være mere tålmodig som funktion af rentens størrelse. 


%Hvis renten er positiv, vil forbrugeren ikke længere distribuere sit forbrug ligeligt. Dette skyldes, at den mest ligelige 


%hvis renten er positiv, vil den mest ligelige fordeling af forbruget være stigende, eftersom den fornuftige forbruger er tvunget til at forbruge alt til sidste beslutningstidspunkt. --- 

% % % % % %

% For en fornuftig forbruger har renten en indflydelse på forbrugstilbøjelighederne, men i mindre grad end for den tålmodige forbruger. 








\section{Basis-problemet}
I basis-problemet analyseres det, hvordan arbejdsstatus og indlånsrenten påvirker forbruget og den forventede maksimale belønning for forbrugeren. Det antages, at forbrugeren er fornuftig, og dermed er belønningsfunktionen givet ved \eqref{eq:belønning_fornuftig_forbruger}. 
%Det antages, at investoren har mulighed for at få afkast på sin opsparing.
%Det gælder, at opsparingen til beslutningstidspunktet, $t+1$, afhænger af den forrige opsparing således at
%Altså vil den fremtidige opsparing afhænge af den forrige opsparing, indlånsrenten, forbrug, løn og arbejdsstatus.

Opsparingen til beslutningstidspunktet, $t+1$, afhænger af den forrige opsparing, indlånsrenten, forbrug, løn og arbejdsstatus, og er givet ved
%Det gælder, at opsparingen til beslutningstidspunktet, $t+1$, er givet ved
\begin{align}\label{st+1}
    S_{t+1}=(1+\gamma_i)(S_t-a_t)+I \cdot \xi_t,
\end{align}
hvor $\gamma_i$ er indlånsrenten i banken, og $I>0$ er forbrugererens faste indkomst.

Til hver måned er det muligt for forbrugeren at forbruge af hans opsparing til dette beslutningstidspunkt. Dermed er beslutningsrummet givet ved
\begin{align*}
    A_t(S_t) = \left[0, S_t\right].
\end{align*}
%
I basis-problemet antages det, at opsparingen til beslutningstidspunkt, $t+1$, er givet ved \eqref{st+1}, overgangssandsynlighederne for at være i arbejde er givet ved \eqref{eq:markov_kæde_problem} med tilstandsrummet $\Xi = \{0,1\}$, og belønningsfunktionen er givet ved \eqref{eq:belønning_fornuftig_forbruger}.

Den optimale belønning undersøges for en tidsperiode på et år. Det antages, at startopsparingen er 100, og lønnen er 10. Altså gælder det, at
\begin{align}\label{damdamdam..}
\begin{split}
    N &= 11,\\
    S_0 &= 100,\\
    I &= 10,\\
    \Delta &= 0.1,
    \end{split}
\end{align}
hvor $\Delta$ er den diskretisering, som anvendes på beslutningsrummet, $A_t$. Dermed kan der forbruges for hver $\Delta>0$ enhed, forbrugeren har, og der konstrueres en endelig mængde af beslutninger i beslutningsrummet. Derudover antages det, at forbrugeren er i arbejde til begyndelsestidspunktet, $t=0$. Da Python indekserer fra 0, er $N=11$. Bemærk, at der for hver simulering anvendes $\alpha$ og $\beta$ til at bestemme arbejdsstatus til hvert beslutningstidspunkt. Derved kan forbruget variere alt efter, hvor meget forbrugeren har været i arbejde. 

%Da der ønskes at bestemme forbruget over den gennemsnitlige forventede opførsel (af en forbruger), beregnes den gennemsnitlige optimale strategi for 10000 simuleringer af forbrugere.



%Altså er arbejdsstatus til ethvert tidspunkt gennemsnitlig, hvilket medfører at alle faktorer er gennemsnitlige. 


% Gennemsnittet (og ikke den "sande" bedste tingtang)
% Fejl ved at det er et endeligt antal simuleringer

% Skriv noget med hvordan dette påvirker resultater

Af denne grund tages den gennemsnitlige optimale strategi over 10000 simuleringer af forbrugere. Dermed er arbejdsstatus over tid, samt strategien også gennemsnitlig, hvilket medfører, at den forventede maksimale belønning er gennemsnitlig. 

Ud fra de ovenstående antagelser følger det af \autoref{prop:markov_det_strategi}, at der eksisterer en optimal Markov deterministisk strategi, og af \eqref{eq:v_N=sup} er den forventede maksimale belønning givet ved
\begin{align*}
    v_N^{*} (S_0) = \sup_{\Psi\in \Pi^{MD}}\left( E^{\Psi}_{S_0}\left[\sum_{t=1}^{N-1} r_t(a) + r_N(a)\right] \right).
\end{align*}
Den forventede maksimale belønning bestemmes derfor ved brug af \eqref{eq:u_N}, \eqref{eq:u_t_max} og \autoref{Algoritme1} og er implementeret i Python, se \autoref{Bilag:python_kode}.


\subsection[Variation af  \texorpdfstring{$\alpha$}{alpha}]{Variation af \bm{$\alpha$}}
 
For at analysere hvordan sandsynligheden for at være arbejdsløs påvirker forbruget og den forventede maksimale belønning, bestemmes disse for forskellige $\alpha$- og $\beta$-værdier, mens de andre værdier fastholdes som i \eqref{damdamdam..}, og indlånsrenten er lig $0$. De forskellige værdier for $\alpha$ og $\beta$ vil være følgende
\begin{align*}
    \alpha = 0.1 &\text{ og } \beta = 0.5,\\
    \alpha = 0.5 &\text{ og } \beta = 0.5,\\
    \alpha = 0.9 &\text{ og } \beta = 0.5.
\end{align*}
Før resultaterne for de optimale strategier og tilhørende forventede maksimale belønninger præsenteres, vil Markov-kæden, som beskriver forbrugerens arbejdsstatus, blive analyseret for de forskellige værdier for $\alpha$ og $\beta$.

Først vælges $\alpha=0.1$ og $\beta=0.5$, og dermed er alle overgangssandsynlighederne i Markov-kæden positive. Markov-kædens $t$'te trins overgangsmatrice bestemmes ved brug af \autoref{sæt:P(n)Pn}.
\begin{align}
    \P^t=\begin{bmatrix}0.5 & 0.5\\ 0.1 & 0.9\end{bmatrix}^t &= \begin{bmatrix}0.167+0.833\cdot 0.4^t & 0.833-0.833\cdot 0.4^t\\ 0.167-0.167\cdot 0.4^t & 0.833+0.167\cdot 0.4^t\end{bmatrix}. \label{eq:p_t.p0.1}
    \intertext{For $t\to\infty$ har Markov-kæden følgende overgangsmatrice}
    \lim_{t \to \infty} \P^t &=\lim_{t \to \infty} \begin{bmatrix}0.5 & 0.5\\ 0.1 & 0.9\end{bmatrix}^t = \begin{bmatrix}0.167 & 0.833\\ 0.167 & 0.833\end{bmatrix}.\label{grænse_for_P}
\end{align}
Af \autoref{def:tilgængelighed} er de to tilstande tilgængelige fra hinanden, da $t$'te trins overgangssandsynlighederne er strengt positive. Dermed følger det, at de to tilstande kommunikerer. Da alle tilstande i tilstandsrummet kommunikerer, er Markov-kæden ureducerbar af \autoref{def:ureducerbar}. 

Det skal bestemmes, om Markov-kæden har en invariant fordeling, $\bm \pi$. Af \autoref{sæt:den_vi_ikke_beviser} gælder det, at hvis Markov-kæden er ureducerbar og har en positiv tilbagevendende tilstand, så eksisterer der en entydig invariant fordeling. Derfor undersøges det, om Markov-kæden har en positiv tilbagevendende tilstand. 

For at gøre dette bestemmes det først, om tilstanden $\xi_t = 0$ er tilbagevendende eller forbigående. Dette gøres ved at beregne følgende sum
\begin{align*}
    \sum_{t=1}^\infty p_{00}^{(t)} = \sum_{t=1}^\infty 0.167+0.833\cdot 0.4^t = \infty.
\end{align*}
Af \autoref{tilbagevendende} følger det, at tilstanden $\xi_t = 0$ er tilbagevendende, og af
\autoref{kor:enten_forbigå_eller_tilbagevend} er alle tilstande i Markov-kæden dermed tilbagevendende.

Herefter undersøges det, om tilstanden $\xi_t = 0$ er positiv tilbagevendende. For at bestemme om en tilstand er positiv tilbagevendende, beregnes den forventede værdi af antallet af skridt, før kæden vender tilbage til tilstanden. Dette gøres ved at anvende \autoref{def:betinget_forventet_værdi_af_diskrete_tilfældige_variabler}.
 \begin{align*}
     E_0[\tau_{0}|\xi_0=0]&=\sum_{i\in \N} iP(\tau_{0}=i|\xi_0=0)\\
     &=0.5+2(1-0.5)0.1+3(1-0.5)(1-0.1)0.1\\
     &\phantom{=} +4(1-0.5)(1-0.1)^20.1+5(1-0.5)(1-0.1)^30.1+\ldots\\
     &=0.5 + \sum_{2\leq i \in \N } i(1-0.5)(1-0.1)^{i-2}0.1 = 6.
 \end{align*}
Da $E_0[\tau_{0}|\xi_0=0]=6<\infty$, følger det af \autoref{def:positiv_null_tilbagevendende}, at tilstanden, $\xi_t = 0$, er positiv tilbagevendende.

Da Markov-kæden er ureducerbar og har en positiv tilbagevendende tilstand, følger det af \autoref{sæt:den_vi_ikke_beviser}, at der eksisterer en entydig invariant fordeling, $\bm \pi$.
Derudover gælder det, at alle tilstandene er positiv tilbagevendende.

%at da Markov-kæden er ureducerbar og har en positiv tilbagevendende tilstand, så eksisterer der en entydig invariant fordeling, $\bm \pi$. Derudover gælder det, at alle tilstandene er positiv tilbagevendende.

For at bestemme om Markov-kæden konvergerer mod den invariante fordeling, undersøges det, om Markovkæden er aperiodisk. Såfremt $d(i)=1$, følger det jævnfør \autoref{def:aperiodisk}, at Markov-kæden er aperiodisk. Da begge tilstande i Markov-kæden har en strengt positiv $t$'te trins overgangssandsynlighed til sig selv, gælder det, at de er aperiodiske. 

Eftersom Markov-kæden er ureducerbar, aperiodisk og positiv tilbagevendende, følger det af \autoref{sæt:konvergens_for_markov}, at den konvergerer mod den invariante fordeling, der er givet ved 
\begin{align}\label{eq:pi_p,q}
    \bm \pi = \begin{bmatrix} 0.167 & 0.833\end{bmatrix}.
\end{align}
%Altså vil sandsynligheden for, at forbrugeren er i arbejde være beskrevet ved ovenstående sandsynlighedsfordeling for $t \to \infty$. Der er dermed større sandsynlighed for, at forbrugeren er i arbejde, end at han er arbejdsløs for en uendelig tidsperiode. 
For $t \to \infty$ vil der dermed være større sandsynlighed for, at forbrugeren er i arbejde, end at han er arbejdsløs, som angivet i \eqref{eq:pi_p,q}. 

Herefter vælges $\alpha = \beta = 0.5$. Overgangsmatricen er i så fald givet ved
\begin{align*}
    \P=\begin{bmatrix}0.5 & 0.5\\ 0.5 & 0.5\end{bmatrix}.
\end{align*}
Det er i \autoref{bilag:beregninger} bestemt, at der eksisterer en entydig invariant fordeling, som er givet ved
\begin{align*}
    \bm \pi =\begin{bmatrix}0.5 & 0.5\end{bmatrix},
\end{align*}
og som Markov-kæden konvergerer imod. Dermed er sandsynligheden for, at forbrugeren er i arbejde lig sandsynligheden for, at han er arbejdsløs for $t \to \infty$. 

Til sidst vælges $\alpha=0.9$ og $\beta=0.5$. Dermed er overgangsmatricen givet ved
\begin{align*}
    \P=\begin{bmatrix}0.5 & 0.5\\ 0.9 & 0.1\end{bmatrix}.
\end{align*}
Det er i \autoref{bilag:beregninger} bestemt, at der eksisterer en entydig invariant fordeling, som er givet ved
\begin{align*}
    \bm \pi =\begin{bmatrix}0.643 & 0.357\end{bmatrix},
\end{align*}
og som Markov-kæden konvergerer imod. Dermed er der større sandsynlighed for, at forbrugeren er arbejdsløs, end at han er i arbejde for $t \to \infty$.

De optimale strategier og forventede maksimale belønninger for hver af værdierne for $\alpha$ og $\beta$ vil præsenteres. Disse er bestemt ved brug af den implementerede algoritme, se \autoref{Bilag:python_kode} og præsenteres i nedenstående tabel, \autoref{tab:ændring_i_p_og_q}.

\begin{table}[H]
\captionsetup{font=large}
\centering
\resizebox{10cm}{!}{%
\begin{tabular}{|c|ccc|}   \hline
 $\alpha$     & 0.1 & 0.5 & 0.9  \\ 
 $\beta$     & 0.5 & 0.5 & 0.5  \\   \thickhline
 $t$    &   &   Forbrug & \\ \hline
 0   & 15.70 & 12.50 & 11.20   \\ \hline
 1   & 15.74 & 13.00 & 11.98    \\  \hline
 2   & 15.78 & 13.05 & 11.72   \\ \hline 
 3   & 15.91 & 13.04 & 11.87   \\ \hline 
 4   & 15.89 & 13.06 & 11.82  \\  \hline 
 5   & 15.94 & 13.12 & 11.91   \\    \hline 
 6   & 16.06 & 13.14 & 11.86  \\ \hline 
 7   & 16.08 & 13.23 & 11.92   \\   \hline 
 8   & 16.24 & 13.31 & 11.97   \\ \hline 
 9   & 16.46 & 13.50 & 12.01 \\   \hline 
 10  & 16.86 & 13.92 & 12.23 \\ \hline 
 11  & 17.74 & 15.28 & 13.22  \\   \thickhline
 Belønning & 48.29 & 43.83 & 41.52  \\ \thickhline
\end{tabular}
}
\caption{Optimal strategi samt belønning ved variation af $\alpha$ og $\beta$.}\label{tab:ændring_i_p_og_q}
\end{table}

I ovenstående tabel er de optimale strategier samt den tilhørende forventede maksimale belønning præsenteret i kolonne 2-4. Hver beslutningsregel har et tilhørende beslutningstidspunkt, som er i kolonne 1. Altså gælder det, at i første måned, $t=0$, med $\alpha=0.1$, $\beta=0.5$ og $\gamma_i=0$ opnås den forventede maksimale belønning ved at forbruge $15.70$. De optimale strategier i \autoref{tab:ændring_i_p_og_q} illustreres på følgende figur

\begin{figure} [H] 
    \begin{center}
      \resizebox{12cm}{!}{\input{Afsnit/PGF/ændring_i_p.pgf}}
    \end{center}
    \caption{Optimale strategi ved $\gamma_i=0$ og ved variation af $\alpha$ og $\beta$.}\label{fig:Optimal_strategi_ved_ændring_i_p}
\end{figure}


%Det bemærkes, at investoren forsøger at fordele forbruget ligeligt i starten, hvorefter det stiger markant til sidst. Desuden ses det, at forbruget stiger mere i starten i takt med sandsynligheden for, at han er arbejdsløs til det pågældende tidspunkt er større. Jo større $p$ er, des mere ustabil er konvergensen mod den invariante fordeling. Grunden til, at investoren forbruger mere til sidst er fordi han ved en ustabil fremtid vil forbruge konservativt op til $t=8$, hvorefter den forventede belønning er tættere på den rigtige belønning. 

%Eftersom det er garanteret, at investoren får løn til $t=1$, stiger forbruget (selvom det kun er tydelig for $p = 0.5$). 

%# Akser osv  ✓
%#Ingen rente ✓
%# kvadratroden fordeler stadig ud  ✓
%# Forbruget er generelt stigende  ✓
%# Usikkerhed gør, at den er voksende ✓
%# Voksende i starten - Fordi han er i job til t=0 ✓
%# Sandsynligheden for at være eller ikke at være i arbejde konvergerer mod den invariante fordeling over tid 
%# grafen er flot
% 

% konservativ
% 
I \autoref{fig:Optimal_strategi_ved_ændring_i_p} afbildes forbruget over tid for $\gamma_i=0$ og $\beta=0.5$ givet, at forbrugeren starter i arbejde. Sandsynligheden for at forbrugeren mister sit arbejde, er henholdsvis $\alpha=0.1, \alpha=0.5$ og $\alpha=0.9$ for de tre kurver. 

%I \autoref{fig:Optimal_strategi_ved_ændring_i_p} afbildes forbruget over tid givet, at forbrugeren starter i arbejde. I \autoref{fig:Optimal_strategi_ved_ændring_i_p} afbildes tre kurver, hvor $\gamma_i=0$ og $\beta=0.5$. Sandsynligheden for at forbrugeren mister sit arbejde, er henholdsvis $\alpha=0.1, \alpha=0.5$ og $\alpha=0.9$ for de tre kurver. 


%Generelt medfører en nyttefunktion, som er givet ved kvadratroden, at investoren bliver mere tilbøjelig til at fordele sit forbrug over hele perioden.
%For den blå graf er det let at se, at forbruget stiger i takt med, at den forventede værdi bliver en mere nøjagtig approksimation, 

Det er givet, at forbrugeren med sikkerhed er i arbejde for $t=0$ og sandsynligheden for at være i arbejde i de efterfølgende tidperioder er lavere. Ved $t=0$ har forbrugeren kun sin startopsparing at forbruge af, og til $t=1$ er han garanteret at få løn. Derfor kan der aflæses en stigning i forbruget fra $t=0$ til $t=1$. På grund af dette analyseres der fra $t=1$ til $t=11$ i følgende. 

Da forbrugerens belønningsfunktion er givet ved \eqref{eq:belønning_fornuftig_forbruger}, forventes det, at han fordeler sit forbrug ligeligt mellem månederne, hvilket tilnærmelsesvis er tilfældet. Dog bemærkes det i \autoref{tab:ændring_i_p_og_q}, at forbruget er stigende uafhængigt af værdierne for $\alpha$ over tidsperioden. Dette skyldes, at forbrugeren er mere konservativ i starten, da han i gennemsnit gerne vil have overskydende kapital i tilfælde af, at han ikke er i arbejde til næste beslutningstidspunkt. 
Ved $\alpha=0.1$ er der lavere sandsynlig for, at forbrugeren mister sit arbejde, og derfor er han mindre konservativ med sit forbrug i de senere tidsperioder. For $\alpha=0.9$ er sandsynligheden for, at forbrugeren er arbejdsløs størst sammenlignet med de andre. I dette tilfælde er forbrugeren derfor mere konservativ med sit forbrug end for $\alpha=0.1$ og $\alpha=0.5$. I alle tre tilfælde bemærkes en stor stigning i forbruget til sidste beslutningstidspunkt. Dette skyldes, at forbrugeren ikke behøver at tage højde for, om han er i arbejde til fremtidige beslutningstidspunkter, og han skal dermed ikke spare penge op. Altså opnår forbrugeren størst mulig belønning ved at forbruge hans resterende opsparing til sidste beslutningstidspunkt.

Derudover gælder det, at når sandsynligheden, for at forbrugeren er arbejdsløs, er større, vil han generelt være i arbejde til færre beslutningstidspunkter. Dermed vil forbrugeren ikke modtage løn til lige så mange beslutningstidspunkter, og han vil derfor have mindre at forbruge. Dette kan aflæses i \autoref{tab:ændring_i_p_og_q} og \autoref{fig:Optimal_strategi_ved_ændring_i_p}, hvor forbrugeren generelt har et lavere forbrug, når sandsynligheden for at han mister sit job, er større. 

%Udover dette, er forbruget en funktion af $t$'te trins overgangssandsynligheden. Dette er eksempevelses tydeligt i den grønne graf, eftersom $t$'te trins overgangssandsynligheden approksimerer den invariante fordeling. 

For at beskrive grafen yderligere analyseres $t$'te trins overgangssandsynlighederne for, at forbrugeren mister sit arbejde. Disse er beregnet i henholdvis \eqref{eq:p_t.p0.1}, \eqref{eq:matrice_p=0.5} og \eqref{eq:matrice_p=0.9}, og er givet ved følgende
\begin{align*}
    p_{10}^{(t)}&=0.167-0.167\cdot 0.4^t \text{ for } \alpha=0.1\\
    p_{10}^{(t)} &= 0.5 \text{ for } \alpha = 0.5\\
    p_{10}^{(t)} &= 0.643 - 0.357 \cdot (-0.4)^t \text{ for } \alpha=0.9
\end{align*}
Altså er $p_{10}^{(t)}$ for $\alpha=0.1$ monotont voksende og konvergerer mod den invariante fordeling, $\pi_0 = 0.167$. Overgangssandsynligheden for $\alpha=0.5$ forbliver konstant for alle beslutningstidspunkter, mens den for $\alpha=0.9$ er oscillerende og konvergerer mod den invariante fordeling, $\pi_0=0.643$. 

Forbruget er direkte afhængig af sandsynligheden for at være i arbejde. Dette afspejles også i \autoref{fig:Optimal_strategi_ved_ændring_i_p}, hvilket er tydeligst for $\alpha=0.9$. På den grønne kurve ses det, at ved de første beslutningstidspunkter oscillerer kurven på samme måde, som overgangssandsynligheden oscillerer. For den orange kurve er overgangssandsynligheden konstant og har ingen effekt på hældningen. Da overgangssandsynligheden for den blå kurve er monotont voksende, vil dette afspejles på kurven ved, at den blå kurve ligeledes er monotont voksende. 

%Da forbrugeren er fornuftig, vil han forsøge at fordele sit forbrug ligeligt mellem beslutningstidspunkterne. 






%For alle tre værdier for $\alpha$ kan det aflæses at forbrugeren vil fordele sit forbrug





% \MapleInput{$p := 0.5$}

% \MapleOutput{$\begin{bmatrix} test & test\\
% test & test\end{bmatrix} $}


%I den grønne graf er det usandsynligt, at investoren er i arbejde efter begyndelsestidspunktet, og derved vil forbruget falde igen. 

%Sandsynligheden for at være i arbejde til tiden $t$ er givet ved $t$'te trins overgangssandsynlighederne, som for den grønne graf er oscillerende. 

%\textbf{Analyse:\\
%- Alle fordeler det ligeligt med en stigning tilsidst (hvorfor)\\
%- Skal vi nævne at hvis der er flere tidsperioder så er det ikke sådan at den allerede %stiger fra t = 9-10 stykker, det er stadig først tilsidst. 
%- Kan man bestemme varians af dem og giver det mening?\\
%- Jo større sandsynlighed for at være i arbejde, jo flere penge forbruge generelt - mere i arbejde har flere penge (giver mening)\\
%- Hvorfor er hakket større jo højere p og q?\\ illustrer med graf}



%\subsection{Variation af indlånsrente}
\subsection[Variation af  \texorpdfstring{$\gamma_i$}{indlånsrenten}]{Variation af \bm{$\gamma_i$}}

Ved at variere indlånsrenten, $\gamma_i$, mens $\alpha$, $\beta$ samt de andre variable fastholdes, kan det analyseres, hvordan indlånsrenten påvirker forbruget og den forventede maksimale belønning.

Da belønningsfunktionen er givet ved kvadratroden af forbruget, forventes det, at forbrugeren vil opnå den forventede maksimale belønning ved at fordele forbruget ligeligt mellem månederne. Når indlånsrenten vokser, forventes det, at forbrugeren er mere tilbøjelig til at udnytte renters rente end tendensen for at fordele forbruget ligeligt. 

Det antages for alle de følgende tilfælde, at $\alpha=0.1$ og $\beta=0.5$. Dermed er overgangssandsynlighederne givet ved \eqref{eq:p_t.p0.1}, som går mod sandsynlighedsfordelingen \eqref{eq:pi_p,q}. Altså er sandsynligheden for, at forbrugeren er i arbejde større end sandsynligheden for, at han er arbejdsløs for $t\to\infty $. Indlånsrenterne der sammenlignes, er $\gamma_i = 0, \gamma_i=0.02, \gamma_i = 0.05$ og $\gamma_i = 0.1$. I \autoref{tab:ændring_r} præsenteres den optimale strategi for forbrugeren, givet de forskellige indlånsrenter samt den forventede maksimale belønning givet denne strategi. 

\begin{table}[H]
\captionsetup{font=large}
\centering
\resizebox{12cm}{!}{%
\begin{tabular}{|c|cccc|}   \hline
 $\gamma_i$     & 0  & 0.02  & 0.05 & 0.1  \\   \thickhline
 $t$    &  & \multicolumn{2}{c}{Forbrug} &  \\ \hline
 0   & 15.70 & 13.30 & 10.00  & \phantom{0}7.00  \\ \hline
 1   & 15.74 & 13.40 & 12.31  & \phantom{0}9.20  \\ \hline
 2   & 15.78 & 15.85 & 12.61  & 10.26  \\ \hline 
 3   & 15.91 & 15.79 & 14.48  & 13.07  \\ \hline 
 4   & 15.89 & 16.12 & 16.20  & 15.64  \\ \hline 
 5   & 15.95 & 16.44 & 17.70  & 19.01  \\ \hline 
 6   & 16.06 & 17.39 & 19.48  & 23.03  \\ \hline 
 7   & 16.08 & 17.81 & 21.02  & 27.65  \\ \hline 
 8   & 16.24 & 18.68 & 23.55  & 33.38  \\ \hline 
 9   & 16.46 & 19.23 & 26.26  & 40.81  \\ \hline 
 10  & 16.86 & 20.72 & 28.93  & 49.51  \\ \hline 
 11  & 17.74 & 22.29 & 32.33  & 60.45  \\ \thickhline
 Belønning & 48.29 & 49.70 & 52.30 & 57.84 \\ \thickhline
\end{tabular}
}
\caption{Optimal strategi samt belønning ved variation af $\gamma_i$.}\label{tab:ændring_r}
\end{table}


I ovenstående tabel, \autoref{tab:ændring_r}, kan den optimale strategi samt de tilhørende forventede maksimale belønninger aflæses i kolonne 2-5, og i kolonne 1 er hvert beslutningstidspunkt præsenteret. I nedenstående figur, \autoref{fig:renter}, illustreres de optimale strategier.

\begin{figure}[H]
    \begin{center}
        \resizebox{12cm}{!}{%% Creator: Matplotlib, PGF backend
%%
%% To include the figure in your LaTeX document, write
%%   \input{<filename>.pgf}
%%
%% Make sure the required packages are loaded in your preamble
%%   \usepackage{pgf}
%%
%% Figures using additional raster images can only be included by \input if
%% they are in the same directory as the main LaTeX file. For loading figures
%% from other directories you can use the `import` package
%%   \usepackage{import}
%%
%% and then include the figures with
%%   \import{<path to file>}{<filename>.pgf}
%%
%% Matplotlib used the following preamble
%%
\begingroup%
\makeatletter%
\begin{pgfpicture}%
\pgfpathrectangle{\pgfpointorigin}{\pgfqpoint{6.400000in}{4.800000in}}%
\pgfusepath{use as bounding box, clip}%
\begin{pgfscope}%
\pgfsetbuttcap%
\pgfsetmiterjoin%
\definecolor{currentfill}{rgb}{1.000000,1.000000,1.000000}%
\pgfsetfillcolor{currentfill}%
\pgfsetlinewidth{0.000000pt}%
\definecolor{currentstroke}{rgb}{1.000000,1.000000,1.000000}%
\pgfsetstrokecolor{currentstroke}%
\pgfsetdash{}{0pt}%
\pgfpathmoveto{\pgfqpoint{0.000000in}{0.000000in}}%
\pgfpathlineto{\pgfqpoint{6.400000in}{0.000000in}}%
\pgfpathlineto{\pgfqpoint{6.400000in}{4.800000in}}%
\pgfpathlineto{\pgfqpoint{0.000000in}{4.800000in}}%
\pgfpathclose%
\pgfusepath{fill}%
\end{pgfscope}%
\begin{pgfscope}%
\pgfsetbuttcap%
\pgfsetmiterjoin%
\definecolor{currentfill}{rgb}{1.000000,1.000000,1.000000}%
\pgfsetfillcolor{currentfill}%
\pgfsetlinewidth{0.000000pt}%
\definecolor{currentstroke}{rgb}{0.000000,0.000000,0.000000}%
\pgfsetstrokecolor{currentstroke}%
\pgfsetstrokeopacity{0.000000}%
\pgfsetdash{}{0pt}%
\pgfpathmoveto{\pgfqpoint{0.800000in}{0.528000in}}%
\pgfpathlineto{\pgfqpoint{5.760000in}{0.528000in}}%
\pgfpathlineto{\pgfqpoint{5.760000in}{4.224000in}}%
\pgfpathlineto{\pgfqpoint{0.800000in}{4.224000in}}%
\pgfpathclose%
\pgfusepath{fill}%
\end{pgfscope}%
\begin{pgfscope}%
\pgfsetbuttcap%
\pgfsetroundjoin%
\definecolor{currentfill}{rgb}{0.000000,0.000000,0.000000}%
\pgfsetfillcolor{currentfill}%
\pgfsetlinewidth{0.803000pt}%
\definecolor{currentstroke}{rgb}{0.000000,0.000000,0.000000}%
\pgfsetstrokecolor{currentstroke}%
\pgfsetdash{}{0pt}%
\pgfsys@defobject{currentmarker}{\pgfqpoint{0.000000in}{-0.048611in}}{\pgfqpoint{0.000000in}{0.000000in}}{%
\pgfpathmoveto{\pgfqpoint{0.000000in}{0.000000in}}%
\pgfpathlineto{\pgfqpoint{0.000000in}{-0.048611in}}%
\pgfusepath{stroke,fill}%
}%
\begin{pgfscope}%
\pgfsys@transformshift{1.025455in}{0.528000in}%
\pgfsys@useobject{currentmarker}{}%
\end{pgfscope}%
\end{pgfscope}%
\begin{pgfscope}%
\definecolor{textcolor}{rgb}{0.000000,0.000000,0.000000}%
\pgfsetstrokecolor{textcolor}%
\pgfsetfillcolor{textcolor}%
\pgftext[x=1.025455in,y=0.430778in,,top]{\color{textcolor}\rmfamily\fontsize{10.000000}{12.000000}\selectfont \(\displaystyle {0}\)}%
\end{pgfscope}%
\begin{pgfscope}%
\pgfsetbuttcap%
\pgfsetroundjoin%
\definecolor{currentfill}{rgb}{0.000000,0.000000,0.000000}%
\pgfsetfillcolor{currentfill}%
\pgfsetlinewidth{0.803000pt}%
\definecolor{currentstroke}{rgb}{0.000000,0.000000,0.000000}%
\pgfsetstrokecolor{currentstroke}%
\pgfsetdash{}{0pt}%
\pgfsys@defobject{currentmarker}{\pgfqpoint{0.000000in}{-0.048611in}}{\pgfqpoint{0.000000in}{0.000000in}}{%
\pgfpathmoveto{\pgfqpoint{0.000000in}{0.000000in}}%
\pgfpathlineto{\pgfqpoint{0.000000in}{-0.048611in}}%
\pgfusepath{stroke,fill}%
}%
\begin{pgfscope}%
\pgfsys@transformshift{1.845289in}{0.528000in}%
\pgfsys@useobject{currentmarker}{}%
\end{pgfscope}%
\end{pgfscope}%
\begin{pgfscope}%
\definecolor{textcolor}{rgb}{0.000000,0.000000,0.000000}%
\pgfsetstrokecolor{textcolor}%
\pgfsetfillcolor{textcolor}%
\pgftext[x=1.845289in,y=0.430778in,,top]{\color{textcolor}\rmfamily\fontsize{10.000000}{12.000000}\selectfont \(\displaystyle {2}\)}%
\end{pgfscope}%
\begin{pgfscope}%
\pgfsetbuttcap%
\pgfsetroundjoin%
\definecolor{currentfill}{rgb}{0.000000,0.000000,0.000000}%
\pgfsetfillcolor{currentfill}%
\pgfsetlinewidth{0.803000pt}%
\definecolor{currentstroke}{rgb}{0.000000,0.000000,0.000000}%
\pgfsetstrokecolor{currentstroke}%
\pgfsetdash{}{0pt}%
\pgfsys@defobject{currentmarker}{\pgfqpoint{0.000000in}{-0.048611in}}{\pgfqpoint{0.000000in}{0.000000in}}{%
\pgfpathmoveto{\pgfqpoint{0.000000in}{0.000000in}}%
\pgfpathlineto{\pgfqpoint{0.000000in}{-0.048611in}}%
\pgfusepath{stroke,fill}%
}%
\begin{pgfscope}%
\pgfsys@transformshift{2.665124in}{0.528000in}%
\pgfsys@useobject{currentmarker}{}%
\end{pgfscope}%
\end{pgfscope}%
\begin{pgfscope}%
\definecolor{textcolor}{rgb}{0.000000,0.000000,0.000000}%
\pgfsetstrokecolor{textcolor}%
\pgfsetfillcolor{textcolor}%
\pgftext[x=2.665124in,y=0.430778in,,top]{\color{textcolor}\rmfamily\fontsize{10.000000}{12.000000}\selectfont \(\displaystyle {4}\)}%
\end{pgfscope}%
\begin{pgfscope}%
\pgfsetbuttcap%
\pgfsetroundjoin%
\definecolor{currentfill}{rgb}{0.000000,0.000000,0.000000}%
\pgfsetfillcolor{currentfill}%
\pgfsetlinewidth{0.803000pt}%
\definecolor{currentstroke}{rgb}{0.000000,0.000000,0.000000}%
\pgfsetstrokecolor{currentstroke}%
\pgfsetdash{}{0pt}%
\pgfsys@defobject{currentmarker}{\pgfqpoint{0.000000in}{-0.048611in}}{\pgfqpoint{0.000000in}{0.000000in}}{%
\pgfpathmoveto{\pgfqpoint{0.000000in}{0.000000in}}%
\pgfpathlineto{\pgfqpoint{0.000000in}{-0.048611in}}%
\pgfusepath{stroke,fill}%
}%
\begin{pgfscope}%
\pgfsys@transformshift{3.484959in}{0.528000in}%
\pgfsys@useobject{currentmarker}{}%
\end{pgfscope}%
\end{pgfscope}%
\begin{pgfscope}%
\definecolor{textcolor}{rgb}{0.000000,0.000000,0.000000}%
\pgfsetstrokecolor{textcolor}%
\pgfsetfillcolor{textcolor}%
\pgftext[x=3.484959in,y=0.430778in,,top]{\color{textcolor}\rmfamily\fontsize{10.000000}{12.000000}\selectfont \(\displaystyle {6}\)}%
\end{pgfscope}%
\begin{pgfscope}%
\pgfsetbuttcap%
\pgfsetroundjoin%
\definecolor{currentfill}{rgb}{0.000000,0.000000,0.000000}%
\pgfsetfillcolor{currentfill}%
\pgfsetlinewidth{0.803000pt}%
\definecolor{currentstroke}{rgb}{0.000000,0.000000,0.000000}%
\pgfsetstrokecolor{currentstroke}%
\pgfsetdash{}{0pt}%
\pgfsys@defobject{currentmarker}{\pgfqpoint{0.000000in}{-0.048611in}}{\pgfqpoint{0.000000in}{0.000000in}}{%
\pgfpathmoveto{\pgfqpoint{0.000000in}{0.000000in}}%
\pgfpathlineto{\pgfqpoint{0.000000in}{-0.048611in}}%
\pgfusepath{stroke,fill}%
}%
\begin{pgfscope}%
\pgfsys@transformshift{4.304793in}{0.528000in}%
\pgfsys@useobject{currentmarker}{}%
\end{pgfscope}%
\end{pgfscope}%
\begin{pgfscope}%
\definecolor{textcolor}{rgb}{0.000000,0.000000,0.000000}%
\pgfsetstrokecolor{textcolor}%
\pgfsetfillcolor{textcolor}%
\pgftext[x=4.304793in,y=0.430778in,,top]{\color{textcolor}\rmfamily\fontsize{10.000000}{12.000000}\selectfont \(\displaystyle {8}\)}%
\end{pgfscope}%
\begin{pgfscope}%
\pgfsetbuttcap%
\pgfsetroundjoin%
\definecolor{currentfill}{rgb}{0.000000,0.000000,0.000000}%
\pgfsetfillcolor{currentfill}%
\pgfsetlinewidth{0.803000pt}%
\definecolor{currentstroke}{rgb}{0.000000,0.000000,0.000000}%
\pgfsetstrokecolor{currentstroke}%
\pgfsetdash{}{0pt}%
\pgfsys@defobject{currentmarker}{\pgfqpoint{0.000000in}{-0.048611in}}{\pgfqpoint{0.000000in}{0.000000in}}{%
\pgfpathmoveto{\pgfqpoint{0.000000in}{0.000000in}}%
\pgfpathlineto{\pgfqpoint{0.000000in}{-0.048611in}}%
\pgfusepath{stroke,fill}%
}%
\begin{pgfscope}%
\pgfsys@transformshift{5.124628in}{0.528000in}%
\pgfsys@useobject{currentmarker}{}%
\end{pgfscope}%
\end{pgfscope}%
\begin{pgfscope}%
\definecolor{textcolor}{rgb}{0.000000,0.000000,0.000000}%
\pgfsetstrokecolor{textcolor}%
\pgfsetfillcolor{textcolor}%
\pgftext[x=5.124628in,y=0.430778in,,top]{\color{textcolor}\rmfamily\fontsize{10.000000}{12.000000}\selectfont \(\displaystyle {10}\)}%
\end{pgfscope}%
\begin{pgfscope}%
\definecolor{textcolor}{rgb}{0.000000,0.000000,0.000000}%
\pgfsetstrokecolor{textcolor}%
\pgfsetfillcolor{textcolor}%
\pgftext[x=3.280000in,y=0.251766in,,top]{\color{textcolor}\rmfamily\fontsize{10.000000}{12.000000}\selectfont \(\displaystyle t\)}%
\end{pgfscope}%
\begin{pgfscope}%
\pgfsetbuttcap%
\pgfsetroundjoin%
\definecolor{currentfill}{rgb}{0.000000,0.000000,0.000000}%
\pgfsetfillcolor{currentfill}%
\pgfsetlinewidth{0.803000pt}%
\definecolor{currentstroke}{rgb}{0.000000,0.000000,0.000000}%
\pgfsetstrokecolor{currentstroke}%
\pgfsetdash{}{0pt}%
\pgfsys@defobject{currentmarker}{\pgfqpoint{-0.048611in}{0.000000in}}{\pgfqpoint{-0.000000in}{0.000000in}}{%
\pgfpathmoveto{\pgfqpoint{-0.000000in}{0.000000in}}%
\pgfpathlineto{\pgfqpoint{-0.048611in}{0.000000in}}%
\pgfusepath{stroke,fill}%
}%
\begin{pgfscope}%
\pgfsys@transformshift{0.800000in}{0.884596in}%
\pgfsys@useobject{currentmarker}{}%
\end{pgfscope}%
\end{pgfscope}%
\begin{pgfscope}%
\definecolor{textcolor}{rgb}{0.000000,0.000000,0.000000}%
\pgfsetstrokecolor{textcolor}%
\pgfsetfillcolor{textcolor}%
\pgftext[x=0.563888in, y=0.836371in, left, base]{\color{textcolor}\rmfamily\fontsize{10.000000}{12.000000}\selectfont \(\displaystyle {10}\)}%
\end{pgfscope}%
\begin{pgfscope}%
\pgfsetbuttcap%
\pgfsetroundjoin%
\definecolor{currentfill}{rgb}{0.000000,0.000000,0.000000}%
\pgfsetfillcolor{currentfill}%
\pgfsetlinewidth{0.803000pt}%
\definecolor{currentstroke}{rgb}{0.000000,0.000000,0.000000}%
\pgfsetstrokecolor{currentstroke}%
\pgfsetdash{}{0pt}%
\pgfsys@defobject{currentmarker}{\pgfqpoint{-0.048611in}{0.000000in}}{\pgfqpoint{-0.000000in}{0.000000in}}{%
\pgfpathmoveto{\pgfqpoint{-0.000000in}{0.000000in}}%
\pgfpathlineto{\pgfqpoint{-0.048611in}{0.000000in}}%
\pgfusepath{stroke,fill}%
}%
\begin{pgfscope}%
\pgfsys@transformshift{0.800000in}{1.513250in}%
\pgfsys@useobject{currentmarker}{}%
\end{pgfscope}%
\end{pgfscope}%
\begin{pgfscope}%
\definecolor{textcolor}{rgb}{0.000000,0.000000,0.000000}%
\pgfsetstrokecolor{textcolor}%
\pgfsetfillcolor{textcolor}%
\pgftext[x=0.563888in, y=1.465025in, left, base]{\color{textcolor}\rmfamily\fontsize{10.000000}{12.000000}\selectfont \(\displaystyle {20}\)}%
\end{pgfscope}%
\begin{pgfscope}%
\pgfsetbuttcap%
\pgfsetroundjoin%
\definecolor{currentfill}{rgb}{0.000000,0.000000,0.000000}%
\pgfsetfillcolor{currentfill}%
\pgfsetlinewidth{0.803000pt}%
\definecolor{currentstroke}{rgb}{0.000000,0.000000,0.000000}%
\pgfsetstrokecolor{currentstroke}%
\pgfsetdash{}{0pt}%
\pgfsys@defobject{currentmarker}{\pgfqpoint{-0.048611in}{0.000000in}}{\pgfqpoint{-0.000000in}{0.000000in}}{%
\pgfpathmoveto{\pgfqpoint{-0.000000in}{0.000000in}}%
\pgfpathlineto{\pgfqpoint{-0.048611in}{0.000000in}}%
\pgfusepath{stroke,fill}%
}%
\begin{pgfscope}%
\pgfsys@transformshift{0.800000in}{2.141904in}%
\pgfsys@useobject{currentmarker}{}%
\end{pgfscope}%
\end{pgfscope}%
\begin{pgfscope}%
\definecolor{textcolor}{rgb}{0.000000,0.000000,0.000000}%
\pgfsetstrokecolor{textcolor}%
\pgfsetfillcolor{textcolor}%
\pgftext[x=0.563888in, y=2.093678in, left, base]{\color{textcolor}\rmfamily\fontsize{10.000000}{12.000000}\selectfont \(\displaystyle {30}\)}%
\end{pgfscope}%
\begin{pgfscope}%
\pgfsetbuttcap%
\pgfsetroundjoin%
\definecolor{currentfill}{rgb}{0.000000,0.000000,0.000000}%
\pgfsetfillcolor{currentfill}%
\pgfsetlinewidth{0.803000pt}%
\definecolor{currentstroke}{rgb}{0.000000,0.000000,0.000000}%
\pgfsetstrokecolor{currentstroke}%
\pgfsetdash{}{0pt}%
\pgfsys@defobject{currentmarker}{\pgfqpoint{-0.048611in}{0.000000in}}{\pgfqpoint{-0.000000in}{0.000000in}}{%
\pgfpathmoveto{\pgfqpoint{-0.000000in}{0.000000in}}%
\pgfpathlineto{\pgfqpoint{-0.048611in}{0.000000in}}%
\pgfusepath{stroke,fill}%
}%
\begin{pgfscope}%
\pgfsys@transformshift{0.800000in}{2.770557in}%
\pgfsys@useobject{currentmarker}{}%
\end{pgfscope}%
\end{pgfscope}%
\begin{pgfscope}%
\definecolor{textcolor}{rgb}{0.000000,0.000000,0.000000}%
\pgfsetstrokecolor{textcolor}%
\pgfsetfillcolor{textcolor}%
\pgftext[x=0.563888in, y=2.722332in, left, base]{\color{textcolor}\rmfamily\fontsize{10.000000}{12.000000}\selectfont \(\displaystyle {40}\)}%
\end{pgfscope}%
\begin{pgfscope}%
\pgfsetbuttcap%
\pgfsetroundjoin%
\definecolor{currentfill}{rgb}{0.000000,0.000000,0.000000}%
\pgfsetfillcolor{currentfill}%
\pgfsetlinewidth{0.803000pt}%
\definecolor{currentstroke}{rgb}{0.000000,0.000000,0.000000}%
\pgfsetstrokecolor{currentstroke}%
\pgfsetdash{}{0pt}%
\pgfsys@defobject{currentmarker}{\pgfqpoint{-0.048611in}{0.000000in}}{\pgfqpoint{-0.000000in}{0.000000in}}{%
\pgfpathmoveto{\pgfqpoint{-0.000000in}{0.000000in}}%
\pgfpathlineto{\pgfqpoint{-0.048611in}{0.000000in}}%
\pgfusepath{stroke,fill}%
}%
\begin{pgfscope}%
\pgfsys@transformshift{0.800000in}{3.399211in}%
\pgfsys@useobject{currentmarker}{}%
\end{pgfscope}%
\end{pgfscope}%
\begin{pgfscope}%
\definecolor{textcolor}{rgb}{0.000000,0.000000,0.000000}%
\pgfsetstrokecolor{textcolor}%
\pgfsetfillcolor{textcolor}%
\pgftext[x=0.563888in, y=3.350986in, left, base]{\color{textcolor}\rmfamily\fontsize{10.000000}{12.000000}\selectfont \(\displaystyle {50}\)}%
\end{pgfscope}%
\begin{pgfscope}%
\pgfsetbuttcap%
\pgfsetroundjoin%
\definecolor{currentfill}{rgb}{0.000000,0.000000,0.000000}%
\pgfsetfillcolor{currentfill}%
\pgfsetlinewidth{0.803000pt}%
\definecolor{currentstroke}{rgb}{0.000000,0.000000,0.000000}%
\pgfsetstrokecolor{currentstroke}%
\pgfsetdash{}{0pt}%
\pgfsys@defobject{currentmarker}{\pgfqpoint{-0.048611in}{0.000000in}}{\pgfqpoint{-0.000000in}{0.000000in}}{%
\pgfpathmoveto{\pgfqpoint{-0.000000in}{0.000000in}}%
\pgfpathlineto{\pgfqpoint{-0.048611in}{0.000000in}}%
\pgfusepath{stroke,fill}%
}%
\begin{pgfscope}%
\pgfsys@transformshift{0.800000in}{4.027865in}%
\pgfsys@useobject{currentmarker}{}%
\end{pgfscope}%
\end{pgfscope}%
\begin{pgfscope}%
\definecolor{textcolor}{rgb}{0.000000,0.000000,0.000000}%
\pgfsetstrokecolor{textcolor}%
\pgfsetfillcolor{textcolor}%
\pgftext[x=0.563888in, y=3.979639in, left, base]{\color{textcolor}\rmfamily\fontsize{10.000000}{12.000000}\selectfont \(\displaystyle {60}\)}%
\end{pgfscope}%
\begin{pgfscope}%
\definecolor{textcolor}{rgb}{0.000000,0.000000,0.000000}%
\pgfsetstrokecolor{textcolor}%
\pgfsetfillcolor{textcolor}%
\pgftext[x=0.508333in,y=2.376000in,,bottom,rotate=90.000000]{\color{textcolor}\rmfamily\fontsize{10.000000}{12.000000}\selectfont Forbrug}%
\end{pgfscope}%
\begin{pgfscope}%
\pgfpathrectangle{\pgfqpoint{0.800000in}{0.528000in}}{\pgfqpoint{4.960000in}{3.696000in}}%
\pgfusepath{clip}%
\pgfsetrectcap%
\pgfsetroundjoin%
\pgfsetlinewidth{1.505625pt}%
\definecolor{currentstroke}{rgb}{0.121569,0.466667,0.705882}%
\pgfsetstrokecolor{currentstroke}%
\pgfsetdash{}{0pt}%
\pgfpathmoveto{\pgfqpoint{1.025455in}{1.242929in}}%
\pgfpathlineto{\pgfqpoint{1.435372in}{1.245277in}}%
\pgfpathlineto{\pgfqpoint{1.845289in}{1.248243in}}%
\pgfpathlineto{\pgfqpoint{2.255207in}{1.256107in}}%
\pgfpathlineto{\pgfqpoint{2.665124in}{1.254946in}}%
\pgfpathlineto{\pgfqpoint{3.075041in}{1.258508in}}%
\pgfpathlineto{\pgfqpoint{3.484959in}{1.265321in}}%
\pgfpathlineto{\pgfqpoint{3.894876in}{1.266941in}}%
\pgfpathlineto{\pgfqpoint{4.304793in}{1.276951in}}%
\pgfpathlineto{\pgfqpoint{4.714711in}{1.290789in}}%
\pgfpathlineto{\pgfqpoint{5.124628in}{1.315555in}}%
\pgfpathlineto{\pgfqpoint{5.534545in}{1.371148in}}%
\pgfusepath{stroke}%
\end{pgfscope}%
\begin{pgfscope}%
\pgfpathrectangle{\pgfqpoint{0.800000in}{0.528000in}}{\pgfqpoint{4.960000in}{3.696000in}}%
\pgfusepath{clip}%
\pgfsetrectcap%
\pgfsetroundjoin%
\pgfsetlinewidth{1.505625pt}%
\definecolor{currentstroke}{rgb}{1.000000,0.498039,0.054902}%
\pgfsetstrokecolor{currentstroke}%
\pgfsetdash{}{0pt}%
\pgfpathmoveto{\pgfqpoint{1.025455in}{1.092052in}}%
\pgfpathlineto{\pgfqpoint{1.435372in}{1.098338in}}%
\pgfpathlineto{\pgfqpoint{1.845289in}{1.252385in}}%
\pgfpathlineto{\pgfqpoint{2.255207in}{1.248284in}}%
\pgfpathlineto{\pgfqpoint{2.665124in}{1.269254in}}%
\pgfpathlineto{\pgfqpoint{3.075041in}{1.289618in}}%
\pgfpathlineto{\pgfqpoint{3.484959in}{1.349052in}}%
\pgfpathlineto{\pgfqpoint{3.894876in}{1.375858in}}%
\pgfpathlineto{\pgfqpoint{4.304793in}{1.430302in}}%
\pgfpathlineto{\pgfqpoint{4.714711in}{1.465146in}}%
\pgfpathlineto{\pgfqpoint{5.124628in}{1.558732in}}%
\pgfpathlineto{\pgfqpoint{5.534545in}{1.657107in}}%
\pgfusepath{stroke}%
\end{pgfscope}%
\begin{pgfscope}%
\pgfpathrectangle{\pgfqpoint{0.800000in}{0.528000in}}{\pgfqpoint{4.960000in}{3.696000in}}%
\pgfusepath{clip}%
\pgfsetrectcap%
\pgfsetroundjoin%
\pgfsetlinewidth{1.505625pt}%
\definecolor{currentstroke}{rgb}{0.172549,0.627451,0.172549}%
\pgfsetstrokecolor{currentstroke}%
\pgfsetdash{}{0pt}%
\pgfpathmoveto{\pgfqpoint{1.025455in}{0.884596in}}%
\pgfpathlineto{\pgfqpoint{1.435372in}{1.029853in}}%
\pgfpathlineto{\pgfqpoint{1.845289in}{1.048641in}}%
\pgfpathlineto{\pgfqpoint{2.255207in}{1.166429in}}%
\pgfpathlineto{\pgfqpoint{2.665124in}{1.274806in}}%
\pgfpathlineto{\pgfqpoint{3.075041in}{1.368347in}}%
\pgfpathlineto{\pgfqpoint{3.484959in}{1.480708in}}%
\pgfpathlineto{\pgfqpoint{3.894876in}{1.577565in}}%
\pgfpathlineto{\pgfqpoint{4.304793in}{1.736264in}}%
\pgfpathlineto{\pgfqpoint{4.714711in}{1.906603in}}%
\pgfpathlineto{\pgfqpoint{5.124628in}{2.074880in}}%
\pgfpathlineto{\pgfqpoint{5.534545in}{2.288074in}}%
\pgfusepath{stroke}%
\end{pgfscope}%
\begin{pgfscope}%
\pgfpathrectangle{\pgfqpoint{0.800000in}{0.528000in}}{\pgfqpoint{4.960000in}{3.696000in}}%
\pgfusepath{clip}%
\pgfsetrectcap%
\pgfsetroundjoin%
\pgfsetlinewidth{1.505625pt}%
\definecolor{currentstroke}{rgb}{0.839216,0.152941,0.156863}%
\pgfsetstrokecolor{currentstroke}%
\pgfsetdash{}{0pt}%
\pgfpathmoveto{\pgfqpoint{1.025455in}{0.696000in}}%
\pgfpathlineto{\pgfqpoint{1.435372in}{0.834291in}}%
\pgfpathlineto{\pgfqpoint{1.845289in}{0.900886in}}%
\pgfpathlineto{\pgfqpoint{2.255207in}{1.077417in}}%
\pgfpathlineto{\pgfqpoint{2.665124in}{1.239115in}}%
\pgfpathlineto{\pgfqpoint{3.075041in}{1.450883in}}%
\pgfpathlineto{\pgfqpoint{3.484959in}{1.703425in}}%
\pgfpathlineto{\pgfqpoint{3.894876in}{1.994103in}}%
\pgfpathlineto{\pgfqpoint{4.304793in}{2.354693in}}%
\pgfpathlineto{\pgfqpoint{4.714711in}{2.821300in}}%
\pgfpathlineto{\pgfqpoint{5.124628in}{3.368177in}}%
\pgfpathlineto{\pgfqpoint{5.534545in}{4.056000in}}%
\pgfusepath{stroke}%
\end{pgfscope}%
\begin{pgfscope}%
\pgfsetrectcap%
\pgfsetmiterjoin%
\pgfsetlinewidth{0.803000pt}%
\definecolor{currentstroke}{rgb}{0.000000,0.000000,0.000000}%
\pgfsetstrokecolor{currentstroke}%
\pgfsetdash{}{0pt}%
\pgfpathmoveto{\pgfqpoint{0.800000in}{0.528000in}}%
\pgfpathlineto{\pgfqpoint{0.800000in}{4.224000in}}%
\pgfusepath{stroke}%
\end{pgfscope}%
\begin{pgfscope}%
\pgfsetrectcap%
\pgfsetmiterjoin%
\pgfsetlinewidth{0.803000pt}%
\definecolor{currentstroke}{rgb}{0.000000,0.000000,0.000000}%
\pgfsetstrokecolor{currentstroke}%
\pgfsetdash{}{0pt}%
\pgfpathmoveto{\pgfqpoint{5.760000in}{0.528000in}}%
\pgfpathlineto{\pgfqpoint{5.760000in}{4.224000in}}%
\pgfusepath{stroke}%
\end{pgfscope}%
\begin{pgfscope}%
\pgfsetrectcap%
\pgfsetmiterjoin%
\pgfsetlinewidth{0.803000pt}%
\definecolor{currentstroke}{rgb}{0.000000,0.000000,0.000000}%
\pgfsetstrokecolor{currentstroke}%
\pgfsetdash{}{0pt}%
\pgfpathmoveto{\pgfqpoint{0.800000in}{0.528000in}}%
\pgfpathlineto{\pgfqpoint{5.760000in}{0.528000in}}%
\pgfusepath{stroke}%
\end{pgfscope}%
\begin{pgfscope}%
\pgfsetrectcap%
\pgfsetmiterjoin%
\pgfsetlinewidth{0.803000pt}%
\definecolor{currentstroke}{rgb}{0.000000,0.000000,0.000000}%
\pgfsetstrokecolor{currentstroke}%
\pgfsetdash{}{0pt}%
\pgfpathmoveto{\pgfqpoint{0.800000in}{4.224000in}}%
\pgfpathlineto{\pgfqpoint{5.760000in}{4.224000in}}%
\pgfusepath{stroke}%
\end{pgfscope}%
\begin{pgfscope}%
\pgfsetbuttcap%
\pgfsetmiterjoin%
\definecolor{currentfill}{rgb}{1.000000,1.000000,1.000000}%
\pgfsetfillcolor{currentfill}%
\pgfsetfillopacity{0.800000}%
\pgfsetlinewidth{1.003750pt}%
\definecolor{currentstroke}{rgb}{0.800000,0.800000,0.800000}%
\pgfsetstrokecolor{currentstroke}%
\pgfsetstrokeopacity{0.800000}%
\pgfsetdash{}{0pt}%
\pgfpathmoveto{\pgfqpoint{0.916667in}{3.161038in}}%
\pgfpathlineto{\pgfqpoint{2.091426in}{3.161038in}}%
\pgfpathquadraticcurveto{\pgfqpoint{2.124759in}{3.161038in}}{\pgfqpoint{2.124759in}{3.194372in}}%
\pgfpathlineto{\pgfqpoint{2.124759in}{4.107333in}}%
\pgfpathquadraticcurveto{\pgfqpoint{2.124759in}{4.140667in}}{\pgfqpoint{2.091426in}{4.140667in}}%
\pgfpathlineto{\pgfqpoint{0.916667in}{4.140667in}}%
\pgfpathquadraticcurveto{\pgfqpoint{0.883333in}{4.140667in}}{\pgfqpoint{0.883333in}{4.107333in}}%
\pgfpathlineto{\pgfqpoint{0.883333in}{3.194372in}}%
\pgfpathquadraticcurveto{\pgfqpoint{0.883333in}{3.161038in}}{\pgfqpoint{0.916667in}{3.161038in}}%
\pgfpathclose%
\pgfusepath{stroke,fill}%
\end{pgfscope}%
\begin{pgfscope}%
\pgfsetrectcap%
\pgfsetroundjoin%
\pgfsetlinewidth{1.505625pt}%
\definecolor{currentstroke}{rgb}{0.121569,0.466667,0.705882}%
\pgfsetstrokecolor{currentstroke}%
\pgfsetdash{}{0pt}%
\pgfpathmoveto{\pgfqpoint{0.950000in}{4.015667in}}%
\pgfpathlineto{\pgfqpoint{1.283333in}{4.015667in}}%
\pgfusepath{stroke}%
\end{pgfscope}%
\begin{pgfscope}%
\definecolor{textcolor}{rgb}{0.000000,0.000000,0.000000}%
\pgfsetstrokecolor{textcolor}%
\pgfsetfillcolor{textcolor}%
\pgftext[x=1.416667in,y=3.957333in,left,base]{\color{textcolor}\rmfamily\fontsize{12.000000}{14.400000}\selectfont \(\displaystyle \gamma_i = 0\)}%
\end{pgfscope}%
\begin{pgfscope}%
\pgfsetrectcap%
\pgfsetroundjoin%
\pgfsetlinewidth{1.505625pt}%
\definecolor{currentstroke}{rgb}{1.000000,0.498039,0.054902}%
\pgfsetstrokecolor{currentstroke}%
\pgfsetdash{}{0pt}%
\pgfpathmoveto{\pgfqpoint{0.950000in}{3.783260in}}%
\pgfpathlineto{\pgfqpoint{1.283333in}{3.783260in}}%
\pgfusepath{stroke}%
\end{pgfscope}%
\begin{pgfscope}%
\definecolor{textcolor}{rgb}{0.000000,0.000000,0.000000}%
\pgfsetstrokecolor{textcolor}%
\pgfsetfillcolor{textcolor}%
\pgftext[x=1.416667in,y=3.724926in,left,base]{\color{textcolor}\rmfamily\fontsize{12.000000}{14.400000}\selectfont \(\displaystyle \gamma_i = 0.02\)}%
\end{pgfscope}%
\begin{pgfscope}%
\pgfsetrectcap%
\pgfsetroundjoin%
\pgfsetlinewidth{1.505625pt}%
\definecolor{currentstroke}{rgb}{0.172549,0.627451,0.172549}%
\pgfsetstrokecolor{currentstroke}%
\pgfsetdash{}{0pt}%
\pgfpathmoveto{\pgfqpoint{0.950000in}{3.550853in}}%
\pgfpathlineto{\pgfqpoint{1.283333in}{3.550853in}}%
\pgfusepath{stroke}%
\end{pgfscope}%
\begin{pgfscope}%
\definecolor{textcolor}{rgb}{0.000000,0.000000,0.000000}%
\pgfsetstrokecolor{textcolor}%
\pgfsetfillcolor{textcolor}%
\pgftext[x=1.416667in,y=3.492519in,left,base]{\color{textcolor}\rmfamily\fontsize{12.000000}{14.400000}\selectfont \(\displaystyle \gamma_i = 0.05\)}%
\end{pgfscope}%
\begin{pgfscope}%
\pgfsetrectcap%
\pgfsetroundjoin%
\pgfsetlinewidth{1.505625pt}%
\definecolor{currentstroke}{rgb}{0.839216,0.152941,0.156863}%
\pgfsetstrokecolor{currentstroke}%
\pgfsetdash{}{0pt}%
\pgfpathmoveto{\pgfqpoint{0.950000in}{3.318445in}}%
\pgfpathlineto{\pgfqpoint{1.283333in}{3.318445in}}%
\pgfusepath{stroke}%
\end{pgfscope}%
\begin{pgfscope}%
\definecolor{textcolor}{rgb}{0.000000,0.000000,0.000000}%
\pgfsetstrokecolor{textcolor}%
\pgfsetfillcolor{textcolor}%
\pgftext[x=1.416667in,y=3.260112in,left,base]{\color{textcolor}\rmfamily\fontsize{12.000000}{14.400000}\selectfont \(\displaystyle \gamma_i = 0.1\)}%
\end{pgfscope}%
\end{pgfpicture}%
\makeatother%
\endgroup%
}
    \end{center}
    \caption{Optimal strategi ved $\alpha=0.1$ og $\beta=0.5$ og ved variation af $\gamma_i$.}\label{fig:renter}
\end{figure}

Når $\gamma_i=0$, kan det aflæses i \autoref{tab:ændring_r}, at forbrugeren fordeler sit forbrug relativt ligeligt. Dette er også illustreret på \autoref{fig:renter}, hvor det ses, at den blå kurve er relativt konstant. Det kan på figuren aflæses, at hældningen af kurverne til ethvert tidspunkt er større for højere indlånsrenter. Dette illustrerer også, at forbruget fordeles mindre ligeligt mellem månederne, når indlånsrenten er højere, hvilket skyldes, at forbrugeren udnytter renters renter mere. Grunden til at forbrugeren ikke vælger at forbruge det hele til sidst, er, at belønningsfunktionen er givet ved kvadratroden af forbruget. 

Herudover kan det i \autoref{tab:ændring_r} aflæses, at forbruget til hver måned og dermed det samlede forbrug er højere, jo større indlånsrenten er. Derfor er den forventede maksimale belønning også højere, når indlånsrenten er større, hvilket stemmer overens med forventningerne. Da sandsynlighedsfordelingen for arbejdstatus er konstant, er dette en konsekvens af den stigende indlånsrente.

%Altså gælder det, at når indlånsrenten stiger, påvirkes den optimale strategi således, at forbrugeren generelt har mere at forbruge. Derudover forbruger han mere til sidst i stedet for at fordele forbruget ligeligt mellem månederne. Den maksimale belønning vil herudover stige, når indlånsrenten stiger.


% 15.375659700000026

% q = 0,5 p = 0,5, r = 0,02
% \begin{align*}
%     \begin{bmatrix}
%         11,90& 11,60& 11,79& 11,93& 12,08& 12,26& 12,46& 12,65& 12,94& 13,26 &13,54& 14,038 &15,14
%     \end{bmatrix}
% \end{align*}
% 14,597563249999993

% q = 0,9 p = 0,7, r = 0,02
% \begin{align*}
%     \begin{bmatrix}
%     8,60 & 8,76 & 9,01 & 9,35 & 9,62 & 9,96 & 10,32 & 10,75 & 11,20 & 11,82 & 12,52 & 13,36 & 14,96
%     \end{bmatrix}
% \end{align*}

% 13,42288233000003

% q = 0,5 p = 0,1, r = 0,1 \\
% \begin{align*}
%     \left[
%     14,50 \ \ 15,04 \ \ 15,68\ \ 16,30\ \ 17,00\ \ 17,73\ \ 18,54\ \ 19,38 \ \ 20,27 \ \ 21,23 \ \ 22,14 \ \ 23,13 \ \  24,25
%     \right]
% \end{align*}

% 17,781626239999994

% q = 0,5 p = 0,1, r = - 0,02 
% \begin{align*}
%     \begin{bmatrix}
%     14,50 & 14,16 & 13,89 & 13,54 & 13,23 & 12,92 & 12,52 &12,17 & 11,80 & 11,40 & 11,07 & 10,78 & 10,93
%     \end{bmatrix}
% \end{align*}

% 14,50115696999999
% 




% Da Markov-kæden er ureducerbar og har et endeligt tilstandsrum, følger det af \autoref{sæt:den_vi_ikke_beviser}, at den har en entydig invariant fordeling. Denne bestemmes ved følgende 
% \begin{align*}
%     \begin{bmatrix}\pi_1 & \pi_2\end{bmatrix} \begin{bmatrix}0,5 & 0,5\\ 0,1 & 0,9\end{bmatrix}&=\begin{bmatrix}\pi_1 & \pi_2\end{bmatrix}
%     \intertext{Dermed fås to ligninger med to ubekendte}
%     0,5\pi_1 + 0,1 \pi_2 &= \pi_1\\
%     0,5\pi_1+0,9\pi_2&=\pi_2\\
%     \pi_2 = 5\pi_1\\
%     4,5\pi_1=4,5\pi_1
% \end{align*}

% \begin{table}[!hbt] \centering
% \caption{Variabler for investeringsproblemet}
% \begin{tabular}{@{}l|l@{}}
% \toprule
% Variabler & Forklaring                     \\ \midrule
% $t$         & Startidspunkt                \\
% $N$         & Sluttidspunkt                \\
% $S_t$       & Opsparing til tiden t        \\
% $S$         & Beslutningsrum               \\
% $r$         & Rente                        \\
% $A_t$       & Forbrug                      \\
% $I$         & Indkomst                     \\
% $\xi_t$     & $\begin{cases}1\ \text{Hvis i arbejde}\\0\ \text{Ellers}\end{cases}$ \\ \bottomrule
% \end{tabular}
% \end{table}



%
%
%

% Det antages, at belønningen er ækvivalent med forbruget. Det vil sige, at markedet er perfekt. Altså er der et vist forbrug til tidspunkter mellem 0 og $N-1$ og der forbruges resten ($S_N$) til tidspunkt $N$. Med andre ord, så er belønningen, $r_t(S_t,A_t)=A_t$ og $r_N(S_N)=S_N$.


% \begin{align*}
%     W=\sum_{t=1}^{N-1} r_t(s_t,a_t)+ r_N(s_N)
% \end{align*}

% Lad nyttefunktionen være givet ved
% \begin{align*}
%     \{U\}_t^N=\begin{cases}A_t\ \text{ for } t<N\\ S_N \text{ for } t=N \end{cases}.
% \end{align*}
% Det antages, desuden, at indkomsten kommer sidst på måneden og den kan dermed ikke bruges i den indeværende måned. Beslutningsrummet er givet ved:
% \begin{align*}
%     S=[0,S_t]
% \end{align*}

% Dermed ønskes at finde strategi ($\pi$), der maksimerer
% \begin{align*}
%     E^\pi \left[\sum_{t=1}^N U_t(A_t)\right]
% \end{align*}



% \begin{figure}
%     \begin{center}
%         \resizebox{12cm}{!}{%% Creator: Matplotlib, PGF backend
%%
%% To include the figure in your LaTeX document, write
%%   \input{<filename>.pgf}
%%
%% Make sure the required packages are loaded in your preamble
%%   \usepackage{pgf}
%%
%% Figures using additional raster images can only be included by \input if
%% they are in the same directory as the main LaTeX file. For loading figures
%% from other directories you can use the `import` package
%%   \usepackage{import}
%%
%% and then include the figures with
%%   \import{<path to file>}{<filename>.pgf}
%%
%% Matplotlib used the following preamble
%%
\begingroup%
\makeatletter%
\begin{pgfpicture}%
\pgfpathrectangle{\pgfpointorigin}{\pgfqpoint{6.400000in}{4.800000in}}%
\pgfusepath{use as bounding box, clip}%
\begin{pgfscope}%
\pgfsetbuttcap%
\pgfsetmiterjoin%
\definecolor{currentfill}{rgb}{1.000000,1.000000,1.000000}%
\pgfsetfillcolor{currentfill}%
\pgfsetlinewidth{0.000000pt}%
\definecolor{currentstroke}{rgb}{1.000000,1.000000,1.000000}%
\pgfsetstrokecolor{currentstroke}%
\pgfsetdash{}{0pt}%
\pgfpathmoveto{\pgfqpoint{0.000000in}{0.000000in}}%
\pgfpathlineto{\pgfqpoint{6.400000in}{0.000000in}}%
\pgfpathlineto{\pgfqpoint{6.400000in}{4.800000in}}%
\pgfpathlineto{\pgfqpoint{0.000000in}{4.800000in}}%
\pgfpathclose%
\pgfusepath{fill}%
\end{pgfscope}%
\begin{pgfscope}%
\pgfsetbuttcap%
\pgfsetmiterjoin%
\definecolor{currentfill}{rgb}{1.000000,1.000000,1.000000}%
\pgfsetfillcolor{currentfill}%
\pgfsetlinewidth{0.000000pt}%
\definecolor{currentstroke}{rgb}{0.000000,0.000000,0.000000}%
\pgfsetstrokecolor{currentstroke}%
\pgfsetstrokeopacity{0.000000}%
\pgfsetdash{}{0pt}%
\pgfpathmoveto{\pgfqpoint{0.800000in}{0.528000in}}%
\pgfpathlineto{\pgfqpoint{5.760000in}{0.528000in}}%
\pgfpathlineto{\pgfqpoint{5.760000in}{4.224000in}}%
\pgfpathlineto{\pgfqpoint{0.800000in}{4.224000in}}%
\pgfpathclose%
\pgfusepath{fill}%
\end{pgfscope}%
\begin{pgfscope}%
\pgfsetbuttcap%
\pgfsetroundjoin%
\definecolor{currentfill}{rgb}{0.000000,0.000000,0.000000}%
\pgfsetfillcolor{currentfill}%
\pgfsetlinewidth{0.803000pt}%
\definecolor{currentstroke}{rgb}{0.000000,0.000000,0.000000}%
\pgfsetstrokecolor{currentstroke}%
\pgfsetdash{}{0pt}%
\pgfsys@defobject{currentmarker}{\pgfqpoint{0.000000in}{-0.048611in}}{\pgfqpoint{0.000000in}{0.000000in}}{%
\pgfpathmoveto{\pgfqpoint{0.000000in}{0.000000in}}%
\pgfpathlineto{\pgfqpoint{0.000000in}{-0.048611in}}%
\pgfusepath{stroke,fill}%
}%
\begin{pgfscope}%
\pgfsys@transformshift{1.025455in}{0.528000in}%
\pgfsys@useobject{currentmarker}{}%
\end{pgfscope}%
\end{pgfscope}%
\begin{pgfscope}%
\definecolor{textcolor}{rgb}{0.000000,0.000000,0.000000}%
\pgfsetstrokecolor{textcolor}%
\pgfsetfillcolor{textcolor}%
\pgftext[x=1.025455in,y=0.430778in,,top]{\color{textcolor}\rmfamily\fontsize{10.000000}{12.000000}\selectfont \(\displaystyle {0}\)}%
\end{pgfscope}%
\begin{pgfscope}%
\pgfsetbuttcap%
\pgfsetroundjoin%
\definecolor{currentfill}{rgb}{0.000000,0.000000,0.000000}%
\pgfsetfillcolor{currentfill}%
\pgfsetlinewidth{0.803000pt}%
\definecolor{currentstroke}{rgb}{0.000000,0.000000,0.000000}%
\pgfsetstrokecolor{currentstroke}%
\pgfsetdash{}{0pt}%
\pgfsys@defobject{currentmarker}{\pgfqpoint{0.000000in}{-0.048611in}}{\pgfqpoint{0.000000in}{0.000000in}}{%
\pgfpathmoveto{\pgfqpoint{0.000000in}{0.000000in}}%
\pgfpathlineto{\pgfqpoint{0.000000in}{-0.048611in}}%
\pgfusepath{stroke,fill}%
}%
\begin{pgfscope}%
\pgfsys@transformshift{1.845289in}{0.528000in}%
\pgfsys@useobject{currentmarker}{}%
\end{pgfscope}%
\end{pgfscope}%
\begin{pgfscope}%
\definecolor{textcolor}{rgb}{0.000000,0.000000,0.000000}%
\pgfsetstrokecolor{textcolor}%
\pgfsetfillcolor{textcolor}%
\pgftext[x=1.845289in,y=0.430778in,,top]{\color{textcolor}\rmfamily\fontsize{10.000000}{12.000000}\selectfont \(\displaystyle {2}\)}%
\end{pgfscope}%
\begin{pgfscope}%
\pgfsetbuttcap%
\pgfsetroundjoin%
\definecolor{currentfill}{rgb}{0.000000,0.000000,0.000000}%
\pgfsetfillcolor{currentfill}%
\pgfsetlinewidth{0.803000pt}%
\definecolor{currentstroke}{rgb}{0.000000,0.000000,0.000000}%
\pgfsetstrokecolor{currentstroke}%
\pgfsetdash{}{0pt}%
\pgfsys@defobject{currentmarker}{\pgfqpoint{0.000000in}{-0.048611in}}{\pgfqpoint{0.000000in}{0.000000in}}{%
\pgfpathmoveto{\pgfqpoint{0.000000in}{0.000000in}}%
\pgfpathlineto{\pgfqpoint{0.000000in}{-0.048611in}}%
\pgfusepath{stroke,fill}%
}%
\begin{pgfscope}%
\pgfsys@transformshift{2.665124in}{0.528000in}%
\pgfsys@useobject{currentmarker}{}%
\end{pgfscope}%
\end{pgfscope}%
\begin{pgfscope}%
\definecolor{textcolor}{rgb}{0.000000,0.000000,0.000000}%
\pgfsetstrokecolor{textcolor}%
\pgfsetfillcolor{textcolor}%
\pgftext[x=2.665124in,y=0.430778in,,top]{\color{textcolor}\rmfamily\fontsize{10.000000}{12.000000}\selectfont \(\displaystyle {4}\)}%
\end{pgfscope}%
\begin{pgfscope}%
\pgfsetbuttcap%
\pgfsetroundjoin%
\definecolor{currentfill}{rgb}{0.000000,0.000000,0.000000}%
\pgfsetfillcolor{currentfill}%
\pgfsetlinewidth{0.803000pt}%
\definecolor{currentstroke}{rgb}{0.000000,0.000000,0.000000}%
\pgfsetstrokecolor{currentstroke}%
\pgfsetdash{}{0pt}%
\pgfsys@defobject{currentmarker}{\pgfqpoint{0.000000in}{-0.048611in}}{\pgfqpoint{0.000000in}{0.000000in}}{%
\pgfpathmoveto{\pgfqpoint{0.000000in}{0.000000in}}%
\pgfpathlineto{\pgfqpoint{0.000000in}{-0.048611in}}%
\pgfusepath{stroke,fill}%
}%
\begin{pgfscope}%
\pgfsys@transformshift{3.484959in}{0.528000in}%
\pgfsys@useobject{currentmarker}{}%
\end{pgfscope}%
\end{pgfscope}%
\begin{pgfscope}%
\definecolor{textcolor}{rgb}{0.000000,0.000000,0.000000}%
\pgfsetstrokecolor{textcolor}%
\pgfsetfillcolor{textcolor}%
\pgftext[x=3.484959in,y=0.430778in,,top]{\color{textcolor}\rmfamily\fontsize{10.000000}{12.000000}\selectfont \(\displaystyle {6}\)}%
\end{pgfscope}%
\begin{pgfscope}%
\pgfsetbuttcap%
\pgfsetroundjoin%
\definecolor{currentfill}{rgb}{0.000000,0.000000,0.000000}%
\pgfsetfillcolor{currentfill}%
\pgfsetlinewidth{0.803000pt}%
\definecolor{currentstroke}{rgb}{0.000000,0.000000,0.000000}%
\pgfsetstrokecolor{currentstroke}%
\pgfsetdash{}{0pt}%
\pgfsys@defobject{currentmarker}{\pgfqpoint{0.000000in}{-0.048611in}}{\pgfqpoint{0.000000in}{0.000000in}}{%
\pgfpathmoveto{\pgfqpoint{0.000000in}{0.000000in}}%
\pgfpathlineto{\pgfqpoint{0.000000in}{-0.048611in}}%
\pgfusepath{stroke,fill}%
}%
\begin{pgfscope}%
\pgfsys@transformshift{4.304793in}{0.528000in}%
\pgfsys@useobject{currentmarker}{}%
\end{pgfscope}%
\end{pgfscope}%
\begin{pgfscope}%
\definecolor{textcolor}{rgb}{0.000000,0.000000,0.000000}%
\pgfsetstrokecolor{textcolor}%
\pgfsetfillcolor{textcolor}%
\pgftext[x=4.304793in,y=0.430778in,,top]{\color{textcolor}\rmfamily\fontsize{10.000000}{12.000000}\selectfont \(\displaystyle {8}\)}%
\end{pgfscope}%
\begin{pgfscope}%
\pgfsetbuttcap%
\pgfsetroundjoin%
\definecolor{currentfill}{rgb}{0.000000,0.000000,0.000000}%
\pgfsetfillcolor{currentfill}%
\pgfsetlinewidth{0.803000pt}%
\definecolor{currentstroke}{rgb}{0.000000,0.000000,0.000000}%
\pgfsetstrokecolor{currentstroke}%
\pgfsetdash{}{0pt}%
\pgfsys@defobject{currentmarker}{\pgfqpoint{0.000000in}{-0.048611in}}{\pgfqpoint{0.000000in}{0.000000in}}{%
\pgfpathmoveto{\pgfqpoint{0.000000in}{0.000000in}}%
\pgfpathlineto{\pgfqpoint{0.000000in}{-0.048611in}}%
\pgfusepath{stroke,fill}%
}%
\begin{pgfscope}%
\pgfsys@transformshift{5.124628in}{0.528000in}%
\pgfsys@useobject{currentmarker}{}%
\end{pgfscope}%
\end{pgfscope}%
\begin{pgfscope}%
\definecolor{textcolor}{rgb}{0.000000,0.000000,0.000000}%
\pgfsetstrokecolor{textcolor}%
\pgfsetfillcolor{textcolor}%
\pgftext[x=5.124628in,y=0.430778in,,top]{\color{textcolor}\rmfamily\fontsize{10.000000}{12.000000}\selectfont \(\displaystyle {10}\)}%
\end{pgfscope}%
\begin{pgfscope}%
\definecolor{textcolor}{rgb}{0.000000,0.000000,0.000000}%
\pgfsetstrokecolor{textcolor}%
\pgfsetfillcolor{textcolor}%
\pgftext[x=3.280000in,y=0.251766in,,top]{\color{textcolor}\rmfamily\fontsize{10.000000}{12.000000}\selectfont \(\displaystyle t\)}%
\end{pgfscope}%
\begin{pgfscope}%
\pgfsetbuttcap%
\pgfsetroundjoin%
\definecolor{currentfill}{rgb}{0.000000,0.000000,0.000000}%
\pgfsetfillcolor{currentfill}%
\pgfsetlinewidth{0.803000pt}%
\definecolor{currentstroke}{rgb}{0.000000,0.000000,0.000000}%
\pgfsetstrokecolor{currentstroke}%
\pgfsetdash{}{0pt}%
\pgfsys@defobject{currentmarker}{\pgfqpoint{-0.048611in}{0.000000in}}{\pgfqpoint{-0.000000in}{0.000000in}}{%
\pgfpathmoveto{\pgfqpoint{-0.000000in}{0.000000in}}%
\pgfpathlineto{\pgfqpoint{-0.048611in}{0.000000in}}%
\pgfusepath{stroke,fill}%
}%
\begin{pgfscope}%
\pgfsys@transformshift{0.800000in}{0.884596in}%
\pgfsys@useobject{currentmarker}{}%
\end{pgfscope}%
\end{pgfscope}%
\begin{pgfscope}%
\definecolor{textcolor}{rgb}{0.000000,0.000000,0.000000}%
\pgfsetstrokecolor{textcolor}%
\pgfsetfillcolor{textcolor}%
\pgftext[x=0.563888in, y=0.836371in, left, base]{\color{textcolor}\rmfamily\fontsize{10.000000}{12.000000}\selectfont \(\displaystyle {10}\)}%
\end{pgfscope}%
\begin{pgfscope}%
\pgfsetbuttcap%
\pgfsetroundjoin%
\definecolor{currentfill}{rgb}{0.000000,0.000000,0.000000}%
\pgfsetfillcolor{currentfill}%
\pgfsetlinewidth{0.803000pt}%
\definecolor{currentstroke}{rgb}{0.000000,0.000000,0.000000}%
\pgfsetstrokecolor{currentstroke}%
\pgfsetdash{}{0pt}%
\pgfsys@defobject{currentmarker}{\pgfqpoint{-0.048611in}{0.000000in}}{\pgfqpoint{-0.000000in}{0.000000in}}{%
\pgfpathmoveto{\pgfqpoint{-0.000000in}{0.000000in}}%
\pgfpathlineto{\pgfqpoint{-0.048611in}{0.000000in}}%
\pgfusepath{stroke,fill}%
}%
\begin{pgfscope}%
\pgfsys@transformshift{0.800000in}{1.513250in}%
\pgfsys@useobject{currentmarker}{}%
\end{pgfscope}%
\end{pgfscope}%
\begin{pgfscope}%
\definecolor{textcolor}{rgb}{0.000000,0.000000,0.000000}%
\pgfsetstrokecolor{textcolor}%
\pgfsetfillcolor{textcolor}%
\pgftext[x=0.563888in, y=1.465025in, left, base]{\color{textcolor}\rmfamily\fontsize{10.000000}{12.000000}\selectfont \(\displaystyle {20}\)}%
\end{pgfscope}%
\begin{pgfscope}%
\pgfsetbuttcap%
\pgfsetroundjoin%
\definecolor{currentfill}{rgb}{0.000000,0.000000,0.000000}%
\pgfsetfillcolor{currentfill}%
\pgfsetlinewidth{0.803000pt}%
\definecolor{currentstroke}{rgb}{0.000000,0.000000,0.000000}%
\pgfsetstrokecolor{currentstroke}%
\pgfsetdash{}{0pt}%
\pgfsys@defobject{currentmarker}{\pgfqpoint{-0.048611in}{0.000000in}}{\pgfqpoint{-0.000000in}{0.000000in}}{%
\pgfpathmoveto{\pgfqpoint{-0.000000in}{0.000000in}}%
\pgfpathlineto{\pgfqpoint{-0.048611in}{0.000000in}}%
\pgfusepath{stroke,fill}%
}%
\begin{pgfscope}%
\pgfsys@transformshift{0.800000in}{2.141904in}%
\pgfsys@useobject{currentmarker}{}%
\end{pgfscope}%
\end{pgfscope}%
\begin{pgfscope}%
\definecolor{textcolor}{rgb}{0.000000,0.000000,0.000000}%
\pgfsetstrokecolor{textcolor}%
\pgfsetfillcolor{textcolor}%
\pgftext[x=0.563888in, y=2.093678in, left, base]{\color{textcolor}\rmfamily\fontsize{10.000000}{12.000000}\selectfont \(\displaystyle {30}\)}%
\end{pgfscope}%
\begin{pgfscope}%
\pgfsetbuttcap%
\pgfsetroundjoin%
\definecolor{currentfill}{rgb}{0.000000,0.000000,0.000000}%
\pgfsetfillcolor{currentfill}%
\pgfsetlinewidth{0.803000pt}%
\definecolor{currentstroke}{rgb}{0.000000,0.000000,0.000000}%
\pgfsetstrokecolor{currentstroke}%
\pgfsetdash{}{0pt}%
\pgfsys@defobject{currentmarker}{\pgfqpoint{-0.048611in}{0.000000in}}{\pgfqpoint{-0.000000in}{0.000000in}}{%
\pgfpathmoveto{\pgfqpoint{-0.000000in}{0.000000in}}%
\pgfpathlineto{\pgfqpoint{-0.048611in}{0.000000in}}%
\pgfusepath{stroke,fill}%
}%
\begin{pgfscope}%
\pgfsys@transformshift{0.800000in}{2.770557in}%
\pgfsys@useobject{currentmarker}{}%
\end{pgfscope}%
\end{pgfscope}%
\begin{pgfscope}%
\definecolor{textcolor}{rgb}{0.000000,0.000000,0.000000}%
\pgfsetstrokecolor{textcolor}%
\pgfsetfillcolor{textcolor}%
\pgftext[x=0.563888in, y=2.722332in, left, base]{\color{textcolor}\rmfamily\fontsize{10.000000}{12.000000}\selectfont \(\displaystyle {40}\)}%
\end{pgfscope}%
\begin{pgfscope}%
\pgfsetbuttcap%
\pgfsetroundjoin%
\definecolor{currentfill}{rgb}{0.000000,0.000000,0.000000}%
\pgfsetfillcolor{currentfill}%
\pgfsetlinewidth{0.803000pt}%
\definecolor{currentstroke}{rgb}{0.000000,0.000000,0.000000}%
\pgfsetstrokecolor{currentstroke}%
\pgfsetdash{}{0pt}%
\pgfsys@defobject{currentmarker}{\pgfqpoint{-0.048611in}{0.000000in}}{\pgfqpoint{-0.000000in}{0.000000in}}{%
\pgfpathmoveto{\pgfqpoint{-0.000000in}{0.000000in}}%
\pgfpathlineto{\pgfqpoint{-0.048611in}{0.000000in}}%
\pgfusepath{stroke,fill}%
}%
\begin{pgfscope}%
\pgfsys@transformshift{0.800000in}{3.399211in}%
\pgfsys@useobject{currentmarker}{}%
\end{pgfscope}%
\end{pgfscope}%
\begin{pgfscope}%
\definecolor{textcolor}{rgb}{0.000000,0.000000,0.000000}%
\pgfsetstrokecolor{textcolor}%
\pgfsetfillcolor{textcolor}%
\pgftext[x=0.563888in, y=3.350986in, left, base]{\color{textcolor}\rmfamily\fontsize{10.000000}{12.000000}\selectfont \(\displaystyle {50}\)}%
\end{pgfscope}%
\begin{pgfscope}%
\pgfsetbuttcap%
\pgfsetroundjoin%
\definecolor{currentfill}{rgb}{0.000000,0.000000,0.000000}%
\pgfsetfillcolor{currentfill}%
\pgfsetlinewidth{0.803000pt}%
\definecolor{currentstroke}{rgb}{0.000000,0.000000,0.000000}%
\pgfsetstrokecolor{currentstroke}%
\pgfsetdash{}{0pt}%
\pgfsys@defobject{currentmarker}{\pgfqpoint{-0.048611in}{0.000000in}}{\pgfqpoint{-0.000000in}{0.000000in}}{%
\pgfpathmoveto{\pgfqpoint{-0.000000in}{0.000000in}}%
\pgfpathlineto{\pgfqpoint{-0.048611in}{0.000000in}}%
\pgfusepath{stroke,fill}%
}%
\begin{pgfscope}%
\pgfsys@transformshift{0.800000in}{4.027865in}%
\pgfsys@useobject{currentmarker}{}%
\end{pgfscope}%
\end{pgfscope}%
\begin{pgfscope}%
\definecolor{textcolor}{rgb}{0.000000,0.000000,0.000000}%
\pgfsetstrokecolor{textcolor}%
\pgfsetfillcolor{textcolor}%
\pgftext[x=0.563888in, y=3.979639in, left, base]{\color{textcolor}\rmfamily\fontsize{10.000000}{12.000000}\selectfont \(\displaystyle {60}\)}%
\end{pgfscope}%
\begin{pgfscope}%
\definecolor{textcolor}{rgb}{0.000000,0.000000,0.000000}%
\pgfsetstrokecolor{textcolor}%
\pgfsetfillcolor{textcolor}%
\pgftext[x=0.508333in,y=2.376000in,,bottom,rotate=90.000000]{\color{textcolor}\rmfamily\fontsize{10.000000}{12.000000}\selectfont Forbrug}%
\end{pgfscope}%
\begin{pgfscope}%
\pgfpathrectangle{\pgfqpoint{0.800000in}{0.528000in}}{\pgfqpoint{4.960000in}{3.696000in}}%
\pgfusepath{clip}%
\pgfsetrectcap%
\pgfsetroundjoin%
\pgfsetlinewidth{1.505625pt}%
\definecolor{currentstroke}{rgb}{0.121569,0.466667,0.705882}%
\pgfsetstrokecolor{currentstroke}%
\pgfsetdash{}{0pt}%
\pgfpathmoveto{\pgfqpoint{1.025455in}{1.242929in}}%
\pgfpathlineto{\pgfqpoint{1.435372in}{1.245277in}}%
\pgfpathlineto{\pgfqpoint{1.845289in}{1.248243in}}%
\pgfpathlineto{\pgfqpoint{2.255207in}{1.256107in}}%
\pgfpathlineto{\pgfqpoint{2.665124in}{1.254946in}}%
\pgfpathlineto{\pgfqpoint{3.075041in}{1.258508in}}%
\pgfpathlineto{\pgfqpoint{3.484959in}{1.265321in}}%
\pgfpathlineto{\pgfqpoint{3.894876in}{1.266941in}}%
\pgfpathlineto{\pgfqpoint{4.304793in}{1.276951in}}%
\pgfpathlineto{\pgfqpoint{4.714711in}{1.290789in}}%
\pgfpathlineto{\pgfqpoint{5.124628in}{1.315555in}}%
\pgfpathlineto{\pgfqpoint{5.534545in}{1.371148in}}%
\pgfusepath{stroke}%
\end{pgfscope}%
\begin{pgfscope}%
\pgfpathrectangle{\pgfqpoint{0.800000in}{0.528000in}}{\pgfqpoint{4.960000in}{3.696000in}}%
\pgfusepath{clip}%
\pgfsetrectcap%
\pgfsetroundjoin%
\pgfsetlinewidth{1.505625pt}%
\definecolor{currentstroke}{rgb}{1.000000,0.498039,0.054902}%
\pgfsetstrokecolor{currentstroke}%
\pgfsetdash{}{0pt}%
\pgfpathmoveto{\pgfqpoint{1.025455in}{1.092052in}}%
\pgfpathlineto{\pgfqpoint{1.435372in}{1.098338in}}%
\pgfpathlineto{\pgfqpoint{1.845289in}{1.252385in}}%
\pgfpathlineto{\pgfqpoint{2.255207in}{1.248284in}}%
\pgfpathlineto{\pgfqpoint{2.665124in}{1.269254in}}%
\pgfpathlineto{\pgfqpoint{3.075041in}{1.289618in}}%
\pgfpathlineto{\pgfqpoint{3.484959in}{1.349052in}}%
\pgfpathlineto{\pgfqpoint{3.894876in}{1.375858in}}%
\pgfpathlineto{\pgfqpoint{4.304793in}{1.430302in}}%
\pgfpathlineto{\pgfqpoint{4.714711in}{1.465146in}}%
\pgfpathlineto{\pgfqpoint{5.124628in}{1.558732in}}%
\pgfpathlineto{\pgfqpoint{5.534545in}{1.657107in}}%
\pgfusepath{stroke}%
\end{pgfscope}%
\begin{pgfscope}%
\pgfpathrectangle{\pgfqpoint{0.800000in}{0.528000in}}{\pgfqpoint{4.960000in}{3.696000in}}%
\pgfusepath{clip}%
\pgfsetrectcap%
\pgfsetroundjoin%
\pgfsetlinewidth{1.505625pt}%
\definecolor{currentstroke}{rgb}{0.172549,0.627451,0.172549}%
\pgfsetstrokecolor{currentstroke}%
\pgfsetdash{}{0pt}%
\pgfpathmoveto{\pgfqpoint{1.025455in}{0.884596in}}%
\pgfpathlineto{\pgfqpoint{1.435372in}{1.029853in}}%
\pgfpathlineto{\pgfqpoint{1.845289in}{1.048641in}}%
\pgfpathlineto{\pgfqpoint{2.255207in}{1.166429in}}%
\pgfpathlineto{\pgfqpoint{2.665124in}{1.274806in}}%
\pgfpathlineto{\pgfqpoint{3.075041in}{1.368347in}}%
\pgfpathlineto{\pgfqpoint{3.484959in}{1.480708in}}%
\pgfpathlineto{\pgfqpoint{3.894876in}{1.577565in}}%
\pgfpathlineto{\pgfqpoint{4.304793in}{1.736264in}}%
\pgfpathlineto{\pgfqpoint{4.714711in}{1.906603in}}%
\pgfpathlineto{\pgfqpoint{5.124628in}{2.074880in}}%
\pgfpathlineto{\pgfqpoint{5.534545in}{2.288074in}}%
\pgfusepath{stroke}%
\end{pgfscope}%
\begin{pgfscope}%
\pgfpathrectangle{\pgfqpoint{0.800000in}{0.528000in}}{\pgfqpoint{4.960000in}{3.696000in}}%
\pgfusepath{clip}%
\pgfsetrectcap%
\pgfsetroundjoin%
\pgfsetlinewidth{1.505625pt}%
\definecolor{currentstroke}{rgb}{0.839216,0.152941,0.156863}%
\pgfsetstrokecolor{currentstroke}%
\pgfsetdash{}{0pt}%
\pgfpathmoveto{\pgfqpoint{1.025455in}{0.696000in}}%
\pgfpathlineto{\pgfqpoint{1.435372in}{0.834291in}}%
\pgfpathlineto{\pgfqpoint{1.845289in}{0.900886in}}%
\pgfpathlineto{\pgfqpoint{2.255207in}{1.077417in}}%
\pgfpathlineto{\pgfqpoint{2.665124in}{1.239115in}}%
\pgfpathlineto{\pgfqpoint{3.075041in}{1.450883in}}%
\pgfpathlineto{\pgfqpoint{3.484959in}{1.703425in}}%
\pgfpathlineto{\pgfqpoint{3.894876in}{1.994103in}}%
\pgfpathlineto{\pgfqpoint{4.304793in}{2.354693in}}%
\pgfpathlineto{\pgfqpoint{4.714711in}{2.821300in}}%
\pgfpathlineto{\pgfqpoint{5.124628in}{3.368177in}}%
\pgfpathlineto{\pgfqpoint{5.534545in}{4.056000in}}%
\pgfusepath{stroke}%
\end{pgfscope}%
\begin{pgfscope}%
\pgfsetrectcap%
\pgfsetmiterjoin%
\pgfsetlinewidth{0.803000pt}%
\definecolor{currentstroke}{rgb}{0.000000,0.000000,0.000000}%
\pgfsetstrokecolor{currentstroke}%
\pgfsetdash{}{0pt}%
\pgfpathmoveto{\pgfqpoint{0.800000in}{0.528000in}}%
\pgfpathlineto{\pgfqpoint{0.800000in}{4.224000in}}%
\pgfusepath{stroke}%
\end{pgfscope}%
\begin{pgfscope}%
\pgfsetrectcap%
\pgfsetmiterjoin%
\pgfsetlinewidth{0.803000pt}%
\definecolor{currentstroke}{rgb}{0.000000,0.000000,0.000000}%
\pgfsetstrokecolor{currentstroke}%
\pgfsetdash{}{0pt}%
\pgfpathmoveto{\pgfqpoint{5.760000in}{0.528000in}}%
\pgfpathlineto{\pgfqpoint{5.760000in}{4.224000in}}%
\pgfusepath{stroke}%
\end{pgfscope}%
\begin{pgfscope}%
\pgfsetrectcap%
\pgfsetmiterjoin%
\pgfsetlinewidth{0.803000pt}%
\definecolor{currentstroke}{rgb}{0.000000,0.000000,0.000000}%
\pgfsetstrokecolor{currentstroke}%
\pgfsetdash{}{0pt}%
\pgfpathmoveto{\pgfqpoint{0.800000in}{0.528000in}}%
\pgfpathlineto{\pgfqpoint{5.760000in}{0.528000in}}%
\pgfusepath{stroke}%
\end{pgfscope}%
\begin{pgfscope}%
\pgfsetrectcap%
\pgfsetmiterjoin%
\pgfsetlinewidth{0.803000pt}%
\definecolor{currentstroke}{rgb}{0.000000,0.000000,0.000000}%
\pgfsetstrokecolor{currentstroke}%
\pgfsetdash{}{0pt}%
\pgfpathmoveto{\pgfqpoint{0.800000in}{4.224000in}}%
\pgfpathlineto{\pgfqpoint{5.760000in}{4.224000in}}%
\pgfusepath{stroke}%
\end{pgfscope}%
\begin{pgfscope}%
\pgfsetbuttcap%
\pgfsetmiterjoin%
\definecolor{currentfill}{rgb}{1.000000,1.000000,1.000000}%
\pgfsetfillcolor{currentfill}%
\pgfsetfillopacity{0.800000}%
\pgfsetlinewidth{1.003750pt}%
\definecolor{currentstroke}{rgb}{0.800000,0.800000,0.800000}%
\pgfsetstrokecolor{currentstroke}%
\pgfsetstrokeopacity{0.800000}%
\pgfsetdash{}{0pt}%
\pgfpathmoveto{\pgfqpoint{0.916667in}{3.161038in}}%
\pgfpathlineto{\pgfqpoint{2.091426in}{3.161038in}}%
\pgfpathquadraticcurveto{\pgfqpoint{2.124759in}{3.161038in}}{\pgfqpoint{2.124759in}{3.194372in}}%
\pgfpathlineto{\pgfqpoint{2.124759in}{4.107333in}}%
\pgfpathquadraticcurveto{\pgfqpoint{2.124759in}{4.140667in}}{\pgfqpoint{2.091426in}{4.140667in}}%
\pgfpathlineto{\pgfqpoint{0.916667in}{4.140667in}}%
\pgfpathquadraticcurveto{\pgfqpoint{0.883333in}{4.140667in}}{\pgfqpoint{0.883333in}{4.107333in}}%
\pgfpathlineto{\pgfqpoint{0.883333in}{3.194372in}}%
\pgfpathquadraticcurveto{\pgfqpoint{0.883333in}{3.161038in}}{\pgfqpoint{0.916667in}{3.161038in}}%
\pgfpathclose%
\pgfusepath{stroke,fill}%
\end{pgfscope}%
\begin{pgfscope}%
\pgfsetrectcap%
\pgfsetroundjoin%
\pgfsetlinewidth{1.505625pt}%
\definecolor{currentstroke}{rgb}{0.121569,0.466667,0.705882}%
\pgfsetstrokecolor{currentstroke}%
\pgfsetdash{}{0pt}%
\pgfpathmoveto{\pgfqpoint{0.950000in}{4.015667in}}%
\pgfpathlineto{\pgfqpoint{1.283333in}{4.015667in}}%
\pgfusepath{stroke}%
\end{pgfscope}%
\begin{pgfscope}%
\definecolor{textcolor}{rgb}{0.000000,0.000000,0.000000}%
\pgfsetstrokecolor{textcolor}%
\pgfsetfillcolor{textcolor}%
\pgftext[x=1.416667in,y=3.957333in,left,base]{\color{textcolor}\rmfamily\fontsize{12.000000}{14.400000}\selectfont \(\displaystyle \gamma_i = 0\)}%
\end{pgfscope}%
\begin{pgfscope}%
\pgfsetrectcap%
\pgfsetroundjoin%
\pgfsetlinewidth{1.505625pt}%
\definecolor{currentstroke}{rgb}{1.000000,0.498039,0.054902}%
\pgfsetstrokecolor{currentstroke}%
\pgfsetdash{}{0pt}%
\pgfpathmoveto{\pgfqpoint{0.950000in}{3.783260in}}%
\pgfpathlineto{\pgfqpoint{1.283333in}{3.783260in}}%
\pgfusepath{stroke}%
\end{pgfscope}%
\begin{pgfscope}%
\definecolor{textcolor}{rgb}{0.000000,0.000000,0.000000}%
\pgfsetstrokecolor{textcolor}%
\pgfsetfillcolor{textcolor}%
\pgftext[x=1.416667in,y=3.724926in,left,base]{\color{textcolor}\rmfamily\fontsize{12.000000}{14.400000}\selectfont \(\displaystyle \gamma_i = 0.02\)}%
\end{pgfscope}%
\begin{pgfscope}%
\pgfsetrectcap%
\pgfsetroundjoin%
\pgfsetlinewidth{1.505625pt}%
\definecolor{currentstroke}{rgb}{0.172549,0.627451,0.172549}%
\pgfsetstrokecolor{currentstroke}%
\pgfsetdash{}{0pt}%
\pgfpathmoveto{\pgfqpoint{0.950000in}{3.550853in}}%
\pgfpathlineto{\pgfqpoint{1.283333in}{3.550853in}}%
\pgfusepath{stroke}%
\end{pgfscope}%
\begin{pgfscope}%
\definecolor{textcolor}{rgb}{0.000000,0.000000,0.000000}%
\pgfsetstrokecolor{textcolor}%
\pgfsetfillcolor{textcolor}%
\pgftext[x=1.416667in,y=3.492519in,left,base]{\color{textcolor}\rmfamily\fontsize{12.000000}{14.400000}\selectfont \(\displaystyle \gamma_i = 0.05\)}%
\end{pgfscope}%
\begin{pgfscope}%
\pgfsetrectcap%
\pgfsetroundjoin%
\pgfsetlinewidth{1.505625pt}%
\definecolor{currentstroke}{rgb}{0.839216,0.152941,0.156863}%
\pgfsetstrokecolor{currentstroke}%
\pgfsetdash{}{0pt}%
\pgfpathmoveto{\pgfqpoint{0.950000in}{3.318445in}}%
\pgfpathlineto{\pgfqpoint{1.283333in}{3.318445in}}%
\pgfusepath{stroke}%
\end{pgfscope}%
\begin{pgfscope}%
\definecolor{textcolor}{rgb}{0.000000,0.000000,0.000000}%
\pgfsetstrokecolor{textcolor}%
\pgfsetfillcolor{textcolor}%
\pgftext[x=1.416667in,y=3.260112in,left,base]{\color{textcolor}\rmfamily\fontsize{12.000000}{14.400000}\selectfont \(\displaystyle \gamma_i = 0.1\)}%
\end{pgfscope}%
\end{pgfpicture}%
\makeatother%
\endgroup%
}
%     \end{center}
%     \caption{Gennemsnitlig forbrug over tid ved p=0,1 og q=0,5}
% \end{figure}



% \begin{figure}
%     \begin{center}
%         \resizebox{12cm}{!}{\input{Afsnit/PGF/ændring_i_p.pgf}}
%     \end{center}
%     \caption{Gennemsnitlig forbrug over tid ved p=0,1 og q=0,5}
% \end{figure}

