Dette kapitel er baseret på \cite[s. 444-462]{olofsson2005probability} og  \cite[s. 205-216]{oxford}.

På baggrund af teorien i \autoref{kapitel:sandsynlighed} er det hermed muligt at definere og analysere \textit{Markov-kæder}. 

En Markov-kæde er en \textit{stokastisk proces}. En stokastisk proces er en sekvens af tilfældige variabler $\bm X$. De tilfældige variabler i $\bm X$ antager værdier i \textit{tilstandsrummet}, $S$, hvor værdierne i $S$ kaldes \textit{tilstande}. $\bm X$ er indekseret af en mængde, $T$, som kaldes for \textit{indeksmængden}. Indeksmængden kan eksempelvis beskrive tiden. 
I dette projekt indekseres $\bm X$ over de naturlige tal. 
\section{Markov-kæder}
Markov-kæder er en stokastisk proces med en bestemt \textbf{\textit{afhængighedsstruktur}}. For Markov-kæder gælder, at sandsynligheden for en tilfældig variabel til indekset $n+1$ afhænger kun af udfaldet at den tilfældige variabel indekset $n$. Dette er formuleret i følgende definition.

\begin{minipage}\textwidth
\begin{defn}\textbf{Markov egenskaben} %Ny definition
\newline
Lad $\bm{X}=(X_n:n\geq 0)$ være en sekvens at diskrete tilfældige variabler, som tager værdier i en mængde $S$.
Sekvensen $\bm{X}$ er en \textit{Markov-kæde}, hvis den opfylder \textit{Markov egenskaben}
\begin{align*}
    P(X_{n+1} = j | X_0 = i_0, \cdots, X_{n-1} = i_{n-1}, X_n = i_n) =  P(X_{n+1} = j | X_n = i_n)
\end{align*}
for alle $n\geq 0$, samt $i, i_0, i_1, \cdots i_{n+1}$ i $S$.
Markov-kæden kaldes homogen, hvis den betingede sandsynlighed, $P(X_{n+1}=j|X_n=i)$ ikke afhænger af $n$ for alle $i,j\in S$.
\end{defn}
\end{minipage}

I ovenstående definition, siges $X_n$ at være \textit{tilstanden} af kæden til tiden $n$, hvoraf tilstandsrummet, $S$ både kan være endeligt og tællelig uendeligt. 
I de følgende afsnit antages alle Markov-kæder at være homogene.


%Hvis sandsynligheden, $P(X_{n+1}=j|X_n=i)=P(X_{n}=j|X_{n-1}=i)$, så er Markov-kæden \textit{tids-homogen}. Altså, afhænger den kun af $i,j$, men ikke $n$. 

\subsection{Overgangssandsynlighed}
\begin{minipage}\textwidth
\begin{defn}\textbf{Overgangssandsynligheden} %Ny definition
\newline
Lad $i, j \in S$. Så er overgangssandsynligheden for en Markov-kæde givet ved $$p_{ij}=P(X_{n+1}=j|X_n=i)$$
\end{defn}
\end{minipage}

Overgangssandsynligheden, $p_{ij}$, beskriver sandsynligheden for at nå fra tilstanden i $i$ til tilstanden i $j$.

\begin{minipage}\textwidth
\begin{defn}\textbf{Overgangsmatrice} %Ny definition
\newline
Lad $p_{ij}$ for $i,j \in S$ være overgangssandsynligheden for en Markov-kæde. Så gælder det at matricen $T = [p_{ij}]$, er overgangsmatricen for Markov-kæden, hvor $p_{ij}$ er den $ij'te$ indgang i matricen.
\end{defn}
\end{minipage}

Følgende korrollar omhandler disse mængder:
\begin{enumerate}[label=(\alph*)]
    \item \textit{Overgangsmatricen} er givet ved $P=(p_{ij}:i,j\in S)$
    \item \textit{Begyndelsesfordelingen}, $\bm{\lambda}=(\lambda_i:i\in S)$, hvor $\lambda_i=P(X_0=i)$.
\end{enumerate}

\begin{minipage}\textwidth
\begin{kor} \textbf{} %Ny proposition
\newline
\begin{enumerate}[label=(\alph*)]
    \item Vektoren, $\bm{\lambda}$ er en begyndelsesfordeling, hvis $\lambda_i\geq 0$ for $i\in S$ og $\sum_{i\in S}\lambda_i=1$.
    \item Matricen, $P=(p_{ij})$ er en overgangsmatrice hvis 
    \begin{enumerate}[label=(\roman*)]
    \item $p_{ij}\geq0$ for $i,j\in S$ og
    \item $\sum_{j\in S}p_{ij}=1$ for $i\in S$, sådan at rækkerne i $P$ summer henholdsvist til $1$. 
    \end{enumerate}
\end{enumerate}
\end{kor}
\end{minipage}
\begin{bev} \textbf{} %Nyt bevis
\newline
For ethvert $i$ og $j$ er $\lambda_i$ og $p_{ij}$ frekvensfunktioner, hvorfor det følger af \autoref{prop:frekvensfunktion}, at 
de begge er ikke-negative, samt at summen over tilstandsrummet er 1.
\end{bev}

\begin{minipage}\textwidth
\begin{thmx} \textbf{} %Ny sætning
\newline
Lad $\bm{\lambda}$ være en fordeling og $P$ - en overgangsmatrice. Den tilfældige sekvens, $\mathbf{X}=(X_n:n\geq0)$ er en Markov-kæde med begyndelsesfordelingen $\bm{\lambda}$ og overgangsmatricen, $P$ hvis og kun hvis
\begin{align}\label{markov-kæde-hovedsætning}
    P(X_0=i_0,X_1=i_1,\dots, X_n=i_n)=\lambda_{i_0}p_{i_0i_1}\cdots p_{i_{n-1}i_n}
\end{align}
for alle $n\geq0$ og $i_0,i_1,\dots,i_n\in S$.
\end{thmx}
\end{minipage}
\begin{bev} \textbf{}
\newline
Lad $A_n=\{X_n=i_n\}$ sådan at $\bm A_n=\bigcap_0^n A_n$. Så kan \eqref{markov-kæde-hovedsætning} omskrives til
\begin{align}\label{omskrivning}
    P(\bm A_n)=\lambda_{i_0}p_{i_0i_1}\cdots p_{i_{n-1}i_n}
\end{align}
Antag, at $\bm X$ er en Markov-kæde med begyndelsesfordeling, $\lambda$ og overgangsmatrice, $P$. Der ønskes at bevise \eqref{omskrivning} ved induktion for $n$. Tilfældet ved $n=0$ er trivielt. Der antages derfor nu, at $N\geq1$ sådan at $n<N$. Så gælder, at
\begin{align*}
    P(\bm A_N)&=P(\bm A_{N-1})P(A_N|\bm A_{N-1})\\
    &=P(\bm A_{N-1})P(A_N|A_{N-1})\\
\end{align*}
Eftersom der per definition gælder, at
$P(A_N|A_{N-1})=p_{i_{N-1}i_N}$, er induktionsskridtet færdigt. 
Antag, at \eqref{omskrivning} holder for alle $n$ og sekvenser $(i_m)$. Ved at lade $n=0$ indses det, at begyndelsesfordelingen er givet ved $P(X_0=i_0)=\lambda_{i_0}$. Det følger af \eqref{omskrivning}, at
\begin{align*}
    P(A_{n+1}|\bm A_n)&=\frac{P(\bm A_{n+1})}{P(\bm A_n)}\\
    &=p_{i_ni_{n+1}}
\end{align*}
Eftersom dette ikke afhænger af tilstandende, $i_0,i_1,\dots,i_{n-1}$, har det konsekvensen, at $\bm X$er en homogen Markov-kæde med overgangsmatricen $P$.

\end{bev}

Det er muligt at visualisere en Markov-kæde som en graf kaldet en \textit{overgangsgraf}. 

Nedenstående eksempel, \autoref{eks:overgangsmatrice}, illustrerer brugen af overgangsgrafer og overgangsmatricer.

\begin{eks}\textbf{} \label{eks:overgangsmatrice}%Nyt eksempel
\newline
Et bestemt gen i en plante har to alleler, $A$ og $a$. Denne genotype kan dermed være $AA, Aa$ eller $aa$, altså er tilstandsrummet $S = \{AA, Aa, aa\}$. Antag, at en plante krydses med sig selv og ét afkom bliver valgt, som herefter krydses med sig selv og så videre. Dette er en Markov-kæde, da afkommet kun afhænger at den forgående plante. 

Der gælder, at afkommet af $AA$ og $aa$ altid er sig selv. For gentypen $Aa$ er sandsynligheden for at afkommet har gentypen $Aa$, $\frac{1}{2}$, $AA$, $\frac{1}{4}$ og $aa$, $\frac{1}{4}$. Dermed ser overgangsgraf ud på følgende måde

\begin{center}
	\begin{tikzpicture}[->, >=stealth', auto, semithick, node distance=3cm]
	\tikzstyle{every state}=[fill=white,draw=black,thick,text=black,scale=1]
	\node[state]    (A)                     {$AA$};
	\node[state]    (B)[right of=A]   {$Aa$};
	\node[state]    (C)[right of=B]   {$aa$};
	\path
	(A) edge[loop left]			node{$1$}	(A)
	(B) edge[bend left,below]	node{$1/4$}	(A)
	(B) edge[loop above]		node{$\frac{1}{2}$}	(B)
	(B) edge[bend left,below]	node{$1/4$}	(C)
	(C) edge[loop right]		node{$1$}	(C);
	%\node[above=0.5cm] (A){Patch G};
	%\draw[red] ($(D)+(-1.5,0)$) ellipse (2cm and 3.5cm)node[yshift=3cm]{Patch H};
	\end{tikzpicture}
\end{center}
Det er derudover muligt at lave overgangsmatricen hvor $p_{ij}$ er indgang $i,j$. 
\begin{align*}
    P=
\begin{bmatrix}
1 & 0 & 0 \\
\frac{1}{4} & \frac{1}{2} & \frac{1}{4}\\
0 & 0 & 1
\end{bmatrix}
\end{align*}
\end{eks}

\subsection{Tidsdynamik af Markov-kæder}
Det er muligt at analysere hvordan Markov-kæder udvikler sig over tid. Det er muligt ud fra kendte sandsynligheder at beregne fremtidige sandsynligheder. Overgangsmatricen består af $p_{ij}$ som er \textit{et-trins overgangssandsynligheder}, hvor man generelt kan skrive \textit{n'te-trins overgangssandsynligheder} som 
\begin{align*}
    p_{ij}^{(n)} = P ( X_n = j | X_0 = i)
\end{align*}
Hvorom det gælder at den n'te-trins overgangsmatrice er $P^{(n)}$. Denne matrice opfylder følgende

\begin{minipage}\textwidth
\begin{thmx} \textbf{Chapman-Kolmogorov ligningen}\label{sæt:chapman-kolmogrov} %Ny sætning
\newline
Det gælder at 
\begin{align*}
    p_{ij}^{(n+m)} = \sum_{k \in S} p_{ik}^{(n)}p_{kj}^{(m)}
\end{align*}
for alle $m,n$ og alle $i,j \in S$. Altså gælder $P^{(n+m)} = P^{(n)}P^{(m)} = P^{n}P^{m}$.
\end{thmx}
\end{minipage}

\begin{bev} \textbf{} %Nyt bevis
\newline
Ud fra Loven om total sandsynlighed, \autoref{sæt:loven_om_total_sandsynlighed}, gælder det at
\begin{align*}
    p_{ij}^{(n+m)} &= P(X_{n+m} = j | X_0 = i) = \sum_{k \in S} P (X_{n+m} = j | X_n = k, X_0 = i)P(X_n=k|X_0=i)
    \intertext{Af Markov egenskaben får dermed at}
    p_{ij}^{(n+m)} &=\sum_{k \in S} P (X_{n+m} = j | X_n = k, X_0 = i)P(X_n=k|X_0=i) \\
    &= \sum_{k \in S} P (X_{n+m} = j | X_n = k)P(X_n=k|X_0=i) \\ 
    &= \sum_{k \in S} p_{ik}p_{kj}
\end{align*}
Dermed er \autoref{sæt:chapman-kolmogrov} bevist.
\end{bev}
Når $P^{(n)}$ er bestemt er det muligt bestemme Markov-kædens langsigtet adfærd. Dette gøres ved at bestemme
\begin{align*}
    \lim_{n \to \infty} P^{(n)}
\end{align*}
Hvis de asymptotiske sandsynligheder ikke afhænger af \textbf{?begyndelsestilstanden?} kaldes fordelingen på det givne tilstandsrum for en \textbf{\textit{grænsefordelling}}.

\subsection{Klassifikation af tilstande}
Når Markov-kæder skal analyseres, er en vigtig del at se på om tilstande kan "nå" hinanden eller ej. Derfor defineres følgende.\\
\begin{minipage}\textwidth
\begin{defn}\textbf{Tilgængelighed} \label{def:tilgængelighed} %Ny definition
\newline
Hvis $p_{ij}^{(n)}>0$ for et givent $n$, siges tilstanden, $j$, at være \textit{tilgængeligt} fra tilstanden $i$. Dette noteres, $i\to j$. Hvis $i\to j$ og $j\to i$, siges det, at $i$ og $j$ \textit{kommunikerer}, hvilket noteres $i\leftrightarrow j$. 
\end{defn}
\end{minipage}

Hvis en tilstand, $i \in S$, kun kommunikerer med én tilstand, $j \in S$, danner disse en \textit{kommunikerende klasse}. Det er muligt at alle tilstande i $S$ kommunikerer med hinanden, hvilket fører til følgende definition
 
\begin{minipage}\textwidth
\begin{defn}\textbf{} %Ny definition
\newline
Hvis alle tilstande i $S$ kommunikerer med hinanden, siges Markov-kæden at være $ureducerbar$.
\end{defn}
\end{minipage}

Ud over at det er vigtigt at vide om tilstande kommunikerer er det også vigtigt at vide om man vender tilbage til en tilstand. I følgende definition klassificeres tilstande i forhold til om det er sikkert at man vender tilbage til den. 

\begin{minipage}\textwidth
\begin{defn}\textbf{} %Ny definition
\newline
Lad $i\in S$ være en tilstand, $\tau_i$ være antallet af skridt før kæden besøger $i$ og $P_i$ være sandsynlighedsfordellingen for kæden i begyndelsestilstanden $X_0=i$. Altså
\begin{align*}
    \tau_i=min\{n\geq1:X_n=i\}
\end{align*}
hvor $\tau_i=\infty$ hvis $i$ aldrig bliver besøgt. Hvis $P_i(\tau_i<\infty)=1$, siges tilstanden $i$ at være \textit{tilbagevendende}. Hvis den ikke er tilbagevendende, siges den at være  \textit{forbigående}.
\end{defn}
\end{minipage}

Altså gælder det at hvis en tilstand er tilbagevendende vil Markov-kæden med sikkerhed vende tilbage til tilstanden. Hvis tilstanden derimod er forbigående vil der være en sandsynlighed for, at Markov-kæden ikke vender tilbage til tilstanden. For ureducerbare Markov-kæder gælder følgende

\begin{minipage}\textwidth
\begin{kor} \textbf{} \label{kor:enten_forbigå_eller_tilbagevend}%Nyt k
\newline
I en ureducerbar Markov-kæde er alle tilstande enten forbigående eller tilbagevendende.
\end{kor}
\end{minipage}

\begin{bev} \textbf{} %Nyt bevis
\newline
Da Markov-kæden er ureducerbar gælder det at $i \leftrightarrow j$ for alle $i,j \in S$. Af \autoref{def:tilgængelighed} gælder det da at $i \to j$ og $j \to i$, og dermed eksisterer der $n,m$ således at
\begin{align*}
    p_{ij}^{(n)} > 0 \quad \text{ og } \quad p_{ji}^{(m)} > 0
\end{align*}

\end{bev}

Det er herved muligt at konkluderer at alle tilstande er tilbagevendende eller forbigående hvis dette gælder for én tilstand. For et endeligt tilstandsrum er der givet følgende korollar

\begin{minipage}\textwidth
\begin{kor} \textbf{} \label{kor:forbigående}%Nyt korollar
\newline
Lad S være endelig. En tilstand siges at være forbigående hvis og kun hvis der eksisterer en anden tilstand j således at $i \to j$ men $j \not\to i.$
\end{kor}
\end{minipage}

%Af \autoref{kor:forbigående} fremgår det, at når en Markov-kæde går gennem en forbigående tilstand, er der en sandsynlighed for, at kæden ikke vil gå tilbage igen.

I et endeligt tilstandrum er det kun muligt for en Markov-kæde at være forbigående, hvis der er endnu en tilstand, der kan nåes, men som ikke har en rute tilbage. I en uendeligt tilstandsrum er det muligt, at der kun er forbigående tilstande, selvom de alle er kommunikerende. 

For at bevise Proposition \ref{tilbagevendende}, introduceres følgende genererende funktioner. Lad $i,j\in S$ og
$$p_{ij}=\sum_{n=0}^\infty p_{ij}(n)s^n, \quad F_{ij}(s)=\sum_{n=0}^\infty p_i(\tau_j=n)s^n$$
Med konventionerne, $p_i(\tau_j=0)=0$ og $p_{ij}(0)=\delta_{ij}$ følger det, at \textit{Kronecker} deltaet defineres som
\begin{align*}
    \delta_{ij}=\begin{cases}1\ \text{hvis } i=j,\\0\ ellers\end{cases}
\end{align*}


\begin{lem}\label{overgangssandynlighed Kronecker}
For $i,j\in S$ gælder, at
\begin{align*}
    p_{ij}(s)=\delta_{ij}+F_{ij}(s)p_{jj}(s),\quad s\in(-1,1]
\end{align*}
\end{lem}

\begin{bev} %Nyt bevis
Ved at betinge værdien af $\tau_j$,
gælder ifølge Loven om Total Sandsynlighed, at
\begin{align}\label{tau_j betinget}
    p_{ij}(n)=\sum_{m=1}^\infty p_i(X_n=j|\tau_j=m)p_i(\tau_j=m), \quad n\geq 1
\end{align}
hvor $X_n$ er betinget af $\tau_j$.
Summanden er $0$ for $m>n$, eftersom det første besøg til $j$ ikke har sket tiden $n$. For $m\leq n$ gælder, at
\begin{align*}
    p_i(X_n=j|\tau_j=m)=p_i(X_n=j|X_m=j, H)
\end{align*}
hvor $H=\{X_r\neq j \text{ for } 1\leq r <m\}$ er en hændelse defineret før tiden $m$.
Det følger af antagelsen om homogenitet og den udvidede Markov egenskab, at
\begin{align*}
    p_i(X_n=j|\tau_j=m)=p(X_n=j|X_m=j)=p_j(X_{n-m}=j)
\end{align*}
Ved at indsætte dette i \eqref{tau_j betinget} fås
\begin{align*}
    p_{ij}(n)=\sum_{m=1}^n p_{jj}(n-m)p_i(\tau_j=m), \quad n\geq 1
\end{align*}
Ved at multiplicere ligningen med $s^n$ og summe over $n\geq 1$ fås
\begin{align*}
    p_{ij}(s)-p_{ij}(0)=p_{jj}(s)F_{ij}(s)
\end{align*}
Dette beviser sætningen, eftersom $p_{ij}(0)=\delta_{ij}$.
\end{bev}

\begin{minipage}\textwidth
\begin{thmx}\label{tilbagevendende} \textbf{} %Ny proposition
\newline
Tilstanden, $i$ er tilbagevendende, hvis
\begin{align*}
 \sum_{n=1}^\infty p_{ii}^{(n)}=\infty
\end{align*}
\end{thmx}
\end{minipage}
\begin{bev}
Hvis $i=j$ følger det af \eqref{overgangssandynlighed Kronecker}, at
\begin{align}\label{i=j}
    p_{ii}(s)=\frac{1}{1-F_{ii}(s)} \text{ for } |s|<1
\end{align}
Der gælder ifølge Abel's lemma, at
\begin{align*}
    F_{ii}(s)\uparrow F_{ii}(1), \quad p_{ii}\uparrow\sum_{n=0}^\infty p_{ii}(n)
\end{align*}
Det følger af \eqref{i=j}, at
\begin{align*}
    \sum_{n=0}^\infty p_{ii}(n)=\infty \text{ hvis og kun hvis } F_{ii}(1)=1
\end{align*}

\end{bev}

\subsection{Stationære distributioner}


\begin{minipage}\textwidth
\begin{defn}\textbf{} %Ny definition
\newline
Lad $P$ være en overgangsmatrice for en Markov-kæde med tilstandsrummet, $S$. En sandsynlighedsdistribution, $\bm\pi=(\pi_1,\pi_2,\dots)$ på $S$, som opfylder
\begin{align*}
    \bm\pi P=\bm\pi
\end{align*}
kaldes for en \textit{stationær distribution} på kæden. 
\end{defn}
\end{minipage}

Indgangene i $\pi$ må dermed opfylde, at
\begin{align*}
    \pi_j=\sum_{i\in S}=p_{ij}\pi_i \text{ for alle } j\in S
\end{align*}
hvilket med betingelsen:
\begin{align*}
    \sum_{i\in S} \pi_i=1
\end{align*}
bestemmer den stationære distribution. 



\chapter{Markov beslutningsprocesser}
%https://onlinelibrary-wiley-com.zorac.aub.aau.dk/doi/pdf/10.1002/9780470316887
Dette kapitel er baseret på \cite[s. 17-25, 78-91 og 143-146]{"Markov_decision_process"}.

I dette kapitel redegøres for de processer, der opfylder markovegenskaben. Disse kaldes for \textit{Markov beslutningsprocesser}. 

\section{Beslutningsmodel}
For at beskrive Markov beslutningsprocesser introduceres begrebet beslutningsmodel. De centrale begreber for en beslutningsmodel er følgende  
\begin{enumerate}
    \item Beslutningstager
    \item Beslutningstidspunkt
    \item Tilstande
    \item Beslutninger
    \item Belønninger eller omkostninger som følge af en beslutning % (vi skriver kun belønninger, omkostninger er implicit)
\end{enumerate}
Til et specifik beslutningstidspunkt observerer en beslutningstager en tilstand i et system. Ud fra denne observation tager beslutningstageren en beslutning. Denne beslutning resulterer i, at beslutningstageren får en belønning eller omkostning, og systemet udvikler sig til en ny tilstand til det efterfølgende beslutningstidspunkt. Ved denne nye tilstand skal beslutningstageren tage en ny beslutning, som fører til en ny tilstand. Denne proces fortsætter, hvoraf beslutningstageren modtager en sekvens af belønninger og omkostninger. For beslutningstageren er målet at optimere belønningerne og minimere omkostningerne.

%Problem -- beslutninger, tilstande -- beslutningstager --taget en beslutning-- reward/cost

\section{Markov beslutningsprocesser}\label{afsnit:Markov_beslutningsprocesser}
En Markov beslutningsproces er en beslutningsmodel, og er dermed baseret på begreberne beslutninger, heriblandt beslutningstager og beslutningstidspunkter, samt tilstande og belønninger. For en Markov beslutningsproces afhænger tilstande til fremtidige beslutningstidspunkter af den nuværende tilstand og de forrige beslutninger. Beslutningstageren skal derfor ikke udelukkende tage beslutninger ud fra kortsigtede belønninger, men også ud fra de fremtidige for at kunne optimere sine belønninger. 


%En beslutningstager kan påvirke et probabilistisk system over tid. Målet er at vælge en sekvens af hændelser / beslutninger, som får systemet til at fungere optimalt i forhold til et vis kriterie. Tilstanden af systemet i morgen afhænger af tilstanden i dag, samt beslutningen i dag. Man bliver dermed nødt til at tænke frem og man skal forudse mulighederne og omkostningerne (eller belønninger) forbundet med fremtidige systemtilstande. 


\subsection{Beslutningstidspunkter}
I \autoref{Kap:MArkov-kæder} er hver tilstand indekseret af et indeks i mængden $T$, som kaldes for indeksmængden. For en Markov beslutningsproces består denne indeksmængde af diskrete beslutningstidspunkter, $t$. Mængden af beslutningstidspunkter er enten endelig eller uendelig. %Der gælder, at $T$ er en endelig eller uendelig mængde af diskrete beslutningstidspunkter $t$.%, hvor hvert element i $T$ noteres $t$.

%$T$ er mængden af beslutningstidspunkter. Beslutningstidspunkterne tager udgangspunkt i diskret tid med tilfældige tidsperioder, hvor forskellige begivenheder optræder. Eksempelvis ved et køsystem eller ved tidspunkter valgt af beslutningstageren. Denne tid er opdelt i perioder eller tilstande. Beslutningstidspunkter er enten endelige, uendelige, eller begrænset i intervaller.

\subsection{Tilstande og beslutninger}
Til ethvert beslutningstidspunkt, $t$, er systemet i en tilstand. Mængden af mulige tilstande i systemet betegnes $S$ og lad $s\in S$ være en tilstand. Tilstandsrummet, $S_t$, er mængden af mulige tilstande i systemet til beslutningstidspunktet, $t$. 

Mængden af mulige beslutninger ved tilstanden, $s$, kaldes beslutningsrummet og noteres $A_s$. Det noteres $A_{s,t}$, hvis beslutningerne også afhænger af $t$. Hver enkelte beslutning betegnes $a$. Beslutninger kan enten tages tilfældigt eller deterministisk. Hvis hver beslutning tages tilfældigt, betyder det, at de tages med en bestemt sandsynlighed. Hvis beslutningerne derimod vælges deterministisk, vil sandsynligheden for én beslutning være 1, mens den for alle andre vil være 0.


%Lad $(\Omega, \F, P)$ være et sandsynlighedsrum. 
Det gælder, at $S_t$ og $A_{s,t}$ er diskrete og tællelige mængder. Det bemærkes, at \\$S=\displaystyle \bigcup_{t\in T}S_t$ og $A_s=\displaystyle\bigcup_{t\in T}A_{s,t}$ og .


%Lad $S_t$ angive mængden af mulige tilstande til tiden $t$, sådan at $S=\bigcup_{t\in T} S_t$ og lad $A_{s,t}$ være mængden af tilladte hændelser i tilstand $s\in S$ og tid $t$. Så er $A_s=\bigcup_{t\in T} A_{s,t}$ og beslutningstageren kan i så fald vælge enhver hændelse, $a\in A_s$. Lad $A=\bigcup_{s\in S} A_s$. Hændelser kan enten vælges tilfældigt eller deterministisk. Lad $P(A_s)$ angive mængden af sandsynligshedsfordelinger for delmængder af $A_s$ og ligeledes, $P(A)$ være sandsynlighedsfordelinger på delmængder af $A$.

\subsection{Belønningssandsynlighed}
Når beslutningstageren har taget en beslutning, $a\in A_{s}$ til beslutningstidspunktet, $t$, vil beslutningstageren modtage en belønning, $r_t(s,a)$. Hvis $r_t(s,a)$ er negativ, vil beslutningen resultere i en omkostning. Hvis belønningen er afhængig af tilstanden til næste beslutningstidspunkt, betegnes belønningen $r_t(s, a, s')$, hvor $s'$ er tilstanden til beslutningstidspunktet, $t+1$.
Når beslutningen er taget, vil tilstanden til næste beslutningstidspunkt blive bestemt ved sandsynlighedsfordelingen, $p_t(\cdot|s,a)$. 


% For at gøre rede for belønningen ved et tilfældigt eksperiment, vil handlingen beskrives ved $a \in A_s$, i et stadie, $s$, til beslutningstidspunkt, $t$.
% Denne belønning beskrives ved $r_t(s,a)$. Tilstanden ved næste beslutningstidspunkt er afhængig af sandsynlighedsfordelingen, $p_t(\cdot|s,a)$.

%Denne belønning kan beskrives ved $r_t(s,a)$, hvoraf den næste beslutning er givet ved sandsynligheden $p_t(\cdot | s,a)$.\\
%Desuden bemærkes det, at en negativrt(s,a)vil belønningen være en udgift. Når belønnin-gen afhænger af systemets tilstand til den næste beslutningsperiode, kan r_t(s,a,j)beskrive
% værdien til tiden t af belønningen, og hvor j beskriver et stadie i beslutningsperioden til t+ 1.

%Desuden bemærkes det, at hvis $r_t(s,a)$ er negativ, vil belønningen være en udgift. 
%Når belønningen afhænger af systemets tilstand ved næste beslutningstidspunkt, beskriver $r_t(s,a,j)$ belønningen til tiden $t$, hvor $j$ er tilstanden til beslutningstidspunktet, $t+1$.

Den forventede værdi til beslutningstidspunktet, $t$, kan beregnes ved 
\begin{align*}
    r_t(s,a)=\sum_{s'\in S} r_t(s,a,s')p_t(s' | s,a),
\end{align*}
%
% Følgende definition beskriver den forventede værdi af belønningen til beslutningstidspunktet, $t$.
% \begin{defn}
%  \textbf{Forventet værdi af belønning}
% \newline
% Lad $s$ være en tilstand i tilstandsrummet $S$ og $a$ være en beslutning i beslutningsrummet $A_s$. Lad $r_t$ være belønningen og $p_t$ være sandsynligheden.
% %Lad $p_t(j | s,a)$ være en ikke-negativ funktion
% Den forventede værdi af belønningen til beslutningstidspunktet $t$ er givet ved
% \begin{align*}
%     r_t(s,a)=\sum_{s'\in S} r_t(s,a,s')p_t(s' | s,a),
% \end{align*}
% hvor $s' \in S$ er tilstanden til beslutningstidspunktet $t+1$.
%
% \end{defn}
%
%Funktionen $p_t(j |s,a)$ beskriver sandsynligheden for, at systemet er i tilstand $j \in S$ til tiden $t+1$, når valget, $a \in A$, er givet. 
%p: S \to [0,1]
%
Funktionen, $p_t(s'|s,a)$, kaldes overgangssandsynlighedsfunktionen, hvor $\displaystyle\sum_{s'\in S} p_t(s'|s,a)=1$.


For en Markov beslutningsproces med en endelig indeksmængde, $T$, gælder det, at der ikke tages en beslutning til det sidste beslutningstidspunkt, $N$. Derfor er belønningen til beslutningstidspunktet, $N$, kun en funktion af tilstanden og betegnes $r_N(s)$.

Ud fra ovenstående siges følgende mængde at være en Markov beslutningsproces
%
\begin{align*}
    \left(T, S, A_s, p_t(\cdot | s,a), r_t(s,a)\right).
 \end{align*}
 %
Dette er en Markov beslutningsproces, da overgangssandsynligheden og belønningen kun afhænger af det forrige beslutningstidspunkt. 



%Hvis Markov beslutningsprocessen har en endelig horisont, er der ikke givet nogen beslutning ved beslutningstiden, $N$. Til denne tid er belønningen, $r_N(s)$, udelukkende en funktion af tilstanden og kaldes for \textit{nyværdien}.
% %En Markov beslutningsproces er en mængde af objekter, givet ved:
% \begin{align*}
%     \{T, S, A_s, p_t(\cdot | s,a), r_t(s,a)\}.
% \end{align*}

% Det skyldes nemlig, at overgangssandsynligheden og belønningsfunktionen afhænger af den forrige tilstand og beslutningen taget i denne tilstand.



%decision rule?
\subsection{Beslutningsregel}\label{afsnit:beslutningsregel}
En beslutningsregel er en procedure til at vælge beslutninger for hver tilstand til et givet beslutningstidspunkt, $t$. Beslutningsregler kan både være deterministiske og tilfældige.% men i dette projekt tages der udgangspunkt i deterministiske beslutningsregler.

En deterministisk beslutningsregel er en funktion, $d_t:S\to A_s$, som angiver beslutningsvalget ved tilstanden, $s$, og beslutningstidspunkt, $t$. Beslutningsreglen siges at være Markov, hvis den kun afhænger af de forrige tilstande og beslutninger gennem den nuværende tilstand af systemet. 

En deterministisk beslutningsregel kan være fortidsafhængig, hvis den afhænger af en historik. En historik er en ordnet mængde af skiftevis tilstande og beslutninger, $h_t = (s_1, a_1, \dots,  s_{t-1}, a_{t-1}, s_t)$, som følger rekursionen, $h_t=(h_{t-1}, a_{t-1}, s_t)$. En deterministisk beslutningsregel er fortidsafhængig, hvis den afhænger af den forrige historik, $h_t$.
Lad $H_t$ angive mængden af alle historikker, $h_t$. Bemærk, at 
\begin{align*}
    H_1&=S, \quad H_2=S\times A\times S, \quad H_n =S\times A\times S\times \cdots \times S \text{ og }\\
    H_t&=H_{t-1}\times A\times S. 
\end{align*}
En fortidsafhængig deterministisk beslutningsregel er en funktion, $d_t: H_t\to A$, under betingelsen, at $d_t(h_t)\in A_{s_t}$.

En tilfældig beslutningsregel, $d_t$, er en funktion, $d_t: S\to P(A)$, som angiver sandsynligheden for en beslutning ved tilstanden, $s$, til beslutningstidspunktet, $t$. Altså angiver en tilfældig beslutningsregel sandsynlighedsfordelingen, $q_{d_t}(\cdot)$, for mængden af beslutninger til beslutningstidspunktet, $t$. For en fortidsafhængig tilfældig beslutningsregel, gælder det, at $d_t:H_t\to P(A)$.

Når den tilfældige beslutningsregel er Markov, er $q_{d_t(s_t)}\in P(A_{s_t})$. Hvis den tilfældige beslutningsregel derimod er fortidsafhængig, vil $q_{d_t(h_t)}\in P(A_{s_t})$ for alle $h_t\in H_t$.

En deterministisk beslutningsregel er et specialtilfælde af en tilfældig beslutningsregel, hvor 
\begin{align*}
    q_{d_t(s_t)}(a)=1,\\
    q_{d_t(h_t)}(a)=1
\end{align*}
for ét $a\in A_s$. Mængden af alle beslutningsregler betegnes ved $D_t^K$, hvor $K$ er klassen af beslutningsregler. Der er fire forskellige klasser, som er gennemgået ovenfor Markov deterministisk (MD), fortidsafhængig deterministisk (FD), Markov tilfældig (MT) og fortidsafhængig tilfældig (FT). 

\subsection{Strategi}
En strategi er en sekvens af beslutningsregler $\Psi=(d_1,d_2,\dots,d_{N-1})$, hvor $d_t\in D_t^K$ for alle $t=1, 2, \dots N-1$ for $N \leq \infty$. Altså indeholder en strategi den beslutningsregel, der skal anvendes til ethvert beslutningstidspunkt, og dermed ved beslutningstageren, hvilken beslutning der skal vælges til enhver tilstand. En strategi er invariant, hvis $d_t=d$ for alle $t\in T$. Dette har formen, $\Psi=(d,d,\dots)$ og betegnes ved $d^\infty$. 
Mængden af alle strategier for $K=MD, FD, MT, FT$ betegnes $\Pi^K=D_1^K\times D_2^K\times \cdots\times D_{N-1}^K$.

% En \textit{strategi} bestemmer hvilke beslutningsregler, der skal benyttes til ethvert beslutningstidspunkt. Med andre ord er det en sekvens/følge af beslutningsregler, $\Psi=(d_1,d_2,\dots,d_{N-1})$, hvor $d_t\in D_t^k$.
% Lad nu $\Pi=D_1\times D_2,\times\cdots\times D_{n-1}, N\leq \infty$ være en endelig eller uendelig mængde af alle strategier.

% En strategi er invariant, hvis $d_t=d$ for alle $t\in T$. Dette har formen, $\pi=(d,d,\dots)$ og betegnes ved $d^\infty$. 

\section{Induceret stokastisk proces, betingede sandsynligheder og forventede værdier og et andet navn}

% Symbolerne for sandsynlighedsrum, hændelsesrum og lignende anvendes uden forklaring i følgende afsnit.

%Symbolerne som er præsenteret i underafsnit \ref{afsnit:Markov_beslutningsprocesser} anvendes uden forklaring i det resterende af dette kapitel, og derfor præsenteres de ikke i resultaterne i de følgende underafsnit. 

Af \autoref{def:sandsynlighedsrum} gælder det, at et sandsynlighedsrum er givet ved triplen, $(\Omega, \F, P)$, hvor $\Omega$ betegner udfaldsrummet, $\F$ hændelsesrummet og $P$ et sandsynlighedsmål på $(\Omega,\mathcal{F})$. I dette afsnit er $\F$ potensmængden af $\Omega$.

Udfaldsrummet er givet ved
\begin{align*}
    \Omega = S \times A \times S \times \cdots \times S = \{S \times A\}^{N-1} \times S
\end{align*}
for en endelig indeksmængde $T=\{1, 2,\cdots, N\}$, hvor $N<\infty$, og
\begin{align*}
    \Omega = \{S \times A\}^\infty
\end{align*}
for en uendelig indeksmængde, hvor $N=\infty$.

% For en endelig indeksmængde, $T=\{1, 2,\cdots, N\}$, hvor $N<\infty$, så er udfaldsrummet givet ved gælder det, at
% \begin{align*}
%     \Omega = S \times A \times S \times \cdots \times S = \{S \times A\}^{N-1} \times S,
% \end{align*}
% og for en uendelig indeksmængde, hvor $N=\infty$, er 
% \begin{align*}
%     \Omega = \{S \times A\}^\infty.
% \end{align*}
Hvert element $\omega \in \Omega$ kaldes en \textit{udfaldsvej} og består af en sekvens af tilstande og beslutninger. Altså
\begin{align*}
    \omega=(s_1 , a_1 , s_2 , a_2 , \dots , a_{N-1} , s_N).
\end{align*}
Hvis $N<\infty$, er indeksmængden endelig, og hvis $N=\infty$, er indeksmængden uendelig. 
Lad $X_t$ og $Y_t$ være to tilfældige variabler, som er givet ved
\begin{align*}
    X_t(\omega) = s_t \text{ og } Y_t(\omega) = a_t \text{ for } t=1,2,\dots, N, \  N\leq \infty,
\end{align*}
hvor $X_t$ antager værdier i tilstandsrummet, $S$, og $Y_t$ antager værdier i beslutningsrummet, $A$. Altså gælder det, at $X_t$ betegner tilstanden til $t$ og $Y_t$ betegner beslutningen til $t$ for en udfaldsvej, $\omega$.

Lad derudover $Z_t$ være en tilfældig variabel, som er givet ved
\begin{align*}
    Z_1(\omega) = s_1 \text{ og } Z_t(\omega) = (s_1 , a_1 , s_2 , a_2 , \dots , s_t) \text{ for } t=1,2, \dots, N, \  N\leq \infty.
\end{align*}

En fortidsafhængig strategi $\Psi = (d_1, d_2, \dots, d_{N-1})$, hvor $N \leq \infty$, inducerer en sandsynlighed, $P^\Psi$, på $(\Omega, \F)$ ved
\begin{align}
    P^{\Psi}(X_1=s)&=P_1(s)\label{eqs:de_tre_ligninger},\\
    P^{\Psi}\left(Y_t=a|Z_t=h_t\right)&= q_{d_t(h_t)}(a)\label{eqs:de_tre_ligninger2},\\
    P^{\Psi}\big(X_{t+1}=s|Z_t&=(h_{t-1}, a_{t-1}, s_t), Y_t=a_t \big)=p_t(s|s_t, a_t), \label{eqs:de_tre_ligninger3}
\end{align}
hvor $P_1(s)$ betegner begyndelsesfordelingen til tilstanden $s$.

Af kædereglen for sandsynligheder, (se \autoref{bilag:kædereglen}), følger det, at sandsynligheden for en udfaldsvej, $\omega =(s_1 , a_1 , s_2 , \dots , a_{N-1} , s_N)$, er givet ved
\begin{align}
    P^\Psi(s_1 , a_1 , s_2 , \dots , s_N) &= P(s_1)P(a_1|s_1)P(s_2|a_1 , s_1)P(a_2|s_2 , a_1 , s_1)\nonumber \\
    &\phantom{= \ }\cdots P(a_{N-1}|s_{N-1} , \dots , a_1 , s_1) P(s_N | a_{N-1} , \dots , a_1 , s_1).\nonumber
    % \intertext{Af Markov egenskaben, \autoref{markov-kæde-hovedsætning}, gælder det, at}
    % P^\Psi(s_1 , a_1 , s_2 , \dots , s_N) &= P(s_1)P(a_1|s_1)P(s_2|a_1 , s_1)P(a_2|s_2 , a_1 , s_1)\nonumber \\
    % &\phantom{= \ }\cdots P(a_{N-1}|s_{N-1} , a_{N-2}) P(s_N | a_{N-1} , s_{N-1}).\nonumber
    \intertext{Ved at anvende \eqref{eqs:de_tre_ligninger2} og \eqref{eqs:de_tre_ligninger3} fås}
    P^\Psi(s_1 , a_1 , s_2 , \dots , s_N) &= P(s_1)q_{d_1(s_1)}(a_1)p_1(s_2|s_1 , a_1)q_{d_2(h_2)}(a_2)\nonumber\\
     &\phantom{= \ }\cdots q_{d_{N-1}(h_{N-1})}(a_{N-1})p_{N-1}(s_N|a_{N-1} , s_{N-1}).\label{eq:den_meget lange_der_er_træls_at_skrive}
\end{align}
Fra \autoref{afsnit:beslutningsregel} vides det, at når en strategi, $\Psi$, er deterministisk, eksisterer der et $a\in A_s$ således, at $q_{d_t(s_t)}(a) = 1$ og $q_{d_t(h_t)}(a)=1$. Dermed gælder det for en deterministisk strategi, $\Psi$, at
\begin{align*}
    P^\Psi(s_1 , a_1 , s_2 , \dots , s_N) &= P(s_1)p_1(s_2|s_1 , a_1)\cdots p_{N-1}(s_N|a_{N-1} , s_{N-1}).
\end{align*}
De ovenstående ligninger skal anvendes til at bestemme udtrykket for følgende betingede sandsynlighed 
\begin{align}
    P^\Psi(a_t , s_{t+1}, \dots, s_N | s_1 , a_1 , \dots , s_t) = \frac{P^\Psi(s_1 , a_1 , \dots , s_N)}{P^\Psi(s_1 , a_1 , \dots , s_t)}, \label{eq:betinget_sands_for_meget_lang_udtryk}
\end{align}
som følger af kædereglen for sandsynligheder, (se \autoref{bilag:kædereglen}), for $P^\Psi(s_1 , a_1 , \dots , s_t) \neq 0$. Bemærk, at \\
$ P^\Psi(a_t , s_{t+1}, \dots, s_N | s_1 , a_1 , \dots , s_t) = 0 $, hvis $P^\Psi(s_1 , a_1 , \dots , s_t) = 0$.

% under antagelsen af, at $P^\Psi(s_1 , a_1 , \dots , s_t) \neq 0$.

Ved at sætte $N=t$ i \eqref{eq:den_meget lange_der_er_træls_at_skrive} vides det, at 
\begin{align*}
    P^\Psi(s_1 , a_1 , s_2 , \dots , s_t) &= P(s_1)q_{d_1(s_1)}(a_1)p_1(s_2|s_1 , a_1)q_{d_2(h_2)}(a_2)\\
     &\phantom{= \ }\cdots q_{d_{t-1}(h_{t-1})}(a_{t-1})p_{t-1}(s_t|a_{t-1} , s_{t-1}).
\end{align*}

Udtrykkene i brøken for \eqref{eq:betinget_sands_for_meget_lang_udtryk} indsættes og reduceres
\begin{align*}
P^\Psi(a_t , s_{t+1}, \dots, s_N | s_1 , a_1 , \dots , s_t) &=\frac{P(s_1)q_{d_1(s_1)}(a_1)
    \cdots p_{N-1}(s_N|a_{N-1} , s_{N-1})}{P(s_1)q_{d_1(s_1)}(a_1)
    \cdots p_{t-1}(s_t|a_{t-1} , s_{t-1})}\\
    &= q_{d_t(h_t)}(a_t)p(s_{t+1}|s_t , a_t)\\
    &\phantom{= \ } \cdots q_{d_{N-1}(h_{N-1})}(a_{N-1})p_{N-1}(s_N|s_{N-1} , a_{N-1}).
\end{align*}
Hvis en strategi, $\Psi$, er Markov, så afhænger beslutningsreglen, $d_t$, kun af de forrige tilstande og beslutninger gennem den nuværende tilstand. Det gælder altså, at 
\begin{align*}
    q_{d_t(h_t)}(a)= P^\Psi \left(Y_t = a | Z_t = (h_{t-1} , a_{t-1} , s_t) \right) = P \left(Y_t = a | X_t = s_t\right) = q_{d_t(s_t)}(a).
\end{align*}
Dermed følger det, at
\begin{align*}
    P^{\Psi}\left(a_t , s_{t+1} , \cdots , s_N|s_1 , a_1 , \cdots , s_t\right)=P^{\Psi}\left(a_t , s_{t+1} , \cdots , s_N|s_t\right).
\end{align*}

\subsection{Markov belønningsproces}
En Markov belønningsproces er den bivariate stokastiske proces $\left\{\left(X_t, r_t(X_t, Y_t)\right); t \in T\right\}$, når en strategi, $\Psi$, er Markov. Denne stokastiske proces består af en sekvens af tilstande og belønninger givet en strategi, $\Psi$. 

Lad $W$ være en diskret tilfældig variabel defineret på sandsynlighedsrummet, $(\Omega, \F, P^{\Psi})$, hvor
\begin{align*}
    W(\omega)=W(s_1 , a_1 , \dots , s_N) = \sum_{t=1}^{N-1} r_t(s_t,a_t) + r_N(s_N).
\end{align*}
Dermed er $W(\omega)$ summen af belønninger. Af \autoref{def:Forventetværdi} følger det, at
\begin{align*}
    E^\Psi [W] = \sum_{\omega \in \Omega} W(\omega)P^\Psi\left(\omega\right) = \sum_{w \in Range(W)} w P^\Psi \left(w: W(\omega) = w\right), 
\end{align*}
hvor $E^\Psi(W)$ betegner den forventede værdi til $W$ givet strategien $\Psi$. Hvis Omega ikke er tællelig, bliver summerne i ovenstående til integraler. 
 %Summerne i ovenstående bliver til integraler for $N = \infty$ og $E^\Psi(W)$ betegner den forventede værdi til $W$ givet strategien $\Psi$.

Hvis strategien $\Psi$ er fortidsafhængig med historikken $h_t=(s_1, a_1 , \dots , s_t)$, og $W$ er en funktion af $s_t, a_t,\dots, s_N$, så gælder det, at
\begin{align}\label{eq:forventet_belønning_fortidsafhængig}
    E_{h_t}^\Psi \left[W(X_t , Y_t , \dots , X_N) \right] = \sum W(s_t , a_t , \dots , s_N) P^\Psi (a_t , s_{t+1} , \dots , s_N | s_1 , a_1 , \dots , s_t),
\end{align}
hvor der summes over $(a_t , s_{t+1} , \dots , s_N) \in A \times S \times \cdots \times S$. Er strategien, $\Psi$, derimod Markov, følger det, at
\begin{align*}
     E_{s_t}^\Psi \left[W(X_t , Y_t , \dots , X_N) \right] = \sum W(s_t , a_t , \dots , s_N) P^\Psi (a_t , s_{t+1} , \dots , s_N | s_t).
\end{align*}

%For en beslutningstager gælder det om at optimere denne sum, så der opnås størst mulig belønning.   
% \subsection{Markov modellen}

% I dette afsnit konstrueres en model for den stokastiske model genereret ud fra Markov beslutningsprocessen. Det antages, at $S$ og $A$ er diskrete.

% Som nævnt i sandsynlighedsafsnittet, består et sandsynlighedsrum af tre elementer: et udfaldsrum, $\Omega$, et hændelsesrum, $\mathcal{F}$ og et sandsynlighedsmål, $P$, på $(\Omega,\mathcal{F})$.
% Bemærk, at når $\Omega$ er endelig, så er $\mathcal{F}$ givet ved potensmængden af $\Omega$ og sandsynlighedsmålet er en sandsynlighedsmassefunktion. 

% For en MDP med endelig horisont, vælges
% \begin{align*}
%     \Omega=S\times A\times S\times A\times\cdots\times A\times S=\{S\times A\}^{N-1}\times S,
% \end{align*}

% mens en uendelig horisont indebærer, at $\Omega=\{S\times A\}^\infty$. \\\\
% En \textit{vej} $\omega\in\Omega$ er en sekvens af tilstande og hændelser:
% \begin{align*}
%     \omega=(s_1,a_1,s_2,a_2,\dots,A_{N-1},S_N)
% \end{align*}

% Lad nu $X_t$ være en diskret tilfældig variabel, der antager tilstande, $s_t$ og lad ligeledes $Y_t$ antage hændelser, $a_t$. Dermed er 
% \begin{align*}
%     X_t(\omega)=s_t \ \text{ og } \ Y_t(\omega)=a_t,
% \end{align*}
% for $t\in T$. 

% En historik proces defineres til
% \begin{align*}
%     Z_t(\omega)=s_1 \ \text{ og } \ Z_t(\omega)=(s_1,a_2,\dots,s_t) \text{ for } 1\leq t\leq N;\ N\leq \infty
% \end{align*}
% Lad $P_1(\cdot)$ betegne \textit{begyndelsesdistributionen} for systemets tilstand.



% \subsection{Forventet værdi}
% Når $\pi$ er Markov, siges den bivariate stokastiske proces, $\{(X_t,r_t(X_t,Y_t)); t\in T\}$, at være en \textit{Markov belønningsproces}. 

% Lad $W$ være en tilfældig variable på sandsynlighedsrummet, $\{\omega, \mathcal{F},P^\pi\}$ og lad $n$ være endelig. 
% Den forventede værdi er givet ved
% \begin{align*}
%     E^\pi [W]=\sum_{\omega\in\Omega} W(\omega)P^\pi\{\omega\}
% \end{align*}


% Lad nu $W$ være en funktion af en historik $h_t$. Hvis $\pi$ er markov, så gælder der, at
% \begin{align*}
%     E_{s_t}^\pi [W(X_t,Y_t,\dots, X_n]=\sum W(s_t,a_t,\dots, s_N)P^\pi (a_t,s_{t+1},\dots,s_N|s_t).
% \end{align*}

% Bemærk, at 
% \begin{align*}
%     E^\pi[W]=\sum_{s\in S} P_1(s)E_s^\pi [W] \forall\pi\in\Pi
% \end{align*}


%I foregående afsnit blev der gennemgået, hvordan beslutningstageren til hvert beslutningstidspunkt skal tage en beslutning. For at optimere belønningen skal beslutningstageren vælge den bedste strategi, $\Psi$. Lad $\Psi=(d_1, d_2\dots d_{N-1})$ være en fortidsafhængig strategi, hvor det derfor gælder, at $d_t: H_t\to A_t$. Lad derudover $H_t^{\Psi}$ være den korresponderende mængde af historikker til den valgte strategi og historik udspillet. 

Beslutningstageren er interesseret i at optimere sin belønning for indeksmængden, $T$, hvor beslutningstageren skal tage en beslutning til hvert beslutningstidspunkt. For at optimere belønningen skal beslutningstageren derfor vælge den bedste strategi, $\Psi=(d_1, d_2,\dots, d_{N-1})$.

Lad $v_N^\Psi(s)$ betegne den forventede totale belønning for en valgt strategi, $\Psi$, i tilstanden $s$ til første beslutningstidspunkt. Såfremt strategien er fortidsafhængig tilfældig, $\Psi\in \Pi^{FT}$, er $v_N^\Psi(s)$ givet ved
\begin{align*}
    v_N^{\Psi}(s)=E_{(s)}^{\Psi}\left[\sum_{t=1}^{N-1}r_t(X_t, Y_t)+r_N(X_N)\right].
\end{align*}
Er strategien derimod fortidsafhængig deterministisk, $\Psi\in\Pi^{FD}$, er $Y_t = d_t(h_t)$ til hvert beslutningstidspunkt. Under antagelsen om, at $|r_t(s,a)| < \infty$ for $(s,a) \in S \times A$ og $t \leq N$, kan $v_N^\Psi$ eksistere. 

%Det gælder, at $v_N^\Psi$ eksisterer under antagelsen om, at $|r_t(s,a)| \leq M < \infty$ for $(s,a) \in S \times A$ og $t \leq N$.


%Lad $v_N^{\Psi}(s)= E_{s}^\Psi \left[W(X_t , Y_t , \dots , X_N) \right]$ angive den forventede totale belønning, når $\Psi$ er valgt og $s$ er tilstanden til første beslutningstidspunkt, $X_1=s$. 
%Den forventede totale belønning er givet som følgende
% \begin{align*}
%     v_N^{\Psi}(s)=E_{\Psi, s}\left\{\sum_{t=1}^N r_t\left(X_t^{\Psi}, d_t(H_t^{\Psi})\right)+r_{N+1}^{\Psi}(X_{N+1}^{\Psi})\right\},
% \end{align*}
% hvor $E_{\Psi, s}$ betegner forventningen med hensyn til den fælles sandsynlighedsfordeling af den stokastiske proces bestemt af strategien $\Psi$ betinget af tilstanden, $s$.

Beslutningstageren skal vælge den strategi, $\Psi^*$, der giver den største forventede totale belønning. Denne strategi siges at være den \textit{optimale strategi}, hvis den opfylder, at %For den valgte strategi skal følgende være opfyldt
\begin{align*}
    v_N^{\Psi^*}(s)\geq v_N^{\Psi}(s), \text{ for } s\in S \text{ og for alle } \Psi\in \Pi^{FT}.
\end{align*}
%for at den er den \textit{optimale strategi}. 


%Hvis beslutningsrummet er åben findes der tilfælde hvor man ønsker at komme vilkårligt tæt på en værdi for at maksimere belønningen.

Hvis den optimale strategi ikke eksisterer, bestemmes en $\varepsilon$-optimal strategi. Dette er en strategi $\Psi^*_\varepsilon$, hvor der for et $\varepsilon>0$, er opfyldt, at 
\begin{align*}
    v_N^{\Psi^*_\varepsilon}(s)+\varepsilon>v_N^\Psi(s) \text{ for } s\in S \text{ og for alle } \Psi \in \Pi^{FT}.
\end{align*}

Lad $v_N^*$ betegne den maksimale forventede totale belønning. Så gælder det, at
%
\begin{align}\label{eq:v_N=sup}
    v_N^*(s)=\sup_{\Psi\in \Pi^{FT}} v_N^{\Psi}(s)=v_N^{\Psi^*}(s) \text{ for }  s\in S.
\end{align}
Når både tilstandsrummet, $S$, og beslutningsrummet, $A$ er endeligt, kan beslutningstageren kun vælge mellem et endeligt antal strategier. I dette tilfælde er $v_N^*$ givet ved
\begin{align*}
    v_N^*(s)=\max_{\Psi\in \Pi^{FT}} v_N^{\Psi}(s)=v_N^{\Psi^*}(s) \text{ for }  s\in S,
\end{align*}
hvor $v_N^*(s)$ kaldes for den \textit{optimale belønningsfunktion}.

% Derfor er det muligt at vælge den optimale strategi, $\Psi^*$, der opfylder følgende
% \begin{align}\label{eq:optimale_belønningsfunktion}
%     v_N^{\Psi^*}(s)=\max_{\Psi\in \Pi} v_N^{\Psi}(s)\equiv v_N^*(s) \text{ for alle }  s\in S_1,
% \end{align}
% hvor $v_N^*(s)$ kaldes for den \textit{optimale belønningsfunktion}.

%$E^\Psi_{(s)} [....]$


% \textbf{Ved ikke om dette skal med}\\
% I tilfælde hvor \eqref{eq:optimale_belønningsfunktion} ikke eksisterer, bestemmes den mindste øvre grænse. Altså bestemmes $ v_N^{\Psi^*}(s)=\displaystyle\sup_{\Psi\in\prod} v_N^{\Psi}(s)$ for alle $s\in S_1$.

% Hvis dette er tilfældet, skal beslutningstageren vælge en strategi der er $\epsilon$-optimal. Dermed skal beslutningstageren vælge en strategi, der for alle $\varepsilon>0$ opfylder, at 
% \begin{align*}
%     v_N^{\Psi^*_\varepsilon}+\varepsilon>v_N^*(s) \text{ for alle } s\in S_1.
% \end{align*}
% En sådan strategi eksisterer ud fra definitionen af den mindste øvre grænse.



% ________________________________________
% For at optimere beløningsfunktionen, $v_N^*(s)$, introduceres \textit{Bellman ligninger}. Før det er muligt at introducere Bellman ligninger, defineres den forventede værdi til hvert beslutningstidspunkt, $t$, som følgende
% \begin{align*}
%     u_t^{\Psi}(h_t)=E_{\Psi, h_t}\left\{\sum_{n=t}^{N}r_n\left(X_n^{\Psi},d_n\left(H_n^{\Psi}\right)\right)\right\}.
% \end{align*}
% Dermed er den forventede belønning til ethvert beslutningstidspunkt for $t,\ t+1, \dots, N$ bestemt ud fra en strategi, $\Psi$, der er betinget af historikken op til beslutningstidspunktet. Den forventede værdi til hvert beslutningstidspunkt kaldes også for den \textit{ beslutning-belønningsfunktion}.
% Altså gælder det, at $v_N^{\Psi}$ er defineret ud fra alle fremtidige beslutninger og tilstande, mens $u_t^{\Psi}$ er defineret ud fra en del af de fremtidige beslutninger begyndende ved $t$.

% Den \textit{optimale beslutning-belønningsfunktion} er givet som følgende
% \begin{align*}
%     u_t^{\ast}(h_t)=\max_{\Psi\in\prod}u_t^{\Psi}(h_t).
% \end{align*}

% Herefter er det muligt at præsentere Bellman ligninger, der er givet som følgende
% \begin{align}\label{eq:bellman_equation}
%     u_t(h_t)=\max_{a\in A_{s_t, t}}\left\{r_t(s_t, a)+\sum_{j\in S_{t+1}}p_t(j|s_t, a)u_{t+1}(h_t, a, j)\right\},
% \end{align}
% hvor $t=1, 2, \dots, N$ og $h_t\in H_t$. 
% En løsning til systemet af ligninger, \eqref{eq:bellman_equation}, er en sekvens af funktioner, $u_t: H_t\to A_t$ for $t=1, \dots, N$.
% For Bellman ligninger gælder følgende
% \begin{enumerate}
%     \item Løsningerne til ligningerne er de optimale belønninger fra beslutningstidspunktet $t$ til $N$ for ethvert $t$.
%     \item De kan afgøre om en strategi er optimal.
%     \item De bestemmer en effektiv metode til at finde de optimale belønningsfunktioner og strategier.
%     \item De kan bestemme egenskaber for strategier og belønningsfunktioner.
% \end{enumerate}

% Følgende sætning beskriver disse egenskaber for Bellman ligninger\\

% \begin{minipage}\textwidth
% \begin{thmx} \textbf{Egenskaber for Bellman ligninger}\label{sæt:egenskaber_for_bellman} %Ny sætning
% \newline
% Lad $h_t\in H_t$ være en historik, $s\in S$ være tilstande og $v_N^{\ast}$ være den optimale beløningsfunktion. Lad derudover $u_t$ være den optimale beslutning-belønningsfunktion og en løsning til \eqref{eq:bellman_equation} for $t=1, \dots, N$. Så gælder det, at
% \begin{enumerate}
%     \item $u_t(h_t)=u_t^{\ast}$ for alle $h_t\in H_t$, hvor $t=1, \cdots, N$.\\
%     \item $u_1(s_1)=v_N^\ast(s_1)$ for alle $s_1\in S_1$.
% \end{enumerate}
% \end{thmx}
% \end{minipage}

% Fra \autoref{sæt:egenskaber_for_bellman} punkt 1. gælder det, at alle løsninger til Bellman ligninger er de optimale belønningsfunktioner fra beslutningstidspunktet $t$ til $N$ for ethvert $t$. Fra punkt 2. gælder det, at en løsning til den første Bellman ligning er belønningsfunktionen for Markov beslutningsprocessen. 

% Hvis resultatet af \eqref{eq:bellman_equation} er kendt, og maksimum dermed er bestemt, kan Bellman ligningerne anvendes til at bestemme optimale strategier.

% \begin{minipage}\textwidth
% \begin{thmx} \textbf{}\label{sæt:optimal_strategi_ved_bellman} %Ny sætning
% \newline
% Lad $h_t\in H_t$ være en historik, $s\in S$ være tilstande, $a\in A_s$ være beslutninger og $\Psi^{\ast}=d_1^{\ast}, d_2^{\ast},\dots,d_{N-1}^{\ast}$ være en strategi. Lad derudover $u_t^{\ast}$ være den optimale forventede værdi til ethvert belønningstidspunkt og en løsning til \eqref{eq:bellman_equation} for $t=1, \dots, N$. Lad strategien være defineret som
% \begin{align}\label{eq:optimale_strategi_ved_bellman}
%     r_t\left(s_t, d_t^{\ast}(h_t)\right)+\sum_{j\in S_{t+1}}p_{t+1}\left(j|s_t, d_t^{\ast}(h_t)\right)u_{t+1}^{\ast}\left(h_t, d_t^{\ast}(h_t), j\right)\nonumber\\
%     =\max_{a\in A_{s_t,t}}\left\{r_t(s_t, a)+\sum_{j\in S_{t+1}}p_t(j,s_t, a)u_{t+1}^{\ast}(h_t, a, j)\right\}
% \end{align}
% Så gælder det, at
% \begin{enumerate}
%     \item $\Psi^{\ast}$ er en optimal strategi og $v_N^{\Psi^{\ast}}(s)=v_n^{\ast}(s)$ for alle $s\in S_1$.\\
%     \item $u_t^{\Psi^{\ast}}(h_t)=u_t^{\ast}(h_t)$ for $h_t\in H_t$ og for alle $t=1, 2,\dots, N$.
% \end{enumerate}
% \end{thmx}
% \end{minipage}

% Fra \autoref{sæt:optimal_strategi_ved_bellman} gælder det dermed, at den optimale strategi bestemmes ved først at løse Bellman ligningerne og dernæst vælge en beslutningsregel til enhver historik, der sikrer den beslutning, der medfører det maksimale i \eqref{eq:optimale_strategi_ved_bellman}.

% Punkt 2 i \autoref{sæt:optimal_strategi_ved_bellman} kaldes også for \textit{Princippet for optimalitet/optimalitets-princippet}.

% Den optimale strategi, $\Psi^{\ast}$, blev defineret ved $\eqref{eq:optimale_strategi_ved_bellman}$, som også kan udtrykkes som følgende
% \begin{align*}
%     d_t^{\ast}(h_t)=\argmax_{a\in A_{s_t, t}}\left\{r_t(s_t, a)+\sum_{j\in S_{t+1}}p_t(j,s_t, a)u_{t+1}^{\ast}(h_t, a, j)\right\},
% \end{align*}
% hvor arg max resulterer i en mængde, mens max resulterer i en reel værdi.




\section{Optimering af belønning}\label{afsnit:optimering_af_belønning}
I dette afsnit introduceres en måde, hvorpå den maksimale forventede totale belønning kan bestemmes. I følgende afsnit er indeksmængden endelig. 
%Følgende gælder for en endelig indeksmængde. 

Lad en strategi, $\Psi = (d_1,d_2, \ldots, d_{N-1})$, være fortidsafhængig tilfældig og $u_t^\Psi: H_t \to \R$ være den forventede totale belønning for $\Psi$ til beslutningstidspunkterne $t, t+1, \ldots, N-1$. For historikken, $h_t \in H_t$, er $u_t^\Psi$ givet ved 
% 
\begin{align}\label{eq:forventede_totale_belønning} 
    u_t^{\Psi}(h_t)=E^{\Psi}_{h_t}\left[\sum^{N-1}_{n=t}r_n(X_n, Y_n)+r_N(X_N)\right] \text{ for } t < N. 
\end{align}
For $t=N$, så er $h_N=(h_{N-1},a_{N-1},s)$ og
\begin{align}\label{eq:u_N=r_N}
    u_N^{\Psi}(h_N)=r_N(s). 
\end{align}
Forskellen mellem $v_N^\Psi$ og $u_t^\Psi$ er, at $v_N^\Psi$ betegner den forventede totale belønning for alle beslutningstidspunkter, mens $u_t^\Psi$ kun defineres ud fra de fremtidige beslutninger begyndende ved $t$. Bemærk, at $u_1^\Psi(s) = v_N^\Psi(s)$ for $h_1 = s$. 

%Altså gælder det, at $v_N^{\Psi}$ er defineret ud fra alle fremtidige beslutninger og tilstande, mens $u_t^{\Psi}$ er defineret ud fra en del af de fremtidige beslutninger begyndende ved $t$.
%Dog er $u_1^{\Psi}(s)=v_N^{\Psi}(s)$, hvis det for alle $s\in S$ er opfyldt, at $h_1=s$.

Ved at bestemme $u_t^{\Psi}$ induktivt er det derved muligt at bestemme $v_N^\Psi$. Altså anvendes baglæns induktion til at bestemme den forventede totale belønning. 
Lad $\Psi\in \Pi^{FT}$ og lad $S$ være et diskret tilstandsrum. Så er det muligt at bestemme $u_t^\Psi$ med følgende evalueringsalgoritme for en endelig strategi. %\textit{endelige strategi evalueringsalgoritme}, som er givet ved følgende

\begin{alg} \textbf{Evalueringsalgoritme for en endelig strategi} \label{Algoritme1}%Ny algoritme
\begin{enumerate}
    \item Sæt $t = N $ og $u_N^\Psi(h_N) = r_N(s_N)$ for alle $h_N = (h_{N-1}, a_{N-1}, s_N) \in H_N$.
    \item Hvis $t = 1$, stop, ellers gå til trin 3.
    \item Udskift $t$ med $t-1$ og beregn $u_t^\Psi(h_t)$ for hvert $h_t = (h_{t-1}, a_{t-1}, s_t)\in H_t$ ved \begin{align}\label{eq:ligning_i_algoritme1}
        \vspace{-0.5cm}u_t^\Psi(h_t) = \sum_{a \in A_{s_t}}q_{d_t(h_t)}(a)\left(r_t(s_t, a) + \sum_{s' \in S} p_t\left(s'|s_t,a\right)u_{t+1}^\Psi(h_t,a,s')\right),
    \end{align}
    hvor $(h_t, a ,s') \in H_{t+1}$.
    \item Gå til trin 2.
\end{enumerate}
\end{alg}

Givet historikken $h_t$ kan \eqref{eq:ligning_i_algoritme1} bruges til at evaluere en givet strategi. Altså er den forventede totale belønning lig belønningen modtaget ved at tage beslutningen $a$ adderet med den forventede belønning for de resterende beslutningstidspunkter, $t+1, \ldots, N$. Alt dette multipliceres med sandsynlighedsfordelingen, $q_{d_t(h_t)}$, som er sandsynligheden for at tage beslutningen $a$ givet $h_t$. 

Summen over $s' \in S$ i \eqref{eq:ligning_i_algoritme1} indeholder produktet mellem sandsynligheden for at være i tilstanden $s'$ til beslutningstidspunktet $t+1$, hvis beslutningen $a$ er valgt, og den forventede totale belønning ved at bruge strategien, $\Psi$, for perioderne $t+1, \ldots, N$, når historikken til beslutningstidspunktet $t+1$ er $h_{t+1} = (h_t, a, s')$. 

Da der summes over $s' \in S$, følger det af \autoref{def:betinget_forventet_værdi_af_diskrete_tilfældige_variabler}, at \eqref{eq:ligning_i_algoritme1} kan udtrykkes som 
\begin{align}\label{eq:algoritme_ligning_vol2}
    u_t^\Psi(h_t)=\sum_{a \in A_{s_t}}q_{d_t(h_t)}(a)\left(r_t\left(s_t,a\right)+E_{h_t}^\Psi\left[u_{t+1}^\Psi\left(h_t, a, X_{t+1}\right)\right]\right).
\end{align}
Ved at bestemme den forventede belønning betinget af historikken op til $t+1$, er $X_{t+1}$ kendt. Da algoritmen evaluerer $u^\Psi_{t+1}$ for alle $h_{t+1}$ før $u^\Psi_t$, kan den forventede værdi over $X_{t+1}$ bestemmes.

Følgende resultat viser, at \autoref{Algoritme1} bestemmer $u_t^\Psi$.

\begin{minipage}\textwidth
\begin{thmx} \textbf{Validering af \autoref{Algoritme1}} \label{sæt:den_gælder}%Ny sætning
\newline
Lad $\Psi \in \Pi^{FT}$ være en strategi og antag, at $u_t^\Psi$, for $t \leq N$ er blevet bestemt ved \autoref{Algoritme1}. Så er \eqref{eq:forventede_totale_belønning} opfyldt for $t\leq N$, og $v_N^\Psi(s)=u_1^\Psi(s)$ for alle $s\in S$.
\end{thmx}
\end{minipage}

\begin{bev} \textbf{} %Nyt bevis
\newline
Lad $\Psi \in \Pi^{FT}$ være en strategi og antag, at $u_t^\Psi$, for $t \leq N$ er blevet bestemt ved \autoref{Algoritme1}. Resultat bevises ved baglæns induktion. 

Fra \eqref{eq:u_N=r_N} er resultatet opfyldt for $t=N$, og dermed er induktionsstarten bevist. Lad nu  \eqref{eq:forventede_totale_belønning} være gældende for $t = n+1,\ldots, N$. Det skal vises at dette medfører, at \eqref{eq:forventede_totale_belønning} for $t = n$ er gældende. Så vil der ud fra \eqref{eq:algoritme_ligning_vol2} og ved induktion gælde, at
%
\begin{align*}
     u_n^\Psi(h_n)&=\sum_{a \in A_{s_n}}q_{d_t(h_n)}(a)\left(r_n\left(s_n,a\right)+E_{h_n}^\Psi\left[E_{h_{n+1}}^\Psi \left[ \sum_{k=n+1}^{N-1}r_k(X_k,Y_k)+r_N(X_N)\right]\right]\right)\\
     &= \sum_{a \in A_{s_n}}q_{d_n(h_n)}(a)\left(r_n\left(s_n,a\right)+E_{h_n}^\Psi \left[ \sum_{k=n+1}^{N-1}r_k(X_k,Y_k)+r_N(X_N)\right]\right)
     \intertext{
     %Givet at $X_n = s_n$ og $Y_n = a$, kan belønningsfunktionen indsættes i den forventede værdi af summen, således at.
     %Da $s_n$ og $h_n$ er givet ved beslutningstiden $n, X_n=s_n$, er%
     %Da $s_n$ er kendt til beslutningstidspunktet $n$, som $X_n = s_n$, vil det gælde, at givet $Y_n = a$, kan belønningen $r_n(s_n, a)$ kunne skrives som $r_n(X_n, Y_n)$ og indsættes i summen som følgende%
      Da $s_n$ er kendt til beslutningstidspunktet $n$, som $X_n = s_n$, vil det gælde, at belønningen $r_n(s_n, a)$ kan indsættes i summen, givet at $Y_n = a$. Altså
     }
     u_n^\Psi(h_n)&= \sum_{a \in A_{s_n}}q_{d_n(h_n)}(a)\left(E_{h_n}^\Psi \left[ \sum_{k=n}^{N-1}r_k(X_k,Y_k)+r_N(X_N)| Y_n = a\right]\right)\\
      u_n^\Psi(h_n)&= E_{h_n}^\Psi \left[ \sum_{k=n}^{N-1}r_k(X_k,Y_k)+r_N(X_N)|Y_n = a\right].
\end{align*}
Dermed er \autoref{sæt:den_gælder} bevist.
%så kan det første udtryk i \eqref{eq: 4.1.13} bestemmes ved forventningen, og dermed er sætningen opfyldt.
\end{bev}

Hvis strategien er fortidsafhængig deterministisk, $\Psi\in \Pi^{FD}$, eksisterer der ét $a\in A_s$ således, at $q_{d_t(h_t)}(a)=1$. Dermed erstattes \eqref{eq:ligning_i_algoritme1} i \autoref{Algoritme1} med 
\begin{align}\label{eq:uPsi_for_FD}
        \vspace{-0.5cm}u_t^\Psi(h_t) = r_t\left(s_t, d_t(h_t)\right) + \sum_{s' \in S} p_t\left(s'|s_t,d_t(h_t)\right)u_{t+1}^\Psi\left(h_t,d_t(h_t),s'\right),
    \end{align}
hvor $\left(h_t, d_t(h_t), s'\right) \in H_{t+1}$.
 
For henholdsvis en Markov tilfældig og en Markov deterministisk strategi erstattes \eqref{eq:ligning_i_algoritme1} med 
\begin{align*}
        \vspace{-0.5cm}u_t^\Psi(s_t) &=\sum_{a \in A_{s_t}}q_{d_t(s_t)}(a)\left( r_t\left(s_t, a\right) + \sum_{s' \in S} p_t\left(s'|s_t,a)\right)u_{t+1}^\Psi(s')\right) \text{ og }\\
        \vspace{-0.5cm}u_t^\Psi(s_t) &= r_t\left(s_t, d_t(s_t)\right) + \sum_{s' \in S} p_t\left(s'|s_t,d_t(s_t)\right)u_{t+1}^\Psi(s').
\end{align*}
Altså afhænger $u_t^\Psi$ kun af historikken, $h_t$, gennem $s_t$ for $t = 1, 2, \ldots, N$.
%Dette gælder, da beslutningsreglen, $d_t$, for en markov strategi kun afhænger af den forrige tilstand og beslutninger gennem den nuværende tilstand.

\subsection{Den maksimale forventede totale belønning}
For at bestemme den maksimale forventede totale belønning, $u_t^*$, skal beslutningstageren vælge den strategi, der opfylder
\begin{align}\label{eq:u*}
    u_t^*(h_t)=\sup_{\Psi\in\Pi^{FT}}u_t^\Psi(h_t),
\end{align}
for beslutningstidspunkterne $t, t+1,\ldots, N-1$. 

Følgende ligninger kaldes for \textit{Bellman ligningerne} og anvendes til at bestemme den maksimale forventede totale belønning
\begin{align}\label{eq:u_t_sup}
   u_t(h_t)=\sup_{a\in A_{s_t}}\left(r_t(s_t, a)+\sum_{s'\in S}p_t(s'|s_t, a)u_{t+1}(h_t, a, s')\right)
\end{align}
for $t=1, 2, \ldots, N-1$ og $h_t=(h_{t-1}, a_{t-1}, s_t)\in H_t$.
Såfremt $t= N$ og $h_N = (h_{N-1}, a_{N-1}, s_N) \in H_N$, så er 
\begin{align}\label{eq:u_N}
    u_N(h_N) = r_N(s_N). 
\end{align}

%Når $A_{s_t}$ for eksempel er endeligt vil, $u_t$, være givet ved
Hvis beslutningsrummet, $A_{s_t}$, er endeligt, så er $u_t$ givet ved
\begin{align}\label{eq:u_t_max}
   u_t(h_t)=\max_{a\in A_{s_t}}\left(r_t(s_t, a)+\sum_{s'\in S}p_t(s'|s_t, a)u_{t+1}(h_t, a, s')\right).
\end{align}
%Ligningerne \eqref{eq:u_t_sup} og \eqref{eq:u_t_max} er en sekvens af funktioner $u_t: H_t \to \R$ for $t=1, 2, \ldots, N-1$. 
Løsningerne til ligningerne er de maksimale forventede totale belønninger fra beslutningstidspunktet $t$ til $N$ for ethvert $t$ og $h_t$. Derudover kan de anvendes til at afgøre, om en strategi er optimal. 


% \begin{enumerate}
%     \item Løsningerne til ligningerne er de optimale belønninger fra beslutningstidspunktet $t$ til $N$ for ethvert $t$.
%     \item De kan afgøre om en strategi er optimal.
%     \item De bestemmer en effektiv metode til at finde de optimale belønningsfunktioner og strategier.
%     \item De kan bestemme egenskaber for strategier og belønningsfunktioner.
% \end{enumerate}
Disse egenskaber for Bellman ligningerne præsenteres i følgende resultat.

\begin{minipage}\textwidth
\begin{thmx} \textbf{Egenskaber for Bellman ligningerne} \label{sæt:ret_så_vigtig}%Ny sætning
\newline
Lad $u_t$ være en løsning til \eqref{eq:u_t_sup} for $t = 1, 2, \ldots, N-1$ og $u_N$ være givet som i \eqref{eq:u_N}. Lad derudover $u_t^*$ være den maksimale forventede totale belønning. Da gælder det, at
\begin{enumerate}
    \item $u_t(h_t)=u_t^*(h_t)$ for alle $h_t\in H_t$ og for $t=1, 2,\ldots, N$
    \item $u_1(s_1)=v_N^*(s_1)$ for alle $s_1\in S$.
\end{enumerate}
\end{thmx}
\end{minipage}

For at bevise dette introduceres følgende lemma

\begin{minipage}\textwidth
\begin{lem} \label{lem:ret_vigtig}\textbf{} %Nyt lemma
\newline
Lad $G$ være en vilkårlig diskret mængde og $g: G\to \R$. Lad derudover $q$ være en sandsynlighedsfordeling på $G$. Så er
\begin{align*}
    \sup_{z\in G}g(z)\geq \sum_{z\in G}q(z) g(z).
\end{align*}
\end{lem}
\end{minipage}

\begin{bev} \textbf{} %Nyt bevis
\newline
Lad $G$ være en vilkårlig diskret mængde og $g$ være en reel funktion på $G$. Lad derudover $q$ være sandsynlighedsfordelingen på $G$. Såfremt $\displaystyle g^*=\sup_{z\in G}g(z)$, så er 
\begin{align*}
    g^* = \sum_{z \in G} q(z)g^* \geq \sum_{z \in G} q(z)g(z).
\end{align*}
Første lighed er fra \autoref{prop:frekvensfunktion}, og sidste ulighed gælder, da $g^* \geq g(z)$ jævnfør \autoref{sæt_om_eps}. Dermed er \autoref{lem:ret_vigtig} bevist.
\end{bev}

Det er hermed muligt at bevise \autoref{sæt:ret_så_vigtig}.

\begin{bev} \textbf{} %Nyt bevis 
\newline
Lad $u_t$ være en løsning til \eqref{eq:u_t_sup} for $t = 1, 2, \ldots, N-1$ og $u_N$ være givet som i \eqref{eq:u_N}. Lad derudover $u_t^*$ være den maksimale forventede totale belønning. 

\textbf{Bevis for punkt 1}

Først bevises det ved induktion, at $u_n(h_n) \geq u_n^*(h_n)$ for alle $h_n \in H_n$ og $n = 1, 2, \ldots, N$.
Da der ikke tages en beslutning til beslutningstidspunktet  $N$, følger det af \eqref{eq:u_N=r_N} og \eqref{eq:u_N}, at
\begin{align*}
    u_N(h_N) = r_N(s_N) = u_N^\Psi(h_N), \text{ for alle } h_N \in H_N \text{ og } \Psi \in \Pi^{FT}.
\end{align*}
Dermed er induktionsstarten bevist. Lad nu $u_t(h_t) \geq u_t^*(h_t)$ for alle $h_t \in H_t$ for $t=n+1, \ldots, N$, og lad $\Psi' = (d'_1, d'_2, \ldots, d'_{N-1})$ være en vilkårlig strategi i $\Pi^{FT}$. Det skal vises, at dette medfører, at $u_t(h_t) \geq u_t^*(h_t)$ for $t=n$. For $t=n$ er $u_n(h_n)$ givet ud fra $\eqref{eq:u_t_sup}$, altså
\begin{align*}
    u_n(h_n) &=\sup_{a\in A_{s_t}} \left(r_n(s_n,a)+ \sum_{s'\in S} p_n(s'|s_n,a)u_{n+1}(h_n,a,s') \right).
    \intertext{Af induktionshypotesen gælder det, at }
    u_n(h_n)&\geq \sup_{a\in A_{s_t}} \left(r_n(s_n,a)+ \sum_{s'\in S} p_n(s'|s_n,a)u^*_{n+1}(h_n,a,s') \right).
    \intertext{Ud fra definitionen af $u_{n+1}^*$, \eqref{eq:u*}, følger det, at} 
     u_n(h_n)&\geq \sup_{a\in A_{s_t}} \left(r_n(s_n,a)+ \sum_{s'\in S} p_n(s'|s_n,a)u^{\Psi'}_{n+1}(h_n,a,s') \right).
    \intertext{Jævnfør \autoref{lem:ret_vigtig}, er}
     u_n(h_n)&\geq \sum q_{d_n(h_n)}(a)\left( r_n(s_n,a)+\sum_{s'\in S}p_n(s'|s_n,a)u_{n+1}^{\Psi'}(h_n,a,s')\right)\\
    &=u_n^{\Psi'}(h_n).  
\end{align*}
Hvoraf den sidste lighed gælder ud fra \eqref{eq:ligning_i_algoritme1}. Da $\Psi'$ er vilkårlig, følger det, at
\begin{align*}
    u_n(h_n)\geq u_n^\Psi (h_n) \text{ for alle } \Psi\in\Pi^{FT}.
\end{align*}
Dermed er $u_n(h_n) \geq u^*_n(h_n)$.

Herefter vises, at $u^*_n(h_n) \geq u_n(h_n)$. Dette gøres ved først at vise, at der for ethvert $\varepsilon > 0$, eksisterer en strategi, $\Psi' \in \Pi^{FD}$, således, at
\begin{align}\label{eq:4.3.8}
   u_n^{\Psi'}(h_n)+(N-n)\varepsilon\geq u_n(h_n),
\end{align}
for alle $h_n \in H_n$ og $n = 1, 2, \ldots, N$. For at bevise dette konstrueres en strategi, $\Psi' = (d_1, d_2, \ldots, d_{N-1})$, ved at vælge $d_n(h_n)$, der opfylder
\begin{align}\label{eq:bevis_pt_2}
    &r_n\left(s_n,d_n(h_n)\right)+\sum_{s'\in S}p_n\left(s'|s_n, d_n(h_n)\right)u_{n+1}\left(s_n, d_n(h_n), s'\right)+\varepsilon\geq u_n(h_n), 
\end{align}
som er muligt jævnfør \autoref{sæt_om_eps}. 

Ved induktion vises \eqref{eq:4.3.8}. Da der ikke tages en beslutning til beslutningstidspunktet $N$, følger det af \eqref{eq:u_N=r_N} og \eqref{eq:u_N}, at
%
\begin{align*}
    u_N(h_N) = r_N(s_N) = u_N^{\Psi'}(h_N), \text{ for alle } h_N \in H_N .
\end{align*}
Dermed er induktionsstarten bevist. 
Lad $u_t^{\Psi'}(h_t)+(N-t)\varepsilon\geq u_t(h_t)$ for $t=n+1, \ldots, N$.
Det skal vises, at dette medfører, at $u_n^{\Psi'}(h_n)+(N-n)\varepsilon\geq u_n(h_n)$. Ud fra \eqref{eq:uPsi_for_FD} er
\begin{align*}
    u_n^{\Psi'}&=r_n\left(s_n, d_n(h_n)\right)+\sum_{s'\in S}p_n\left(s'| s_n, d_n(h_n)\right)u_{n+1}^{\Psi'}\left(s_n, d_n(h_n), s'\right).
    \intertext{Ved at anvende induktionshypotesen fås}
    u_n^{\Psi'} &\geq r_n\left(s_n,d_n(h_n)\right) + \sum_{s' \in S} p_n\left(s' | s_n, d_n(h_n)\right)u_{n+1}\left(s_n, d_n(h_n),s'\right) - (N-n-1)\varepsilon.
    \intertext{Af \eqref{eq:bevis_pt_2} følger det, at}
    u_n^{\Psi'}&\geq u_n(h_n) - (N-n)\varepsilon.
\end{align*}
Hermed er det bevist, at $u_n^{\Psi'}(h_n)+(N-n)\varepsilon\geq u_n(h_n)$ for $n = 1, 2, \ldots, N$. 
Derfor gælder det for ethvert $\varepsilon>0$, at der eksisterer en strategi, $\Psi'\in\Pi^{FT}$, således, at
\begin{align*}
    &u_n^*(h_n) + (N-n)\varepsilon \geq u_n^{\Psi'}(h_n) + (N-n)\varepsilon \geq u_n(h_n) %\geq u_n^*(h_n)
\end{align*}
Der eksisterer dermed en strategi, hvorom det gælder, at for et tilpas lille $\varepsilon > 0$, så er $u_n^*(h_n) \geq u_n(h_n)$.  
Da det nu er bevist, at $u_n^*(h_n) \geq u_n(h_n)$ og $ u_n(h_n) \geq u_n^*(h_n)$, må $u_n^*(h_n) = u_n(h_n)$.

\textbf{Bevis for punkt 2}\\
Andet punkt bevises ved følgende ligning
\begin{align} \label{bevis_for_punkt_2}
    u_1(s_1)=u_1^*(s_1)=\sup_{\Psi\in \Pi^{FT}}u_1^\Psi(s_1)=\sup_{\Psi\in \Pi^{FT}}v_N^\Psi(s_1)=v_N^*(s_1).
\end{align}
Den første lighed i \eqref{bevis_for_punkt_2}, følger af punkt 1 i \autoref{sæt:ret_så_vigtig} og den anden lighed af \eqref{eq:u*}. Derudover gælder tredje lighed af \autoref{sæt:den_gælder} og fjerde lighed jævnfør \eqref{eq:v_N=sup}.

Hermed er \autoref{sæt:ret_så_vigtig} bevist.
\end{bev}

Af \autoref{sæt:ret_så_vigtig} punkt 1 følger det, at løsningerne til Bellman ligningerne er den maksimale forventede totale belønning, $u_t^*$, for beslutningstidspunkterne $t, t+1, \ldots, N$.
Derudover gælder det fra punkt 2, at løsningerne til Bellman ligningerne for $n=1$ er den optimale belønningsfunktion, $v_N^*$, for alle beslutningstidspunkter. % $1, 2,\ldots, N$.

Bellman ligningerne kan anvendes til at bestemme den optimale strategi, $\Psi^*$, og bekræfte, at en strategi er optimal. Hvis der eksisterer en optimal strategi, gælder følgende resultat.
\begin{minipage}\textwidth
\begin{thmx}\label{sæt:optimal_strategi_ved_Bellman} \textbf{Optimal strategi ud fra Bellman ligningerne} %Ny sætning
\newline
Lad $u_t^*$ være en løsning til Bellman ligningerne, \eqref{eq:u_N} og \eqref{eq:u_t_max}, for $t=1,2, \ldots, N$. Lad derudover $\Psi^*=(d_1^*, d_2^*, \ldots, d_{N-1}^*)\in \Pi^{FD}$ være en strategi, der opfylder, at
\begin{align}\label{eq:optimal_strategi}
    &r_t\left(s_t, d_t^*(h_t)\right)+\sum_{s'\in S}p_t\left(s'|s_t, d_t^*(h_t)\right)u_{t+1}^*\left(h_t, d_t^*(h_t), s'\right)\nonumber\\
    &=\max_{a\in A_{s_t}}\left(r_t(s_t,a)+\sum_{s'\in S}p_t(s'|s_t, a)u_{t+1}^*(h_t, a, s')\right)
\end{align}
for $t=1, 2, \ldots, N-1$.
Så gælder det, at
\begin{enumerate}
    \item for hvert $t = 1, 2, \ldots, N$ er
    \begin{align*}
        u_t^{\Psi^*}(h_t) = u_t^*(h_t), \text{ for } h_t \in H_t.
    \end{align*}
    \item $\Psi^*$ er en optimal strategi, og 
    \begin{align*}
      v_N^{\Psi^*}(s)  = v_N^*(s), \text{ for } s \in S.
    \end{align*}
\end{enumerate}
\end{thmx}
\end{minipage}

\begin{bev} \textbf{} %Nyt bevis
\newline
Lad $u_t^*$ være en løsning til Bellman ligningerne, \eqref{eq:u_N} og \eqref{eq:u_t_max}, for $t=1, 2, \ldots, N$. Lad derudover $\Psi^*=(d_1^*, d_2^*, \ldots, d_{N-1}^*)\in \Pi^{FD}$ være en strategi.

\textbf{Bevis for punkt 1}\\
Dette bevises ved induktion. Af \autoref{sæt:ret_så_vigtig}, \eqref{eq:u_N} og \eqref{eq:u_N=r_N} følger det, at
\begin{align*}
    u_N^*(h_N)=u_N(h_N)=r_N(s) = u^{\Psi^*}_N(h_N), \text{ for } h_N\in H_N.
\end{align*}
Dermed er induktionsstarten bevist. Lad nu $u_t^*(h_t) = u_t^{\Psi^*}(h_t)$ for $t = n+1, \ldots, N$. Det skal vises, at dette medfører, at $u_t^*(h_t) = u_t^{\Psi^*}(h_t)$ for $t=n$. Af \eqref{eq:u_t_max} er
\begin{align*}
    u_n^*(h_n)&=\max_{a\in A_{s_n}}\left(r_n(s_n, a)+\sum_{s'\in S}p_n(s'|s_n, a)u_{n+1}^*(h_n, a, j)\right).
    \intertext{Fra \eqref{eq:optimal_strategi} fås, at}
    u_n^*(h_n)&=r_n\left(s_n, d_n^*(h_n)\right) + \sum_{s' \in S} p_n\left(s'|s_n, d_n^*(h_n)\right)u_{n+1}^{*}\left(h_n, d_n^*(h_n),s'\right).
    \intertext{Ved at anvende induktionshypotesen fås}
    u_n^*(h_n)&=r_n\left(s_n, d_n^*(h_n)\right) + \sum_{s' \in S} p_n\left(s'|s_n, d_n^*(h_n)\right)u_{n+1}^{\Psi^*}\left(h_n, d_n^*(h_n),s'\right)=u_n^{\Psi^*}(h_n)
\end{align*}
for $h_n=\left(h_{n-1}, d_{n-1}^*(h_{n-1}\right), s_n)$. Dermed er det bevist, at $u_n^*(h_n) = u_n^{\Psi^*}(h_n)$.

\textbf{Bevis for punkt 2}\\
Det følger af \autoref{sæt:den_gælder} og \autoref{sæt:ret_så_vigtig} punkt 2, at 
\begin{align*}
    v_N^{\Psi^*}(s)=v_N^*(s) \text{ for } s\in S,
\end{align*}
og derfor er $\Psi^*$ en optimal strategi.

Dermed er \autoref{sæt:optimal_strategi_ved_Bellman} bevist.
\end{bev}

Fra \autoref{sæt:optimal_strategi_ved_Bellman} gælder det dermed, at den optimale strategi bestemmes ved først at løse Bellman ligningerne. Dernæst vælges en beslutningsregel til enhver historik, der bestemmer den beslutning, der medfører, at højresiden af \eqref{eq:optimal_strategi} for $t=1, 2, \ldots, N$ er maksimal.

I \autoref{sæt:optimal_strategi_ved_Bellman} er strategien fortidsafhængig deterministisk. Hvis der eksisterer en fortidsafhængig tilfældig strategi, der opfylder det generaliserede udtryk for \eqref{eq:optimal_strategi}, så af \autoref{lem:ret_vigtig} eksisterer der også en fortidsafhængig deterministisk, der opfylder \eqref{eq:optimal_strategi}. Dermed er det kun nødvendigt at vise \autoref{sæt:optimal_strategi_ved_Bellman} for fortidsafhængige deterministiske strategier. 


%Dette gælder, da lemmaet resulterer i, at en sådan strategi kan bestemmes selvom strategien havde været fortidsafhæng tilfældig.
%Hvis der eksisterer en fortidsafhængig tilfældig strategi der opfylder ... vil der fra lemma kunne bestemmes en deterministisk strategi der opfylder...
%Punkt 1 i \autoref{sæt:optimal_strategi_ved_Bellman} kaldes også for \textit{Optimalitets-princippet}.

% Den optimale strategi, $\Psi^{\ast}$, blev defineret ved \eqref{eq:optimal_strategi}, hvor max resulterer i en reel værdi. \eqref{eq:optimal_strategi} kan udtrykkes som følgende
% \begin{align*}
%     d_t^{\ast}(h_t)=\arg\max_{a\in A_{s_t}}\left\{r_t(s_t, a)+\sum_{j\in S_{t}}p_t(j,s_t, a)u_{t+1}^{\ast}(h_t, a, j)\right\},
% \end{align*}
% hvor argmax resulterer i en mængde, mens max resulterer i en reel værdi. I ovenstående er det udtrykket for den optimale beslutningsregel, der er givet i stedet for den maksimale forventede totale belønning.



Hvis det ikke er muligt at bestemme supremum i \eqref{eq:u_t_sup}, kan beslutningstageren vælge den strategi, der er $\varepsilon$-optimal. I dette projekt er dette ikke relevant til problemløsningen, men da resultatet anvendes fremadrettet, præsenteres resultatet i \autoref{Snyd}.

I beviset til \autoref{sæt:ret_så_vigtig} blev en fortidsafhængig deterministisk $\varepsilon$-optimal strategi konstrueret, mens der i \autoref{sæt:optimal_strategi_ved_Bellman} og \autoref{sæt:epsopt} blev bestemt, hvorvidt en strategi er optimal og $\varepsilon$-optimal. Det er dermed bestemt, at der for ethvert $\varepsilon> 0$ eksisterer en $\varepsilon$-optimal strategi, der er fortidsafhængig deterministisk, og at enhver strategi i $\Pi^{FD}$ der opfylder \eqref{eq:bilag_epsilon_optimal}, er $\varepsilon$-optimal. Derudover er det bestemt, at såfremt $u_t^*$ er en løsning til \eqref{eq:u_t_sup} og \eqref{eq:u_N}, og der for ethvert $t$ og $s_t\in S$ eksisterer et $a^{\circ}\in A_{s_t}$, hvorom det gælder, at
%
\begin{align}\label{eq:deterministisk_fortidsafhængig_sup}
    &r_t(s_t,a^{\circ}) + \sum_{s'\in S}p_t(s' | s_t,a^{\circ})u^*_{t+1}(h_t, a^{\circ},s')\nonumber \\
    &= \sup_{a \in A_{s_t}}\left(r_t(s_t,a) +  \sum_{s'\in S}p_t(s' | s_t,a)u^*_{t+1}(h_t, a,s')\right)
\end{align}
for alle $h_t=(s_{t-1}, a_{t-1}, s_t)$, så eksisterer der en optimal deterministisk fortidsafhængig strategi, $\Psi^*\in \Pi^{FD}$.

I følgende sætning vises det, at der eksisterer en optimal strategi, som er Markov deterministisk.

\begin{thmx}\label{sæt:deterministisk_Markov_optimal_strategi} \textbf{Eksistens af Markov deterministisk optimal strategi} %Ny sætning
\newline
Lad $u_t^*$ være løsninger til Bellman ligningerne \eqref{eq:u_t_sup} og \eqref{eq:u_N} for $t=1, 2, \ldots, N$. Så
\begin{enumerate}
    \item afhænger $u_t^*(h_t)$ kun af $h_t$ gennem $s_t$ for ethvert $t=1, 2, \ldots, N$.
    \item eksisterer der en Markov deterministisk $\varepsilon$-optimal strategi for alle $\varepsilon>0$.
    \item eksisterer der en Markov deterministisk optimal strategi, såfremt der eksisterer et $a^\circ\in A_{s_t}$ således, at \eqref{eq:deterministisk_fortidsafhængig_sup} er opfyldt for ethvert $s_t\in S$ og $t=1, 2,\ldots, N-1$.
\end{enumerate}
\end{thmx}

\begin{bev} \textbf{} %Nyt bevis
\newline
Lad $u_t^*$ være løsninger til Bellman ligningerne \eqref{eq:u_t_sup} og \eqref{eq:u_N} for $t=1, 2,\ldots, N$.

\textbf{Bevis for punkt 1}\\
Dette bevises ved induktion. Ud fra \eqref{eq:u_N} følger det, at $u^*_N(h_N) = u^*_N(h_{N-1}, a_{N-1}, s) =r_N(s_N)$ for alle $h_{N-1} \in H_{N-1}$ og $a_{N-1} \in A_{s_{N-1}}$. Dermed gælder det, at $u^*_N(h_N) = u^*_N(s_N)$. Altså er induktionsstarten bevist. Lad nu punkt 1 være sand for $t = n+1, \ldots, N$. Det skal vises, at dette medfører, at punkt 1 er sand for $t = n$. 
Der gælder, at
\begin{align*}
    u_n^*(h_n)=\sup_{a\in A_{s_t}}\left(r_n(s_n, a)+\sum_{s'\in S}p_t\left(s'|s_n, a\right)u_{n+1}^*(h_n, a, s')\right).
    \intertext{Af induktionshypotesen følger det, at}
    u_n^*(h_n)=\sup_{a\in A_{s_t}}\left(r_n(s_n, a)+\sum_{s'\in S}p_n\left(s'|s_n, a\right)u_{n+1}^*(s')\right).
\end{align*}
Da højresiden kun afhænger af $h_n$ gennem $s_n$, er punkt 1 bevist.

\textbf{Bevis for punkt 2}\\
Lad $\varepsilon > 0$ og lad $\Psi^\varepsilon = (d_1^\varepsilon, d_2^\varepsilon, \ldots, d_{N-1}^\varepsilon)$ være en vilkårlig strategi i $\Pi^{MD}$ som opfylder, at
%
\begin{align*}
    &r_t\left(s_t, d_t^\varepsilon(s_t)\right)+\sum_{s'\in S}p_t\left(s'|s_t, d_t^\varepsilon(s_t)\right)u_{t+1}^*\left(s'\right) + \frac{\varepsilon}{N-1}\nonumber\\
    &\geq \sup_{a\in A_{s_t}}\left(r_t(s_t,a)+\sum_{s'\in S}p_t(s'|s_t, a)u_{t+1}^*( s')\right),
\end{align*}
som eksisterer jævnfør \autoref{sæt_om_eps}. Af punkt 1 er $u_t^*(h_t)$ kun afhængig af $h_t$ gennem $s_t$, dermed gælder det fra \autoref{sæt:epsopt}, at strategien, $\Psi^\varepsilon $, er $\varepsilon$-optimal.

\textbf{Bevis for punkt 3}\\
Antag, at der eksisterer et $a^\circ \in A_{s_t}$ for ethvert $t$ og $s_t$ således, at der eksisterer et $\Psi^* = (d_1^*, d_2^*, \ldots, d_{N-1}^*) \in \Pi^{MD}$, som opfylder \eqref{eq:deterministisk_fortidsafhængig_sup}. 

Eftersom supremum er antaget ved $a^{\circ}$ for ethvert $t$ og $s_t$, så er det muligt at konstruere en strategi, hvorom der gælder, at $d_t(s_t)=a^{circ}$ for alle $t$ og $s_t$. Derfor eksisterer der en strategi, således at 
%
%Dermed eksisterer supremum og altså gælder det, at $\Psi^*$ opfylder
\begin{align*}
    r_t\left(s_t,d_t^*(s_t)\right) + \sum_{s'\in S}p_t\left(s' | s_t, d_t^*(s_t)\right)u^*_{t+1}(s')\nonumber 
    = \max_{a \in A_{s_t}}\left(r_t(s_t,a) +  \sum_{s'\in S}p_t(s' | s_t,a)u^*_{t+1}(s')\right).
\end{align*}

Af punkt 1 er $u_t^*(h_t)$ kun afhængig af $h_t$ gennem $s_t$, og dermed følger det af \autoref{sæt:optimal_strategi_ved_Bellman}, at $\Psi^*$ er den optimale strategi.
Dermed er \autoref{sæt:deterministisk_Markov_optimal_strategi} bevist.
\end{bev}

Det er dermed vist at
\begin{align*}
    v_N^*(s)=\sup_{\Psi\in\Pi^{FT}}v_N^\Psi(s)=\sup_{\Psi\in\Pi^{MD}}v_N^\Psi(s) \text{ for } s\in S.
\end{align*}

Altså gælder det for en Markov beslutningsproces, at den maksimale forventede totale belønning har samme værdi for en Markov deterministisk strategi som en fortidsafhængig tilfældig strategi. Dermed er det kun nødvendigt at betragte Markov deterministiske strategier for at bestemme den maksimale forventede totale belønning. 

%Følgende proposition kan anvendes til at bestemme om der eksisterer en Markov deterministisk strategi der er optimal. 
For at bestemme hvorvidt en Markov deterministisk strategi er optimal, anvendes følgende proposition. 

\begin{minipage}\textwidth
\begin{pro} \textbf{}\label{prop:markov_det_strategi} %Ny proposition
\newline
Lad tilstandsrummet, $S$, være tællelig og lad beslutningsrummet, $A_s$, være endeligt for ethvert $s\in S$. Så eksisterer der en optimal Markov deterministisk strategi, $\Psi^*\in \Pi^{MD}$.  
\end{pro}
\end{minipage}

\begin{bev} \textbf{} %Nyt bevisw
\newline
Det er tilstrækkeligt at vise, at der for ethvert $t$ og $s_t$ findes et $a\in A_{s,t}$, så
\begin{align*}
    \sup_{a\in A_{s_t}} r_t(s_t,a) + \sum_{s'\in S}p_t\left(s' | s_t, a\right)u^*_{t+1}(s')\nonumber \\
    = \max_{a\in A_{s_t}} r_t(s_t,a) + \sum_{s'\in S}p_t\left(s' | s_t, a\right)u^*_{t+1}(s').
\end{align*}
Da $A_s$ er endelig, må der eksistere et $a$ således, at dette er opfyldt, og af \autoref{sæt:optimal_strategi_ved_Bellman} eksisterer der derfor en optimal Markov deterministisk strategi.

% Det skal vises, at der eksisterer et $a^\circ$, som opfylder \eqref{eq:deterministisk_fortidsafhængig_sup}. Af \autoref{sæt:optimal_strategi_ved_Bellman} gælder det da, at der eksisterer en optimal Markov deterministisk strategi.  
% Jævnfør \autoref{sæt:deterministisk_Markov_optimal_strategi} punkt 3 skal der for ethvert $s\in S$ eksistere et $a^\circ$, der opfylder
% \begin{align*}
%     r_t(s_t,a^\circ) + \sum_{s'\in S}p_t\left(s' | s_t, a^\circ\right)u^*_{t+1}(s')\nonumber 
%     = \sup_{a \in A_{s_t}}\left(r_t(s_t,a) +  \sum_{s'\in S}p_t\left(s' | s_t,a\right)u^*_{t+1}(s')\right).
% \end{align*}
% Da $A_s$ er endelig, må der eksistere et $a^\circ$ således, at dette er opfyldt. Dermed er \autoref{prop:markov_det_strategi} bevist.
\end{bev}







\subsection{Diskonteret belønning}
Belønninger kan variere i værdi over tid. For at tage hensyn til disse tidsafhængige belønninger introduceres en \textit{diskonteringfaktor}, som noteres $\lambda$. Denne diskonteringsfaktor kan både anvendes i beregningen af den forventede totale belønning og Bellman ligningen, \ref{eq:u_t_sup}. For at introducere denne udvidelse med diskonteringfaktoren, skal følgende antagelser gælde

\begin{enumerate}
    \item Faste belønninger og overgangssandsynligheder; $r(s,a)$ og $p(s'|s,a)$ varierer ikke fra et beslutningstidspunkt til et andet.
    \item Begrænsede belønninger; $|r_t(s,a)| < \infty$ for alle $a\in A_s$ og $s\in S$.
    \item Diskontering; fremtidige belønninger er diskonteret med hensyn til en diskonteringsfaktor $\lambda$, hvor $0 \leq \lambda < 1$.
    \item Diskrete tilstandsrum; $S$ er tællelig.
\end{enumerate}

Under de ovenstående antagelser kan den forventede totale diskonterede belønning givet en strategi, $\Psi\in \Pi^{FT}$, udtrykkes som følgende 
\begin{align*}
    v_{N,\lambda}^\Psi(s)=E_s^\Psi\left[\sum_{t=1}^{N-1}\lambda^{t-1}r_t(X_t, Y_t)+\lambda^{N-1}r_N(X_N)\right],
\end{align*}
hvor $\lambda$ skalerer værdien til beslutningstidspunkt $n$ af én enhed belønning modtaget til beslutningstidspunktet $n+1$. Én enhed belønning modtaget $t$ perioder ude i fremtiden har nutidsværdien $\lambda^t$.

Bellman ligningen, \eqref{eq:u_t_sup}, er under disse antagelser givet ved
\begin{align*}
    u_t(s) = \sup_{a \in A_s} \left( r(s,a) + \sum_{s'\in S} \lambda p\left(s'|s,a\right)u_{t+1}(s')\right).
\end{align*}
Når diskonteringfaktoren er inkluderet, vil den optimale strategi ikke nødvendigvis være den samme som bestemt uden diskonteringsfaktoren. Derudover vil den også påvirke den maksimale forventede totale belønning. 

Diskonteringsfaktoren er essentiel når indeksmængden er uendelig. Her sikre diskonteringsfaktoren at den uendelige sum i den forventede totale belønning bliver endelig, og dermed sikre at den konvergerer. 

