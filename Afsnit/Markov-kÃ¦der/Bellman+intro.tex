\section{Optimering af belønning}\label{afsnit:optimering_af_belønning}
I dette afsnit introduceres en måde, hvorpå den maksimale forventede totale belønning kan bestemmes. I følgende afsnit er indeksmængden endelig. 
%Følgende gælder for en endelig indeksmængde. 

Lad en strategi, $\Psi = (d_1,d_2, \ldots, d_{N-1})$, være fortidsafhængig tilfældig og $u_t^\Psi: H_t \to \R$ være den forventede totale belønning for $\Psi$ til beslutningstidspunkterne $t, t+1, \ldots, N-1$. For historikken, $h_t \in H_t$, er $u_t^\Psi$ givet ved 
% 
\begin{align}\label{eq:forventede_totale_belønning} 
    u_t^{\Psi}(h_t)=E^{\Psi}_{h_t}\left[\sum^{N-1}_{n=t}r_n(X_n, Y_n)+r_N(X_N)\right] \text{ for } t < N. 
\end{align}
For $t=N$, så er $h_N=(h_{N-1},a_{N-1},s)$ og
\begin{align}\label{eq:u_N=r_N}
    u_N^{\Psi}(h_N)=r_N(s). 
\end{align}
Forskellen mellem $v_N^\Psi$ og $u_t^\Psi$ er, at $v_N^\Psi$ betegner den forventede totale belønning for alle beslutningstidspunkter, mens $u_t^\Psi$ kun defineres ud fra de fremtidige beslutninger begyndende ved $t$. Bemærk, at $u_1^\Psi(s) = v_N^\Psi(s)$ for $h_1 = s$. 

%Altså gælder det, at $v_N^{\Psi}$ er defineret ud fra alle fremtidige beslutninger og tilstande, mens $u_t^{\Psi}$ er defineret ud fra en del af de fremtidige beslutninger begyndende ved $t$.
%Dog er $u_1^{\Psi}(s)=v_N^{\Psi}(s)$, hvis det for alle $s\in S$ er opfyldt, at $h_1=s$.

Ved at bestemme $u_t^{\Psi}$ induktivt er det derved muligt at bestemme $v_N^\Psi$. Altså anvendes baglæns induktion til at bestemme den forventede totale belønning. 
Lad $\Psi\in \Pi^{FT}$ og lad $S$ være et diskret tilstandsrum. Så er det muligt at bestemme $u_t^\Psi$ med følgende evalueringsalgoritme for en endelig strategi. %\textit{endelige strategi evalueringsalgoritme}, som er givet ved følgende

\begin{alg} \textbf{Evalueringsalgoritme for en endelig strategi} \label{Algoritme1}%Ny algoritme
\begin{enumerate}
    \item Sæt $t = N $ og $u_N^\Psi(h_N) = r_N(s_N)$ for alle $h_N = (h_{N-1}, a_{N-1}, s_N) \in H_N$.
    \item Hvis $t = 1$, stop, ellers gå til trin 3.
    \item Udskift $t$ med $t-1$ og beregn $u_t^\Psi(h_t)$ for hvert $h_t = (h_{t-1}, a_{t-1}, s_t)\in H_t$ ved \begin{align}\label{eq:ligning_i_algoritme1}
        \vspace{-0.5cm}u_t^\Psi(h_t) = \sum_{a \in A_{s_t}}q_{d_t(h_t)}(a)\left(r_t(s_t, a) + \sum_{s' \in S} p_t\left(s'|s_t,a\right)u_{t+1}^\Psi(h_t,a,s')\right),
    \end{align}
    hvor $(h_t, a ,s') \in H_{t+1}$.
    \item Gå til trin 2.
\end{enumerate}
\end{alg}

Givet historikken $h_t$ kan \eqref{eq:ligning_i_algoritme1} bruges til at evaluere en givet strategi. Altså er den forventede totale belønning lig belønningen modtaget ved at tage beslutningen $a$ adderet med den forventede belønning for de resterende beslutningstidspunkter, $t+1, \ldots, N$. Alt dette multipliceres med sandsynlighedsfordelingen, $q_{d_t(h_t)}$, som er sandsynligheden for at tage beslutningen $a$ givet $h_t$. 

Summen over $s' \in S$ i \eqref{eq:ligning_i_algoritme1} indeholder produktet mellem sandsynligheden for at være i tilstanden $s'$ til beslutningstidspunktet $t+1$, hvis beslutningen $a$ er valgt, og den forventede totale belønning ved at bruge strategien, $\Psi$, for perioderne $t+1, \ldots, N$, når historikken til beslutningstidspunktet $t+1$ er $h_{t+1} = (h_t, a, s')$. 

Da der summes over $s' \in S$, følger det af \autoref{def:betinget_forventet_værdi_af_diskrete_tilfældige_variabler}, at \eqref{eq:ligning_i_algoritme1} kan udtrykkes som 
\begin{align}\label{eq:algoritme_ligning_vol2}
    u_t^\Psi(h_t)=\sum_{a \in A_{s_t}}q_{d_t(h_t)}(a)\left(r_t\left(s_t,a\right)+E_{h_t}^\Psi\left[u_{t+1}^\Psi\left(h_t, a, X_{t+1}\right)\right]\right).
\end{align}
Ved at bestemme den forventede belønning betinget af historikken op til $t+1$, er $X_{t+1}$ kendt. Da algoritmen evaluerer $u^\Psi_{t+1}$ for alle $h_{t+1}$ før $u^\Psi_t$, kan den forventede værdi over $X_{t+1}$ bestemmes.

Følgende resultat viser, at \autoref{Algoritme1} bestemmer $u_t^\Psi$.

\begin{minipage}\textwidth
\begin{thmx} \textbf{Validering af \autoref{Algoritme1}} \label{sæt:den_gælder}%Ny sætning
\newline
Lad $\Psi \in \Pi^{FT}$ være en strategi og antag, at $u_t^\Psi$, for $t \leq N$ er blevet bestemt ved \autoref{Algoritme1}. Så er \eqref{eq:forventede_totale_belønning} opfyldt for $t\leq N$, og $v_N^\Psi(s)=u_1^\Psi(s)$ for alle $s\in S$.
\end{thmx}
\end{minipage}

\begin{bev} \textbf{} %Nyt bevis
\newline
Lad $\Psi \in \Pi^{FT}$ være en strategi og antag, at $u_t^\Psi$, for $t \leq N$ er blevet bestemt ved \autoref{Algoritme1}. Resultat bevises ved baglæns induktion. 

Fra \eqref{eq:u_N=r_N} er resultatet opfyldt for $t=N$, og dermed er induktionsstarten bevist. Lad nu  \eqref{eq:forventede_totale_belønning} være gældende for $t = n+1,\ldots, N$. Det skal vises at dette medfører, at \eqref{eq:forventede_totale_belønning} for $t = n$ er gældende. Så vil der ud fra \eqref{eq:algoritme_ligning_vol2} og ved induktion gælde, at
%
\begin{align*}
     u_n^\Psi(h_n)&=\sum_{a \in A_{s_n}}q_{d_t(h_n)}(a)\left(r_n\left(s_n,a\right)+E_{h_n}^\Psi\left[E_{h_{n+1}}^\Psi \left[ \sum_{k=n+1}^{N-1}r_k(X_k,Y_k)+r_N(X_N)\right]\right]\right)\\
     &= \sum_{a \in A_{s_n}}q_{d_n(h_n)}(a)\left(r_n\left(s_n,a\right)+E_{h_n}^\Psi \left[ \sum_{k=n+1}^{N-1}r_k(X_k,Y_k)+r_N(X_N)\right]\right)
     \intertext{
     %Givet at $X_n = s_n$ og $Y_n = a$, kan belønningsfunktionen indsættes i den forventede værdi af summen, således at.
     %Da $s_n$ og $h_n$ er givet ved beslutningstiden $n, X_n=s_n$, er%
     %Da $s_n$ er kendt til beslutningstidspunktet $n$, som $X_n = s_n$, vil det gælde, at givet $Y_n = a$, kan belønningen $r_n(s_n, a)$ kunne skrives som $r_n(X_n, Y_n)$ og indsættes i summen som følgende%
      Da $s_n$ er kendt til beslutningstidspunktet $n$, som $X_n = s_n$, vil det gælde, at belønningen $r_n(s_n, a)$ kan indsættes i summen, givet at $Y_n = a$. Altså
     }
     u_n^\Psi(h_n)&= \sum_{a \in A_{s_n}}q_{d_n(h_n)}(a)\left(E_{h_n}^\Psi \left[ \sum_{k=n}^{N-1}r_k(X_k,Y_k)+r_N(X_N)| Y_n = a\right]\right)\\
      u_n^\Psi(h_n)&= E_{h_n}^\Psi \left[ \sum_{k=n}^{N-1}r_k(X_k,Y_k)+r_N(X_N)|Y_n = a\right].
\end{align*}
Dermed er \autoref{sæt:den_gælder} bevist.
%så kan det første udtryk i \eqref{eq: 4.1.13} bestemmes ved forventningen, og dermed er sætningen opfyldt.
\end{bev}

Hvis strategien er fortidsafhængig deterministisk, $\Psi\in \Pi^{FD}$, eksisterer der ét $a\in A_s$ således, at $q_{d_t(h_t)}(a)=1$. Dermed erstattes \eqref{eq:ligning_i_algoritme1} i \autoref{Algoritme1} med 
\begin{align}\label{eq:uPsi_for_FD}
        \vspace{-0.5cm}u_t^\Psi(h_t) = r_t\left(s_t, d_t(h_t)\right) + \sum_{s' \in S} p_t\left(s'|s_t,d_t(h_t)\right)u_{t+1}^\Psi\left(h_t,d_t(h_t),s'\right),
    \end{align}
hvor $\left(h_t, d_t(h_t), s'\right) \in H_{t+1}$.
 
For henholdsvis en Markov tilfældig og en Markov deterministisk strategi erstattes \eqref{eq:ligning_i_algoritme1} med 
\begin{align*}
        \vspace{-0.5cm}u_t^\Psi(s_t) &=\sum_{a \in A_{s_t}}q_{d_t(s_t)}(a)\left( r_t\left(s_t, a\right) + \sum_{s' \in S} p_t\left(s'|s_t,a)\right)u_{t+1}^\Psi(s')\right) \text{ og }\\
        \vspace{-0.5cm}u_t^\Psi(s_t) &= r_t\left(s_t, d_t(s_t)\right) + \sum_{s' \in S} p_t\left(s'|s_t,d_t(s_t)\right)u_{t+1}^\Psi(s').
\end{align*}
Altså afhænger $u_t^\Psi$ kun af historikken, $h_t$, gennem $s_t$ for $t = 1, 2, \ldots, N$.
%Dette gælder, da beslutningsreglen, $d_t$, for en markov strategi kun afhænger af den forrige tilstand og beslutninger gennem den nuværende tilstand.

\subsection{Den maksimale forventede totale belønning}
For at bestemme den maksimale forventede totale belønning, $u_t^*$, skal beslutningstageren vælge den strategi, der opfylder
\begin{align}\label{eq:u*}
    u_t^*(h_t)=\sup_{\Psi\in\Pi^{FT}}u_t^\Psi(h_t),
\end{align}
for beslutningstidspunkterne $t, t+1,\ldots, N-1$. 

Følgende ligninger kaldes for \textit{Bellman ligningerne} og anvendes til at bestemme den maksimale forventede totale belønning
\begin{align}\label{eq:u_t_sup}
   u_t(h_t)=\sup_{a\in A_{s_t}}\left(r_t(s_t, a)+\sum_{s'\in S}p_t(s'|s_t, a)u_{t+1}(h_t, a, s')\right)
\end{align}
for $t=1, 2, \ldots, N-1$ og $h_t=(h_{t-1}, a_{t-1}, s_t)\in H_t$.
Såfremt $t= N$ og $h_N = (h_{N-1}, a_{N-1}, s_N) \in H_N$, så er 
\begin{align}\label{eq:u_N}
    u_N(h_N) = r_N(s_N). 
\end{align}

%Når $A_{s_t}$ for eksempel er endeligt vil, $u_t$, være givet ved
Hvis beslutningsrummet, $A_{s_t}$, er endeligt, så er $u_t$ givet ved
\begin{align}\label{eq:u_t_max}
   u_t(h_t)=\max_{a\in A_{s_t}}\left(r_t(s_t, a)+\sum_{s'\in S}p_t(s'|s_t, a)u_{t+1}(h_t, a, s')\right).
\end{align}
%Ligningerne \eqref{eq:u_t_sup} og \eqref{eq:u_t_max} er en sekvens af funktioner $u_t: H_t \to \R$ for $t=1, 2, \ldots, N-1$. 
Løsningerne til ligningerne er de maksimale forventede totale belønninger fra beslutningstidspunktet $t$ til $N$ for ethvert $t$ og $h_t$. Derudover kan de anvendes til at afgøre, om en strategi er optimal. 


% \begin{enumerate}
%     \item Løsningerne til ligningerne er de optimale belønninger fra beslutningstidspunktet $t$ til $N$ for ethvert $t$.
%     \item De kan afgøre om en strategi er optimal.
%     \item De bestemmer en effektiv metode til at finde de optimale belønningsfunktioner og strategier.
%     \item De kan bestemme egenskaber for strategier og belønningsfunktioner.
% \end{enumerate}
Disse egenskaber for Bellman ligningerne præsenteres i følgende resultat.

\begin{minipage}\textwidth
\begin{thmx} \textbf{Egenskaber for Bellman ligningerne} \label{sæt:ret_så_vigtig}%Ny sætning
\newline
Lad $u_t$ være en løsning til \eqref{eq:u_t_sup} for $t = 1, 2, \ldots, N-1$ og $u_N$ være givet som i \eqref{eq:u_N}. Lad derudover $u_t^*$ være den maksimale forventede totale belønning. Da gælder det, at
\begin{enumerate}
    \item $u_t(h_t)=u_t^*(h_t)$ for alle $h_t\in H_t$ og for $t=1, 2,\ldots, N$
    \item $u_1(s_1)=v_N^*(s_1)$ for alle $s_1\in S$.
\end{enumerate}
\end{thmx}
\end{minipage}

For at bevise dette introduceres følgende lemma

\begin{minipage}\textwidth
\begin{lem} \label{lem:ret_vigtig}\textbf{} %Nyt lemma
\newline
Lad $G$ være en vilkårlig diskret mængde og $g: G\to \R$. Lad derudover $q$ være en sandsynlighedsfordeling på $G$. Så er
\begin{align*}
    \sup_{z\in G}g(z)\geq \sum_{z\in G}q(z) g(z).
\end{align*}
\end{lem}
\end{minipage}

\begin{bev} \textbf{} %Nyt bevis
\newline
Lad $G$ være en vilkårlig diskret mængde og $g$ være en reel funktion på $G$. Lad derudover $q$ være sandsynlighedsfordelingen på $G$. Såfremt $\displaystyle g^*=\sup_{z\in G}g(z)$, så er 
\begin{align*}
    g^* = \sum_{z \in G} q(z)g^* \geq \sum_{z \in G} q(z)g(z).
\end{align*}
Første lighed er fra \autoref{prop:frekvensfunktion}, og sidste ulighed gælder, da $g^* \geq g(z)$ jævnfør \autoref{sæt_om_eps}. Dermed er \autoref{lem:ret_vigtig} bevist.
\end{bev}

Det er hermed muligt at bevise \autoref{sæt:ret_så_vigtig}.

\begin{bev} \textbf{} %Nyt bevis 
\newline
Lad $u_t$ være en løsning til \eqref{eq:u_t_sup} for $t = 1, 2, \ldots, N-1$ og $u_N$ være givet som i \eqref{eq:u_N}. Lad derudover $u_t^*$ være den maksimale forventede totale belønning. 

\textbf{Bevis for punkt 1}

Først bevises det ved induktion, at $u_n(h_n) \geq u_n^*(h_n)$ for alle $h_n \in H_n$ og $n = 1, 2, \ldots, N$.
Da der ikke tages en beslutning til beslutningstidspunktet  $N$, følger det af \eqref{eq:u_N=r_N} og \eqref{eq:u_N}, at
\begin{align*}
    u_N(h_N) = r_N(s_N) = u_N^\Psi(h_N), \text{ for alle } h_N \in H_N \text{ og } \Psi \in \Pi^{FT}.
\end{align*}
Dermed er induktionsstarten bevist. Lad nu $u_t(h_t) \geq u_t^*(h_t)$ for alle $h_t \in H_t$ for $t=n+1, \ldots, N$, og lad $\Psi' = (d'_1, d'_2, \ldots, d'_{N-1})$ være en vilkårlig strategi i $\Pi^{FT}$. Det skal vises, at dette medfører, at $u_t(h_t) \geq u_t^*(h_t)$ for $t=n$. For $t=n$ er $u_n(h_n)$ givet ud fra $\eqref{eq:u_t_sup}$, altså
\begin{align*}
    u_n(h_n) &=\sup_{a\in A_{s_t}} \left(r_n(s_n,a)+ \sum_{s'\in S} p_n(s'|s_n,a)u_{n+1}(h_n,a,s') \right).
    \intertext{Af induktionshypotesen gælder det, at }
    u_n(h_n)&\geq \sup_{a\in A_{s_t}} \left(r_n(s_n,a)+ \sum_{s'\in S} p_n(s'|s_n,a)u^*_{n+1}(h_n,a,s') \right).
    \intertext{Ud fra definitionen af $u_{n+1}^*$, \eqref{eq:u*}, følger det, at} 
     u_n(h_n)&\geq \sup_{a\in A_{s_t}} \left(r_n(s_n,a)+ \sum_{s'\in S} p_n(s'|s_n,a)u^{\Psi'}_{n+1}(h_n,a,s') \right).
    \intertext{Jævnfør \autoref{lem:ret_vigtig}, er}
     u_n(h_n)&\geq \sum q_{d_n(h_n)}(a)\left( r_n(s_n,a)+\sum_{s'\in S}p_n(s'|s_n,a)u_{n+1}^{\Psi'}(h_n,a,s')\right)\\
    &=u_n^{\Psi'}(h_n).  
\end{align*}
Hvoraf den sidste lighed gælder ud fra \eqref{eq:ligning_i_algoritme1}. Da $\Psi'$ er vilkårlig, følger det, at
\begin{align*}
    u_n(h_n)\geq u_n^\Psi (h_n) \text{ for alle } \Psi\in\Pi^{FT}.
\end{align*}
Dermed er $u_n(h_n) \geq u^*_n(h_n)$.

Herefter vises, at $u^*_n(h_n) \geq u_n(h_n)$. Dette gøres ved først at vise, at der for ethvert $\varepsilon > 0$, eksisterer en strategi, $\Psi' \in \Pi^{FD}$, således, at
\begin{align}\label{eq:4.3.8}
   u_n^{\Psi'}(h_n)+(N-n)\varepsilon\geq u_n(h_n),
\end{align}
for alle $h_n \in H_n$ og $n = 1, 2, \ldots, N$. For at bevise dette konstrueres en strategi, $\Psi' = (d_1, d_2, \ldots, d_{N-1})$, ved at vælge $d_n(h_n)$, der opfylder
\begin{align}\label{eq:bevis_pt_2}
    &r_n\left(s_n,d_n(h_n)\right)+\sum_{s'\in S}p_n\left(s'|s_n, d_n(h_n)\right)u_{n+1}\left(s_n, d_n(h_n), s'\right)+\varepsilon\geq u_n(h_n), 
\end{align}
som er muligt jævnfør \autoref{sæt_om_eps}. 

Ved induktion vises \eqref{eq:4.3.8}. Da der ikke tages en beslutning til beslutningstidspunktet $N$, følger det af \eqref{eq:u_N=r_N} og \eqref{eq:u_N}, at
%
\begin{align*}
    u_N(h_N) = r_N(s_N) = u_N^{\Psi'}(h_N), \text{ for alle } h_N \in H_N .
\end{align*}
Dermed er induktionsstarten bevist. 
Lad $u_t^{\Psi'}(h_t)+(N-t)\varepsilon\geq u_t(h_t)$ for $t=n+1, \ldots, N$.
Det skal vises, at dette medfører, at $u_n^{\Psi'}(h_n)+(N-n)\varepsilon\geq u_n(h_n)$. Ud fra \eqref{eq:uPsi_for_FD} er
\begin{align*}
    u_n^{\Psi'}&=r_n\left(s_n, d_n(h_n)\right)+\sum_{s'\in S}p_n\left(s'| s_n, d_n(h_n)\right)u_{n+1}^{\Psi'}\left(s_n, d_n(h_n), s'\right).
    \intertext{Ved at anvende induktionshypotesen fås}
    u_n^{\Psi'} &\geq r_n\left(s_n,d_n(h_n)\right) + \sum_{s' \in S} p_n\left(s' | s_n, d_n(h_n)\right)u_{n+1}\left(s_n, d_n(h_n),s'\right) - (N-n-1)\varepsilon.
    \intertext{Af \eqref{eq:bevis_pt_2} følger det, at}
    u_n^{\Psi'}&\geq u_n(h_n) - (N-n)\varepsilon.
\end{align*}
Hermed er det bevist, at $u_n^{\Psi'}(h_n)+(N-n)\varepsilon\geq u_n(h_n)$ for $n = 1, 2, \ldots, N$. 
Derfor gælder det for ethvert $\varepsilon>0$, at der eksisterer en strategi, $\Psi'\in\Pi^{FT}$, således, at
\begin{align*}
    &u_n^*(h_n) + (N-n)\varepsilon \geq u_n^{\Psi'}(h_n) + (N-n)\varepsilon \geq u_n(h_n) %\geq u_n^*(h_n)
\end{align*}
Der eksisterer dermed en strategi, hvorom det gælder, at for et tilpas lille $\varepsilon > 0$, så er $u_n^*(h_n) \geq u_n(h_n)$.  
Da det nu er bevist, at $u_n^*(h_n) \geq u_n(h_n)$ og $ u_n(h_n) \geq u_n^*(h_n)$, må $u_n^*(h_n) = u_n(h_n)$.

\textbf{Bevis for punkt 2}\\
Andet punkt bevises ved følgende ligning
\begin{align} \label{bevis_for_punkt_2}
    u_1(s_1)=u_1^*(s_1)=\sup_{\Psi\in \Pi^{FT}}u_1^\Psi(s_1)=\sup_{\Psi\in \Pi^{FT}}v_N^\Psi(s_1)=v_N^*(s_1).
\end{align}
Den første lighed i \eqref{bevis_for_punkt_2}, følger af punkt 1 i \autoref{sæt:ret_så_vigtig} og den anden lighed af \eqref{eq:u*}. Derudover gælder tredje lighed af \autoref{sæt:den_gælder} og fjerde lighed jævnfør \eqref{eq:v_N=sup}.

Hermed er \autoref{sæt:ret_så_vigtig} bevist.
\end{bev}

Af \autoref{sæt:ret_så_vigtig} punkt 1 følger det, at løsningerne til Bellman ligningerne er den maksimale forventede totale belønning, $u_t^*$, for beslutningstidspunkterne $t, t+1, \ldots, N$.
Derudover gælder det fra punkt 2, at løsningerne til Bellman ligningerne for $n=1$ er den optimale belønningsfunktion, $v_N^*$, for alle beslutningstidspunkter. % $1, 2,\ldots, N$.

Bellman ligningerne kan anvendes til at bestemme den optimale strategi, $\Psi^*$, og bekræfte, at en strategi er optimal. Hvis der eksisterer en optimal strategi, gælder følgende resultat.
\begin{minipage}\textwidth
\begin{thmx}\label{sæt:optimal_strategi_ved_Bellman} \textbf{Optimal strategi ud fra Bellman ligningerne} %Ny sætning
\newline
Lad $u_t^*$ være en løsning til Bellman ligningerne, \eqref{eq:u_N} og \eqref{eq:u_t_max}, for $t=1,2, \ldots, N$. Lad derudover $\Psi^*=(d_1^*, d_2^*, \ldots, d_{N-1}^*)\in \Pi^{FD}$ være en strategi, der opfylder, at
\begin{align}\label{eq:optimal_strategi}
    &r_t\left(s_t, d_t^*(h_t)\right)+\sum_{s'\in S}p_t\left(s'|s_t, d_t^*(h_t)\right)u_{t+1}^*\left(h_t, d_t^*(h_t), s'\right)\nonumber\\
    &=\max_{a\in A_{s_t}}\left(r_t(s_t,a)+\sum_{s'\in S}p_t(s'|s_t, a)u_{t+1}^*(h_t, a, s')\right)
\end{align}
for $t=1, 2, \ldots, N-1$.
Så gælder det, at
\begin{enumerate}
    \item for hvert $t = 1, 2, \ldots, N$ er
    \begin{align*}
        u_t^{\Psi^*}(h_t) = u_t^*(h_t), \text{ for } h_t \in H_t.
    \end{align*}
    \item $\Psi^*$ er en optimal strategi, og 
    \begin{align*}
      v_N^{\Psi^*}(s)  = v_N^*(s), \text{ for } s \in S.
    \end{align*}
\end{enumerate}
\end{thmx}
\end{minipage}

\begin{bev} \textbf{} %Nyt bevis
\newline
Lad $u_t^*$ være en løsning til Bellman ligningerne, \eqref{eq:u_N} og \eqref{eq:u_t_max}, for $t=1, 2, \ldots, N$. Lad derudover $\Psi^*=(d_1^*, d_2^*, \ldots, d_{N-1}^*)\in \Pi^{FD}$ være en strategi.

\textbf{Bevis for punkt 1}\\
Dette bevises ved induktion. Af \autoref{sæt:ret_så_vigtig}, \eqref{eq:u_N} og \eqref{eq:u_N=r_N} følger det, at
\begin{align*}
    u_N^*(h_N)=u_N(h_N)=r_N(s) = u^{\Psi^*}_N(h_N), \text{ for } h_N\in H_N.
\end{align*}
Dermed er induktionsstarten bevist. Lad nu $u_t^*(h_t) = u_t^{\Psi^*}(h_t)$ for $t = n+1, \ldots, N$. Det skal vises, at dette medfører, at $u_t^*(h_t) = u_t^{\Psi^*}(h_t)$ for $t=n$. Af \eqref{eq:u_t_max} er
\begin{align*}
    u_n^*(h_n)&=\max_{a\in A_{s_n}}\left(r_n(s_n, a)+\sum_{s'\in S}p_n(s'|s_n, a)u_{n+1}^*(h_n, a, j)\right).
    \intertext{Fra \eqref{eq:optimal_strategi} fås, at}
    u_n^*(h_n)&=r_n\left(s_n, d_n^*(h_n)\right) + \sum_{s' \in S} p_n\left(s'|s_n, d_n^*(h_n)\right)u_{n+1}^{*}\left(h_n, d_n^*(h_n),s'\right).
    \intertext{Ved at anvende induktionshypotesen fås}
    u_n^*(h_n)&=r_n\left(s_n, d_n^*(h_n)\right) + \sum_{s' \in S} p_n\left(s'|s_n, d_n^*(h_n)\right)u_{n+1}^{\Psi^*}\left(h_n, d_n^*(h_n),s'\right)=u_n^{\Psi^*}(h_n)
\end{align*}
for $h_n=\left(h_{n-1}, d_{n-1}^*(h_{n-1}\right), s_n)$. Dermed er det bevist, at $u_n^*(h_n) = u_n^{\Psi^*}(h_n)$.

\textbf{Bevis for punkt 2}\\
Det følger af \autoref{sæt:den_gælder} og \autoref{sæt:ret_så_vigtig} punkt 2, at 
\begin{align*}
    v_N^{\Psi^*}(s)=v_N^*(s) \text{ for } s\in S,
\end{align*}
og derfor er $\Psi^*$ en optimal strategi.

Dermed er \autoref{sæt:optimal_strategi_ved_Bellman} bevist.
\end{bev}

Fra \autoref{sæt:optimal_strategi_ved_Bellman} gælder det dermed, at den optimale strategi bestemmes ved først at løse Bellman ligningerne. Dernæst vælges en beslutningsregel til enhver historik, der bestemmer den beslutning, der medfører, at højresiden af \eqref{eq:optimal_strategi} for $t=1, 2, \ldots, N$ er maksimal.

I \autoref{sæt:optimal_strategi_ved_Bellman} er strategien fortidsafhængig deterministisk. Hvis der eksisterer en fortidsafhængig tilfældig strategi, der opfylder det generaliserede udtryk for \eqref{eq:optimal_strategi}, så af \autoref{lem:ret_vigtig} eksisterer der også en fortidsafhængig deterministisk, der opfylder \eqref{eq:optimal_strategi}. Dermed er det kun nødvendigt at vise \autoref{sæt:optimal_strategi_ved_Bellman} for fortidsafhængige deterministiske strategier. 


%Dette gælder, da lemmaet resulterer i, at en sådan strategi kan bestemmes selvom strategien havde været fortidsafhæng tilfældig.
%Hvis der eksisterer en fortidsafhængig tilfældig strategi der opfylder ... vil der fra lemma kunne bestemmes en deterministisk strategi der opfylder...
%Punkt 1 i \autoref{sæt:optimal_strategi_ved_Bellman} kaldes også for \textit{Optimalitets-princippet}.

% Den optimale strategi, $\Psi^{\ast}$, blev defineret ved \eqref{eq:optimal_strategi}, hvor max resulterer i en reel værdi. \eqref{eq:optimal_strategi} kan udtrykkes som følgende
% \begin{align*}
%     d_t^{\ast}(h_t)=\arg\max_{a\in A_{s_t}}\left\{r_t(s_t, a)+\sum_{j\in S_{t}}p_t(j,s_t, a)u_{t+1}^{\ast}(h_t, a, j)\right\},
% \end{align*}
% hvor argmax resulterer i en mængde, mens max resulterer i en reel værdi. I ovenstående er det udtrykket for den optimale beslutningsregel, der er givet i stedet for den maksimale forventede totale belønning.



Hvis det ikke er muligt at bestemme supremum i \eqref{eq:u_t_sup}, kan beslutningstageren vælge den strategi, der er $\varepsilon$-optimal. I dette projekt er dette ikke relevant til problemløsningen, men da resultatet anvendes fremadrettet, præsenteres resultatet i \autoref{Snyd}.

I beviset til \autoref{sæt:ret_så_vigtig} blev en fortidsafhængig deterministisk $\varepsilon$-optimal strategi konstrueret, mens der i \autoref{sæt:optimal_strategi_ved_Bellman} og \autoref{sæt:epsopt} blev bestemt, hvorvidt en strategi er optimal og $\varepsilon$-optimal. Det er dermed bestemt, at der for ethvert $\varepsilon> 0$ eksisterer en $\varepsilon$-optimal strategi, der er fortidsafhængig deterministisk, og at enhver strategi i $\Pi^{FD}$ der opfylder \eqref{eq:bilag_epsilon_optimal}, er $\varepsilon$-optimal. Derudover er det bestemt, at såfremt $u_t^*$ er en løsning til \eqref{eq:u_t_sup} og \eqref{eq:u_N}, og der for ethvert $t$ og $s_t\in S$ eksisterer et $a^{\circ}\in A_{s_t}$, hvorom det gælder, at
%
\begin{align}\label{eq:deterministisk_fortidsafhængig_sup}
    &r_t(s_t,a^{\circ}) + \sum_{s'\in S}p_t(s' | s_t,a^{\circ})u^*_{t+1}(h_t, a^{\circ},s')\nonumber \\
    &= \sup_{a \in A_{s_t}}\left(r_t(s_t,a) +  \sum_{s'\in S}p_t(s' | s_t,a)u^*_{t+1}(h_t, a,s')\right)
\end{align}
for alle $h_t=(s_{t-1}, a_{t-1}, s_t)$, så eksisterer der en optimal deterministisk fortidsafhængig strategi, $\Psi^*\in \Pi^{FD}$.

I følgende sætning vises det, at der eksisterer en optimal strategi, som er Markov deterministisk.

\begin{thmx}\label{sæt:deterministisk_Markov_optimal_strategi} \textbf{Eksistens af Markov deterministisk optimal strategi} %Ny sætning
\newline
Lad $u_t^*$ være løsninger til Bellman ligningerne \eqref{eq:u_t_sup} og \eqref{eq:u_N} for $t=1, 2, \ldots, N$. Så
\begin{enumerate}
    \item afhænger $u_t^*(h_t)$ kun af $h_t$ gennem $s_t$ for ethvert $t=1, 2, \ldots, N$.
    \item eksisterer der en Markov deterministisk $\varepsilon$-optimal strategi for alle $\varepsilon>0$.
    \item eksisterer der en Markov deterministisk optimal strategi, såfremt der eksisterer et $a^\circ\in A_{s_t}$ således, at \eqref{eq:deterministisk_fortidsafhængig_sup} er opfyldt for ethvert $s_t\in S$ og $t=1, 2,\ldots, N-1$.
\end{enumerate}
\end{thmx}

\begin{bev} \textbf{} %Nyt bevis
\newline
Lad $u_t^*$ være løsninger til Bellman ligningerne \eqref{eq:u_t_sup} og \eqref{eq:u_N} for $t=1, 2,\ldots, N$.

\textbf{Bevis for punkt 1}\\
Dette bevises ved induktion. Ud fra \eqref{eq:u_N} følger det, at $u^*_N(h_N) = u^*_N(h_{N-1}, a_{N-1}, s) =r_N(s_N)$ for alle $h_{N-1} \in H_{N-1}$ og $a_{N-1} \in A_{s_{N-1}}$. Dermed gælder det, at $u^*_N(h_N) = u^*_N(s_N)$. Altså er induktionsstarten bevist. Lad nu punkt 1 være sand for $t = n+1, \ldots, N$. Det skal vises, at dette medfører, at punkt 1 er sand for $t = n$. 
Der gælder, at
\begin{align*}
    u_n^*(h_n)=\sup_{a\in A_{s_t}}\left(r_n(s_n, a)+\sum_{s'\in S}p_t\left(s'|s_n, a\right)u_{n+1}^*(h_n, a, s')\right).
    \intertext{Af induktionshypotesen følger det, at}
    u_n^*(h_n)=\sup_{a\in A_{s_t}}\left(r_n(s_n, a)+\sum_{s'\in S}p_n\left(s'|s_n, a\right)u_{n+1}^*(s')\right).
\end{align*}
Da højresiden kun afhænger af $h_n$ gennem $s_n$, er punkt 1 bevist.

\textbf{Bevis for punkt 2}\\
Lad $\varepsilon > 0$ og lad $\Psi^\varepsilon = (d_1^\varepsilon, d_2^\varepsilon, \ldots, d_{N-1}^\varepsilon)$ være en vilkårlig strategi i $\Pi^{MD}$ som opfylder, at
%
\begin{align*}
    &r_t\left(s_t, d_t^\varepsilon(s_t)\right)+\sum_{s'\in S}p_t\left(s'|s_t, d_t^\varepsilon(s_t)\right)u_{t+1}^*\left(s'\right) + \frac{\varepsilon}{N-1}\nonumber\\
    &\geq \sup_{a\in A_{s_t}}\left(r_t(s_t,a)+\sum_{s'\in S}p_t(s'|s_t, a)u_{t+1}^*( s')\right),
\end{align*}
som eksisterer jævnfør \autoref{sæt_om_eps}. Af punkt 1 er $u_t^*(h_t)$ kun afhængig af $h_t$ gennem $s_t$, dermed gælder det fra \autoref{sæt:epsopt}, at strategien, $\Psi^\varepsilon $, er $\varepsilon$-optimal.

\textbf{Bevis for punkt 3}\\
Antag, at der eksisterer et $a^\circ \in A_{s_t}$ for ethvert $t$ og $s_t$ således, at der eksisterer et $\Psi^* = (d_1^*, d_2^*, \ldots, d_{N-1}^*) \in \Pi^{MD}$, som opfylder \eqref{eq:deterministisk_fortidsafhængig_sup}. 

Eftersom supremum er antaget ved $a^{\circ}$ for ethvert $t$ og $s_t$, så er det muligt at konstruere en strategi, hvorom der gælder, at $d_t(s_t)=a^{circ}$ for alle $t$ og $s_t$. Derfor eksisterer der en strategi, således at 
%
%Dermed eksisterer supremum og altså gælder det, at $\Psi^*$ opfylder
\begin{align*}
    r_t\left(s_t,d_t^*(s_t)\right) + \sum_{s'\in S}p_t\left(s' | s_t, d_t^*(s_t)\right)u^*_{t+1}(s')\nonumber 
    = \max_{a \in A_{s_t}}\left(r_t(s_t,a) +  \sum_{s'\in S}p_t(s' | s_t,a)u^*_{t+1}(s')\right).
\end{align*}

Af punkt 1 er $u_t^*(h_t)$ kun afhængig af $h_t$ gennem $s_t$, og dermed følger det af \autoref{sæt:optimal_strategi_ved_Bellman}, at $\Psi^*$ er den optimale strategi.
Dermed er \autoref{sæt:deterministisk_Markov_optimal_strategi} bevist.
\end{bev}

Det er dermed vist at
\begin{align*}
    v_N^*(s)=\sup_{\Psi\in\Pi^{FT}}v_N^\Psi(s)=\sup_{\Psi\in\Pi^{MD}}v_N^\Psi(s) \text{ for } s\in S.
\end{align*}

Altså gælder det for en Markov beslutningsproces, at den maksimale forventede totale belønning har samme værdi for en Markov deterministisk strategi som en fortidsafhængig tilfældig strategi. Dermed er det kun nødvendigt at betragte Markov deterministiske strategier for at bestemme den maksimale forventede totale belønning. 

%Følgende proposition kan anvendes til at bestemme om der eksisterer en Markov deterministisk strategi der er optimal. 
For at bestemme hvorvidt en Markov deterministisk strategi er optimal, anvendes følgende proposition. 

\begin{minipage}\textwidth
\begin{pro} \textbf{}\label{prop:markov_det_strategi} %Ny proposition
\newline
Lad tilstandsrummet, $S$, være tællelig og lad beslutningsrummet, $A_s$, være endeligt for ethvert $s\in S$. Så eksisterer der en optimal Markov deterministisk strategi, $\Psi^*\in \Pi^{MD}$.  
\end{pro}
\end{minipage}

\begin{bev} \textbf{} %Nyt bevisw
\newline
Det er tilstrækkeligt at vise, at der for ethvert $t$ og $s_t$ findes et $a\in A_{s,t}$, så
\begin{align*}
    \sup_{a\in A_{s_t}} r_t(s_t,a) + \sum_{s'\in S}p_t\left(s' | s_t, a\right)u^*_{t+1}(s')\nonumber \\
    = \max_{a\in A_{s_t}} r_t(s_t,a) + \sum_{s'\in S}p_t\left(s' | s_t, a\right)u^*_{t+1}(s').
\end{align*}
Da $A_s$ er endelig, må der eksistere et $a$ således, at dette er opfyldt, og af \autoref{sæt:optimal_strategi_ved_Bellman} eksisterer der derfor en optimal Markov deterministisk strategi.

% Det skal vises, at der eksisterer et $a^\circ$, som opfylder \eqref{eq:deterministisk_fortidsafhængig_sup}. Af \autoref{sæt:optimal_strategi_ved_Bellman} gælder det da, at der eksisterer en optimal Markov deterministisk strategi.  
% Jævnfør \autoref{sæt:deterministisk_Markov_optimal_strategi} punkt 3 skal der for ethvert $s\in S$ eksistere et $a^\circ$, der opfylder
% \begin{align*}
%     r_t(s_t,a^\circ) + \sum_{s'\in S}p_t\left(s' | s_t, a^\circ\right)u^*_{t+1}(s')\nonumber 
%     = \sup_{a \in A_{s_t}}\left(r_t(s_t,a) +  \sum_{s'\in S}p_t\left(s' | s_t,a\right)u^*_{t+1}(s')\right).
% \end{align*}
% Da $A_s$ er endelig, må der eksistere et $a^\circ$ således, at dette er opfyldt. Dermed er \autoref{prop:markov_det_strategi} bevist.
\end{bev}






