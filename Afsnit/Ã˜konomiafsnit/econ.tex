















































\iffalse
%- Renteteori?



%- Forbrugs- og investeringsvaner?
%(investment and consumption)
%   - Investment: Future gain
%    - Consumption: Use up, short gain (fysisk)
%    - 

%- Optimering af opsparing?


%- Forbrug som funktion af opsparing og indtjening?


%- Offentlige ift. private investeringer?


%- økonomisk vækst?



%Samlet forbrug stiger i form af, at den disponible indkomst stiger, og omvendt. Men den stiger og falder ikke i ligeså stor omfang
  
Hvordan beslutter forbrugeren, hvordan og hvor meget, der skal forbruges? Det er klart, at der med et højere rådighedsbeløb kan forbruges mere. Det er endvidere sådan, at når indkomsten er højere, end hvad der forbruges, så vil der automatisk dannes et overskud på kontoen - i form af en opsparing.
Lad altså $A_t$ være et forbrug og $S_t$ være en opsparing til tiden t.
Lad nu $I$ være den disponible indkomst.
Det er dermed klart for en fast månedligt indkomst, $I$, og et konstant forbrug $A$, at der vil indtræffe følgende tilfælde
\begin{itemize}
    \item For $A > I : S \to - $
    \item For $A = I : S \to 0$
    \item For $A < I : S \to +$
\end{itemize}

Indkomst- og forbrugstilbøjeligheder er dog mere komplekst end dette. Indkomsten vil i fleste tilfælde varierer, og det samme vil forbruget. Antag eksempelvis hvordan den forventede indkomst, $E[I]$, kan varierer over tid...


For at beskrives indkomsten og forbrug nærmere, er det muligt at danne følgende relation
\begin{align*}
A = a+bI,
\end{align*}
hvor a beskriver det autonome forbrug, og b den marginale forbrugstilbøjeligheden. En højere indkomst vil således give et øget forbrug relativt med en stigning i den marginale forbrugstilbøjelighed. Det er dog således, at indkomsten og forbruget ikke vil ændre sig i lige store omfang. 

\fi