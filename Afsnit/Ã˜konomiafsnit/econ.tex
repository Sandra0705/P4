

%kilde https://www.jstor.org/stable/2938366?seq=1#metadata_info_tab_contents
Dette afsnit omhandler opsparingsteori, hvor der vil blive redegjort for 

%når forbrugeren ikke har muligheden for at låne. Heriblandt de forbrugs- og opsparingstilbøjeligheder, som investoren har.

Når en investor er mere tilbøjelig til at forbruge nu end senere og har en ligelig fordelt, uafhængig indkomst, vil renter og aktiver motivere gode forbrugsvalg. 
Det, at en investorer ikke har lånemuligheder, vil sågar være et motiv for, om investoren skal forbruge eller spare op. - Investoren foretrækker forbrug nu fremfor forbrug senere.
Når indkomsten er en "random walk", viser det sig, at dem der ønsker at låne, men ikke kan, er mere tilbøjlige til at forbruge hele deres indkomst.

\iffalse
\subsection*{Modellering}
Forbrugeren ønsker at maksimere nyttefunktionen givet ved
\begin{align}
    u = E_t\big(\sum_{\tau=t}^t (1+\delta)^{t-\tau}v(A_\tau)\big)
\end{align}
Hvor $\delta > 0$ er satsen af tidspreferencen, og $v(c_t)$ er den øjeblikkelige nyttefunktion, der er stigende, konkav og differentiabel.
Det antages, at forbrugeren har aktiver, der påvirker økonomien. Udviklingen af aktiver er givet ved
\begin{align}
    K_{t+1}=(1+r)(K_t+I_t-A_t)
\end{align}
hvor $I_t$ er indkomsten, $K_t$ er det reelle aktiv, og $r$ er rentesatsen. Denne rentesats er en kendt konstant, og der tages udgangspunkt i at fokusere på indkomsten, $I_t$. 
Lad desuden $K_t \geq 0$, således at der er en begrænsning på at optage lån.\\
Den øjeblikkelige marginale nytte for penge beskrives ved
$\lambda(A_t)=v'(A_t)$
hvor $\lambda(\cdot)$ er en positiv monoton og aftagende funktion. Fremadrettet antages denne værdi $\lambda(\cdot)$ til at være konveks. 



\subsubsection{Utålmodig forbruger}
Antag en utålmodig forbruger med $\delta > r$. Og lad $S_t$ betegne den samlede beholdning af penge, givet ved
\begin{align}
    S_t = K_t + I_t
\end{align}
hvor $S_t$ er det maksimale, der kan forbruges til perioden $t$. Forbruget til tiden $t$ og $t+1$ må overholde
\begin{align}
    \lambda(A_t) = max\Big [\lambda(S_t), \ \ \beta E_t \lambda(A_{t+1})\Big]
\end{align}
hvor $\beta=(1+r)/(1+\delta)$, og $\beta$ < 1, eftersom det er antaget, at $r < \delta$.

Hvis forbrugeren er begrænset, så vil forbruget ikke kunne overstige $S_t$, og den marginale nytte kan ikke være lavere end $\lambda(S_t)$. 
Begrænsningen binder sig, hvis den marginale nytte til $S_t$ er højere end diskonteringen til den forventede marginale nytte i den næste periode. Ellers er de to marginale nytter beskrevet på sædvanligvis.
Desuden fremgår det, at forventningen tager højde for muligheden for, at der er fremtidige begrænsninger.
Givet ligning (5.2), fås
\begin{align}
    S_{t+1}=(1+r)(S_t-A_t)+I_{t+1}
\end{align}
Det er muligt at beskrive dette ved en stokastisk proces, hvor forbruget er en funktion til tilstandsvariablen, $S_t$, lad $A_t=f(S_t)$. Lad desuden den marginale nytte af penge være givet ved $p(S_t)$. Så er 
\begin{align}
    p(S_t)=\lambda[f(S_t)],\ \ \intertext{eller} \ \ A_t=f(S_t)=\lambda^{-1}[p(S_t)].
\end{align}
Hvis der eksisterer en stationær løsning $p(S)$ med $f(S)$, må der gælde, at
\begin{align}
    p(S)= max\Big[\lambda(S), \beta \int p \Big( (1+r)(S-\lambda^{-1}p(S))+I\Big) dF(I)\Big]. 
\end{align}
Dette er beskrevet fra (5.4) ved at indsætte (5.5) og (5.6, 5.7). Her fremgår den marginale nytte i dag og den maksimale værdi af den marginale nytte, når den er begrænset, samt diskonteringens forventede værdi af den næste dag. Hvis ligning (5.8) har en løsning, er det muligt at karakterisere ligevægten af den marginale nytte af penge, og dermed strategien for funktionen $f(S)$.

Det er muligt at lade dette problem gå i en endelig tidshorisont. \\
Lad $p_0(S), p_1(S), \ldots, p_n(S)$ være en række af funktioner, hvor $p_0(S)=\lambda(S)$, og derved fås
\begin{align}
    p_n(S)=max\Big[\lambda(S),\beta\int p_{n-1}\Big( (1+r)(S-\lambda^{-1} p_n(S))+I\Big)dF(I)\Big]
\end{align}

Denne rekursion kan tænkes som værende en omvendt/baglæns løsning til et endeligt stokastisk dynamisk system.\\
I den sidste periode, $N=0$ vil alt være brugt, og den marginale nytte af penge, $p_0(S)$, er $\lambda(S),$ da ligemeget hvilken værdi $S$ har, så vil der forbruges. Da der itereres baglæns, vil funktionen i nogle tilfælde være konvergerende, hvilket vil medføre, at der findes en løsning til (5.9), og den uendelige tidshorisont.


\fi





\section{Utålmodig forbruger}
Antag en utålmodig forbruger, der foretrækker at forbruge nu fremfor at lade penge stå i en opsparing. Den utålmodige forbruger tager ikke højde for fremtidige belønninger ved at spare op.
Hvis forbrugeren ikke er i arbejde, vil der dog ikke være betydning, om der bliver forbruget. Det er dog sådan, at en utålmodig ikke vil være 
Antag nu, at investoren modtager en indlånsrente $>0\%$. D



\section{Tålmodig/fornuftig forbruger}







\subsection{Brugen af kvadratrod}
Det antages, at en investorer modtager en 0\% indlånsrente. Det er derfor sådan, at investorer ikke modtager penge for at have dem til at stå i en opsparing i banken. Den optimale strategi vil dog ikke nødvendigvis være at forbruge alt i hver måned. 
Antag, at nytten er givet ved at tage $\sqrt{.}$. Det er der derfor klart, at nytten maksimeres ved at forbruge alt til den sidste tidsperiode.