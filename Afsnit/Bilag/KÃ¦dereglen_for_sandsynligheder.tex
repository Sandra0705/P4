\section{Kædereglen for sandsynligheder}\label{bilag:kædereglen}
Dette er baseret på \cite{"wiki"}.

\begin{minipage}\textwidth
\begin{thmx} \textbf{Kædereglen for sandsynligheder} \label{sæt:kæde}%Ny sætning
\newline
Lad $X_1, X_2, \cdots, X_n$ være diskrete tilfældige variabler og $(\Omega, \F, P)$ være et sandsynlighedsrum. Så gælder det, at 
\begin{align*}
    P\left(\bigcap_{k=1}^n X_k\right)=\prod_{k=1}^n P\left(X_k\bigg|\bigcap_{j=1}^{k-1} X_j\right).
\end{align*}
\end{thmx}
\end{minipage}

\begin{bev} \textbf{} %Nyt bevis
\newline
Lad $X_1, X_2, \cdots, X_n$ være diskrete tilfældige variabler og $(\Omega, \F, P)$ være et sandsynlighedsrum.
Dette bevises ved induktion. Induktionsstarten $n=2$ er opfyldt per \autoref{def:betinget_sandsynlighed}, da
\begin{align*}
    P(X_2 | X_1) = \frac{P(X_2 \cap X_1)}{P(X_1)} \Leftrightarrow P(X_2 \cap X_1) = P(X_1)P(X_2 | X_1)
\end{align*}
Det antages nu, at det er sandt for $n = n-1$, altså at 
\begin{align*}
    P(X_1 \cap \dots \cap X_{n-1}) = P(X_1)P(X_2|X_1)P(X_3|X_2 \cap X_1) \cdots P(X_{n-1} | X_{n-2} \cap \dots \cap X_1),
\end{align*}
og det skal vises, at dette medfører, at $n=n$ er sand. Lad $\{X_1, X_2, \dots, X_{n-1}\} = B$, dermed gælder det af \autoref{def:betinget_sandsynlighed} og induktionshypotesen, at
\begin{align*}
    P(X_1 \cap \dots \cap X_{n-1} \cap X_n) &= P(B \cap X_n) = P(B)P(X_n | B)\\
    &= P(X_1)P(X_2|X_1)P(X_3|X_2 \cap X_1) \\
    &\phantom{= \ } \cdots P(X_{n-1} | X_{n-2} \cap \dots \cap X_1)P(X_n | X_1, X_2, \dots, X_{n-1})
\end{align*}
Dermed er induktionsskridtet vist. Altså er \autoref{sæt:kæde} bevist.
\end{bev}
Når denne sætning anvendes i projektet vil $\cap$ erstattes med komma.