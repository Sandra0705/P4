\section{}

\cite{"ETP"}

% \begin{thmx} \textbf{Mindste øvre grænse} \label{sæt_om_eps} %Ny sætning
% \newline
%     Lad $A\subseteq \R$ og lad $b$ være en øvre grænse for $A$. Så er følgende udsagn ensbetydende:
%     \begin{enumerate}
%         \item $b$ er den mindste øvre grænse for $A$.
%         \item $\forall c \in \R : ( c\text{ er en øvre grænse for}A ) \Rightarrow b\leq c$.
%         \item $\forall c \in \R : c < b \Rightarrow  (\text{c er ikke en øvre grænse for $A$}). $
%         \item $\forall c \in \R : c < b \Rightarrow (\exists a \in A : a > c)$.
%         \item $\forall \varepsilon > 0 \exists a \in A : a > b - \varepsilon$.
%     \end{enumerate}
% \end{thmx}

\begin{thmx} \textbf{Supremum} \label{sæt_om_eps} %Ny sætning
\newline
Lad $A$ være en delmængde af $\R$. Et reelt tal $b$ er  supremum for $A$,  hvis og kun hvis begge betingelser
\begin{align*}
    \forall a \in A &: a \leq b \quad \text{og} \\
    \forall \varepsilon > 0 \exists a \in A &: a > b - \varepsilon
\end{align*}
er opfyldt.
\end{thmx}