For at besvare problemformuleringen om hvordan det er muligt at bestemme det optimale forbrug og den maksimale belønning for en forbruger over en given tidsperiode, er forbrugsproblemet blevet opstillet som en Markov beslutningsproces. Herudover er forbrugerens arbejdsstatus beskrevet som en Markov-kæde, der indgår i denne beslutningsproces. For at kunne beskrive denne Markov beslutningsproces og Markov-kæde er teori om sandsynlighedsregning, Markov-kæder og Markov beslutningsprocesser blevet præsenteret. Ud fra denne teori er det bestemt, at forbrugerproblemet kan løses ved brug af baglæns induktion i form af \autoref{Algoritme1}, og Bellman-ligningen, \autoref{eq:u_t_max}. For at bestemme det optimale forbrug og den maksimale belønning er disse blevet implementeret i Python. Herudover er det blevet analyseret, hvordan varierende sandsynligheder for at være i arbejde, samt hvordan varierende renter og lån vil påvirke forbruget og belønningen. 

Det kan ud fra basis-problemet konkluderes, at sandsynligheden for at forbrugeren er arbejdsløs, påvirker hans forbrug. Jo større sandsynligheden er, for at forbrugeren bliver arbejdsløs, jo mere konservativ er hans forbrug over tidsperioden. Dette skyldes, at forbrugeren vil holde et konstant forbrug også til de beslutningstidspunkter, hvor han er arbejdsløs. Hvis sandsynligheden er større for, at forbrugeren er arbejdsløs, vil han gennemsnitligt være i arbejde mindre og dermed også få løn til færre beslutningstidspunkter. Dermed har forbrugeren mindre at forbruge over hele tidsperioden og forbruger derfor mindre til hvert beslutningtidspunkt. Dette forudsager også, at den maksimale belønning er lavere, jo større sandsynligheden for arbejdsløshed er.

Fra basis-problemet kan det desuden konkluderes, at den fornuftige forbruger vil hæve sit månedlige forbrug, når renten stiger, og det samlede forbrug vil dermed være større.

Ud fra det udvidede problem kan det konkluderes, at forholdet mellem indlåns- og udlånsrenten påvirker forbrugerens lånetendens og forbrug. Når indlåns- og udlånsrenten er ækvivalente, er belønningen uafhængig af lån, dog er der enkelte beslutningstidspunkter, hvor forbrugeren har en præference til at låne. Denne præference kan skyldes, at diskretiseringsfaktoren er for høj i det udvidede problem. I tilfældet hvor udlånsrenten er større end indlånsrenten, konkluderes det, at den optimale strategi er uafhængig af udlånsrenten, da forbrugeren ikke får yderligere belønning af at låne. Når indlånsrenten er større end udlånsrenten, gælder det derimod, at forbrugeren vil låne så længe afkastet af opsparingen kan kompensere for afdraget af lånet. 

Ved variation af $\alpha$ kan det ligesom i basis-problemet konkluderes, at det samlede forbrug falder, og dermed også den maksimale belønning, når sandsynligheden for at blive arbejdsløs stiger. 
%Noget godt når vi har snakket med Toke.

I tilfældet hvor lånebegrænsningen øges, konkluderes det, at forbrugeren udnytter renters rente mere. Dette skyldes, at han har mulighed for at forbruge mindre af sin opsparing, hvorved han kan udnytte renters rente uden at sænke sit forbrug. Altså kan forbrugeren hæve sit samlede forbrug, når lånebegrænsningen stiger. 


%Derudover konkluderes det, at en højere lånebegrænsning og en højere indlånsrente medfører, at forbrugeren bliver mindre konservativ omkring hans forbrug, hvorved det er fordelt mindre ligeligt.
