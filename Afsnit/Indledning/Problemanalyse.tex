\section{Problemanalyse}

Alle mennesker, som køber og forbruger varer, kan klassificeres som forbrugere. Enhver forbruger har en opsparing og en indkomst, som udgør et muligt forbrug. En forbruger har et basisforbrug, hvilket er det nødvendige forbrug, herunder vand, varme og husleje. Derudover har forbrugere et varierende forbrug, som er den del af forbruget, der bruges på eksempelvis tøj og møbler. Det varierende forbrug afhænger af den marginale forbrugstilbøjelighed, som bestemmer hvor meget af den disponible indkomst, der forbruges. Da basisforbruget er et nødvendigt forbrug, vil den marginale forbrugstilbøjelighed ikke påvirke dette.

Ved at købe varer og ydelser får forbrugeren en tilfredshed, som måles i nytte. Forbrugere er interesseret i at få så meget tilfredshed som muligt. Derfor skal man som forbruger beslutte, hvor meget der skal spares op, og hvor meget der skal forbruges til et givet tidspunkt for at maksimere sin nytte. Disse beslutninger afhænger blandt andet af forbrugerens nuværende økonomi samt forventningerne til den fremtidige økonomi. Forbrugerens økonomi kan blandt andet blive påvirket af arbejdsstatus, løn, pandemier og finanskriser. Disse faktorer påvirker det varierende forbrug, da den disponible indkomst og den marginale forbrugstilbøjelighed ændres. 

Forbrugere er forskellige, hvilket resulterer i, at hver forbruger har deres egen strategi  til at opnå mest mulig nytte. Nogle forbrugere får mere nytte af at forbruge så meget som muligt i nuet og er derfor ligeglade med indlånsrenten. Andre forbrugere er indifferente i forhold til, om der forbruges nu fremfor senere. Denne type forbruger maksimerer sin nytte ved at have et højt samlet forbrug. Ved en positiv indlånsrente vil denne forbruger dermed maksimere sin nytte ved at forbruge alt til sidst. Typisk er forbrugere mere påpasselige med deres forbrug. Forbrugere får derfor ofte mest nytte af at fordele forbruget ud, så der kan forbruges i hver måned. De ovenstående forbrugstilbøjeligheder kaldes henholdsvis utålmodig, tålmodig og fornuftig, og vil uddybes senere i projektet. 
\pagebreak
\section{Problemafgrænsing}
Problemet hvor en forbruger skal tage beslutninger over en given tidsperiode for at maksimere sin nytte, kaldes forbrugerproblemet, og kan opstilles som en Markov beslutningsproces.

Denne Markov beslutningsproces er en beslutningsmodel, hvor forbrugeren til hvert beslutningstidspunkt skal vælge, hvor meget der skal forbruges. Beløbet som ikke forbruges spares op og kan dermed forbruges til næste beslutningstidspunkt. Forbrugeren skal til ethvert beslutningstidspunkt vælge det forbrug, som maksimerer nytten over hele tidsperioden. %Dette forbrug kan bestemmes ved at anvende baglæns induktion heriblandt Bellman ligningen. 

I dette projekt vil forbruget ikke blive delt op i basis- og varierende forbrug. Det antages, at forbrugerens løn og forventning til arbejdstatus er fast over tidsperioden. Derfor påvirkes disse faktorer blandt andet ikke af pandemier eller finanskriser, som ofte medfører højere arbejdsløshed og lavere løn. Derudover antages det, at den disponible indkomst er lig lønnen, når forbrugeren er i arbejde og nul i de beslutningstidspunkter, hvor han er arbejdsløs. Den disponible indkomst påvirkes dermed ikke af skatteændringer. Sandsynligheden for at være i arbejde vil blive beskrevet ved en Markov-kæde, hvor forbrugerens arbejdsstatus i næste tidsskridt kun afhænger af hans arbejdsstatus i nuværende tidsskridt.

Dette projekt tager udgangspunkt i den fornuftige forbruger, der opnår størst mulig nytte ved at fordele sit forbrug ligeligt mellem beslutningstidspunkterne. Det antages, at det optimale forbrug til hvert beslutningstidspunkt udelukkende afhænger af sandsynligheden for at blive arbejdsløs, forblive arbejdsløs, opsparing og indlånsrente. Når muligheden for at låne penge tilføjes, afhænger forbruget til hvert beslutningstidspunkt også af dette, samt udlånsrenten.

Det er derfor interessant at undersøge, hvordan forbrugerens nytte og forbrug til hvert beslutningstidspunkt påvirkes, hvis sandsynligheden for at blive arbejdsløs, indlånsrenten og muligheden for at låne varierer. 


%For at kunne beskrive Markov-kæderne for forbrugerens %arbejdsstatus og Markov beslutningsprocessen vil der %introduceres begrebet forventet værdi. Derudover %præsenteres begreber som tilfældige variabler og %sandsynlighedsrum.

%for at kunne beskrive forventet værdi, Markov kæder og %Markovbeslutningsprocesser. 

Ud fra ovenstående problemafgrænsning opstilles følgende problemformulering.





 



