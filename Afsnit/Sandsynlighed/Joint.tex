%I ovenstående afsnit gælder resultaterne for én diskret tilfældig variabel. Dette kan udvides til \textit{diskrete tilfældige vektorer}.

%\begin{minipage}\textwidth
%\begin{defn}\textbf{Diskret tilfældig vektor} %\label{def:fælles_frekvensfunktion}%Ny %definition
%\newline
%Lad $X_1, X_2, \dots, X_n$ være diskrete %tilfældige variabler, så gælder det at $(X_1, %X_2, \dots, X_n)$ kaldes en n-dimensional %diskret tilfældig vektor.
%\end{defn}
%\end{minipage}
%$\{(x_i,x_j\cdots x_k) for i,j,k=1, 2...n\}$
%....

%\begin{minipage}\textwidth
%\begin{defn}\textbf{Fælles frekvensfunktion} %\label{def:fælles_frekvensfunktion}%Ny %definition
%\newline
%Lad $X_1, X_2, \dots, X_n$ være diskrete %tilfældige variabler på sandsynlighedsrummet %$(\omega, \mathcal{F}, P)$. Funktionen $p_{X_1, %X_2, \dots, X_n}:\R^n\to [0, 1]$ defineret ved 
%\begin{align*}
%    p_{X_1, X_2, \dots, X_n}(x_1, x_2, \dots, %x_n)=P(X_1=x_1, X_2=x_2, \dots, X_n=x_n)
%\end{align*}
%kaldes for en \textit{fælles frekvensfunktion} %eller \textit{fælles pmf} til $(X_1, X_2, %\dots, X_n)$.
%\end{defn}
%\end{minipage}
