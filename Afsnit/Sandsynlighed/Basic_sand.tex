%To primære fortolkninger af sandsynlighed er, at en sandsynlighed er grænsen for den relative frekvens eller en vis grad af tro. Ud over disse to fortolkninger er det muligt at formulere en matematisk definition af, hvad sandsynlighed er.


%

%Hvis man kaster en mønt vil de fleste mene at der er $50\%$ chance for at den lander på krone. Derudover kan man kaste mønten mange gange og se at ca halvdelen af gangene man kaster mønten rammer den krone. I dette afsnit vil en matematiske definition på sandsynlighed præsenteres, og det vil være denne definition som fremadrettet vil blive anvendt.

%I dette projekt defineres sandsynlighed aksiomatisk
%Endeligt, er det muligt at formulere en matematisk definition af, hvad sandsynlighed er.


%Derudover bliver ordet sandsynlighed brugt om hvad en person tror vil være 



%For at kunne beskrive en sandsynlighed matematisk skal begreber som \textit{tilfældighed}, \textit{udfaldsrum}, \textit{hændelser} og \textit{hændelsesrum} først beskrives.

Noget kaldes \textit{tilfældigt}, hvis det ikke er muligt at forudse udfaldet. Hvis der udføres et eksperiment, hvor udfaldet er tilfældigt, kaldes det et \textit{tilfældigt eksperiment}, $\E$. Et eksempel på et tilfældigt eksperiment er et terningkast, hvor det ikke er sikkert, om der slås $1, 2, \dots$ eller $6$. %%For et tilfældigt eksperiment defineres begreberne, \textit{udfaldsrum}, \textit{hændelser} og \textit{hændelsesrum} som følgende

Mængden af mulige udfald i et tilfældigt eksperiment, $\E$, kaldes et \textit{udfaldsrum} og bliver noteret, $\Omega$. Et udfaldsrum kan bestå af en \textit{endelig}, \textit{tællelig uendelig} eller en \textit{ikke-tællelig uendelig} mængde af udfald. Hvis udfaldene er adskilt i punkter, eksempelvis $\{1,2,3, \dots\}$, er det et tælleligt udfaldsrum, og hvis det derimod er et kontinuum af punkter, er det ikke-tælleligt. Eksempelvis, er udfaldsrummet $[0, \infty)$ ikke-tælleligt.

%\begin{minipage}\textwidth
%\begin{defn}\textbf{Hændelsesrum}\\
    %Lad $\Omega$ være et udfaldsrum for $\E$
    %Lad $\Omega$ være et udfaldsrum. Så er hændelsesrummet givet ved potensmængden
    %$$\Sigma = \{A:A\subseteq \Omega \},$$
%hvor elementerne, $A\in \Sigma$, kaldes for \textit{hændelser}.
%\end{defn}
%\end{minipage}

\begin{defn}\textbf{Hændelsesrum}
\newline
    Mængden, $\mathcal{F}$, af delmængder af udfaldsrummet, $\Omega$, kaldes et hændelsesrum, hvis
    \begin{enumerate}
        \item $\F$ er ikke-tom,
        \item Hvis $A\in\F$ så er $A^c=\Omega\backslash A \in \mathcal{F}$,
        \item Hvis $A_1, A_2,\dots\in \F$ så er $\displaystyle \bigcup_{i=1}^\infty A_i\in\F$.
    \end{enumerate}
Elementerne, $A\in \F$, kaldes for \textit{hændelser}.
\end{defn}


Hvis der for to hændelser $A$ og $B$ gælder, at $A \cap B = \varnothing$, kaldes de for \textit{disjunkte}.

% Derved er hændelsesrummet givet ud fra alle mulige hændelser.

%Følgende beskriver et eksempel for udfaldsrummet og hændelserne for det tilfældige eksperiment, hvor der kastes med en terning.

% \begin{eks} \textbf{} %Nyt eksempel
% \newline
% Det tilfældige eksperiment går ud på, at der kastes en terning og udfaldet observeres.

% For dette tilfældige eksperiment vil udfaldene kunne være tallene fra 1 til 6. Dermed vil udfaldsrummet være
% \begin{align*}
%     S = \{1, 2, 3, 4, 5, 6\}
% \end{align*}
% %Antag, at udfaldsrummet deles op i to hændelser, lige og ulige. Så er udfaldsrummet givet ved

% % Hvis eksemplet beholdes synes jeg vi skal skrive udfaldsrummet op her, og sige at A og B er hændelserne .. ulige og lige tal.. som viser sig at være disjunkte mængder.... - Mikkel

% Det er også muligt at have to hændelser, når man kaster en terning, hvor den ene hændelse er at få ulige numre, og den anden er at få lige numre. Altså er delmængderne at udfaldsrummet følgende
% \begin{align*}
%     A = \{1, 3, 5\} \quad B = \{2, 4, 6\}
% \end{align*}
% I dette tilfælde er $A$ og $B$ disjunkte, da $A \cap B = \varnothing$. Altså er det ikke muligt, at både $A$ og $B$ kan hænde på samme tid.
% \end{eks}

Det er hermed muligt at definere et \textit{sandsynlighedsmål}.

\begin{defn}\textbf{Sandsynlighedsregningens grundsætninger} \label{def:sandsynlighedsregningens_grundsætninger}%Ny definition
\newline
Lad $\Omega$ være et udfaldsrum, $\F$ et hændelsesrum og $A$ en hændelse.\\
En funktion, $P: \F \to [0,1]$, kaldes et sandsynlighedsmål på $(\Omega, \F)$, hvis den opfylder, at
\begin{enumerate}
    \item $0 \leq P(A) \leq 1$,
    \item $P(\Omega) = 1$ og $P(\emptyset)=0$,
    \item hvis $A_1, A_2, \cdots $ er en sekvens af parvise disjunkte hændelser, altså $A_i \cap A_j = \varnothing$ for $i \neq j$, så er
    \begin{align*}
        P\left(\bigcup_{k=1}^\infty A_k \right) = \sum_{k=1}^\infty P(A_k).
    \end{align*}
\end{enumerate}

\end{defn}
Ud fra punkt 3 i \autoref{def:sandsynlighedsregningens_grundsætninger} følger det, at
\begin{align*}
    P\left(\bigcup_{k=1}^n A_k \right) = \sum_{k=1}^n P(A_k),
\end{align*}
ved at lade $A_k = \emptyset$ for $k > n$.

Ud fra udfaldsrum, hændelsesrum og sandsynlighedsmål defineres \textit{sandsynlighedsrummet} som følgende

\begin{minipage}\textwidth
\begin{defn}\textbf{Sandsynlighedsrum} \label{def:sandsynlighedsrum}\\
    Lad $\Omega$ være et udfaldsrum, $\mathcal{F}$ et hændelsesrum og $P$ et sandsynlighedsmål på $(\Omega, \F)$. Så er sandsynlighedsrummet givet ved triplen, $(\Omega, \F, P)$. 
\end{defn}
\end{minipage}



