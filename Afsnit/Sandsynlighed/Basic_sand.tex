%To primære fortolkninger af sandsynlighed er, at en sandsynlighed er grænsen for den relative frekvens eller en vis grad af tro. Ud over disse to fortolkninger er det muligt at formulere en matematisk definition af, hvad sandsynlighed er.


%

%Hvis man kaster en mønt vil de fleste mene at der er $50\%$ chance for at den lander på krone. Derudover kan man kaste mønten mange gange og se at ca halvdelen af gangene man kaster mønten rammer den krone. I dette afsnit vil en matematiske definition på sandsynlighed præsenteres, og det vil være denne definition som fremadrettet vil blive anvendt.

%I dette projekt defineres sandsynlighed aksiomatisk
%Endeligt, er det muligt at formulere en matematisk definition af, hvad sandsynlighed er.


%Derudover bliver ordet sandsynlighed brugt om hvad en person tror vil være 



%For at kunne beskrive en sandsynlighed matematisk skal begreber som \textit{tilfældighed}, \textit{udfaldsrum}, \textit{hændelser} og \textit{hændelsesrum} først beskrives.

Noget kaldes \textit{tilfældigt}, hvis det ikke er muligt at forudse udfaldet. Hvis der udføres et eksperiment, hvor udfaldet er tilfældigt, kaldes det et \textit{tilfældigt eksperiment}, $\E$. Et eksempel på et tilfældigt eksperiment er et terningkast, hvor det ikke er sikkert, hvorvidt udfaldet er $1, 2, \dots$ eller $6$. %%For et tilfældigt eksperiment defineres begreberne, \textit{udfaldsrum}, \textit{hændelser} og \textit{hændelsesrum} som følgende

Mængden af mulige udfald i et tilfældigt eksperiment, $\E$, kaldes et \textit{udfaldsrum} og noteres $\Omega$. Et udfaldsrum kan bestå af en \textit{endelig}, \textit{tællelig uendelig} eller en \textit{ikke-tællelig uendelig} mængde af udfald. Hvis udfaldene er adskilt i punkter, eksempelvis $\{1,2,3, \dots\}$, er det et tælleligt udfaldsrum, og hvis det derimod er et kontinuum af punkter, er det ikke-tælleligt. Eksempelvis er udfaldsrummet $[0, 10]$ ikke-tælleligt.

Ud fra udfaldsrummet defineres begrebet \textit{hændelsesrum} og \textit{hændelse} som følgende.

\begin{defn}\textbf{Hændelsesrum}
\newline
    Mængden af delmængder af udfaldsrummet, $\Omega$, kaldes et hændelsesrum og noteres $\mathcal{F}$, hvis
    \begin{enumerate}
        \item $\F$ er ikke-tom,
        \item $A\in\F$, så er $A^c=\Omega\backslash A \in \mathcal{F}$,
        \item $A_1, A_2,\dots\in \F$, så er $\displaystyle \bigcup_{k=1}^\infty A_k\in\F$.
    \end{enumerate}
Elementerne, $A\in \F$, kaldes for hændelser.
\end{defn}

Hvis der for to hændelser $A$ og $B$ gælder, at $A \cap B = \varnothing$, kaldes hændelserne for \textit{disjunkte}.

Det er ud fra ovenstående muligt at definere et \textit{sandsynlighedsmål}.

\begin{defn}\textbf{Sandsynlighedsregningens aksiomer} \label{def:sandsynlighedsregningens_grundsætninger}%Ny definition
\newline
Lad $\Omega$ være et udfaldsrum, $\F$ et hændelsesrum og $A$ en hændelse. En funktion,\\ $P: \F \to [0,1]$, kaldes et sandsynlighedsmål på $(\Omega, \F)$, hvis
\begin{enumerate}
    \item $0 \leq P(A) \leq 1$,
    \item $P(\Omega) = 1$ og $P(\emptyset)=0$,
    \item $(A_1, A_2, \ldots) $ er en sekvens af parvise disjunkte hændelser, altså $A_i \cap A_j = \emptyset$ for $i \neq j$, så er
    \begin{align*}
        P\left(\bigcup_{k=1}^\infty A_k \right) = \sum_{k=1}^\infty P(A_k).
    \end{align*}
\end{enumerate}
\end{defn}

Jævnfør punkt 3 i \autoref{def:sandsynlighedsregningens_grundsætninger} følger det, at
\begin{align*}
    P\left(\bigcup_{k=1}^n A_k \right) = \sum_{k=1}^n P(A_k),
\end{align*}
ved at lade $A_k = \emptyset$ for $k > n$.

Ud fra udfaldsrum, hændelsesrum og sandsynlighedsmål defineres \textit{sandsynlighedsrum} som følgende

\begin{minipage}\textwidth
\begin{defn}\textbf{Sandsynlighedsrum} \label{def:sandsynlighedsrum}\\
    Lad $\Omega$ være et udfaldsrum, $\mathcal{F}$ et hændelsesrum og $P$ et sandsynlighedsmål på $(\Omega, \F)$. Så er sandsynlighedsrummet givet ved triplen, $(\Omega, \F, P)$. 
\end{defn}
\end{minipage}





