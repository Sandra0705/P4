To primære fortolkninger af sandsynlighed er, at en sandsynlighed er grænsen for den relative frekvenser eller en vis grad af tro. Ud over disse to fortolkninger er det muligt at formulere en mere matematisk præcis definition af hvad sandsynlighed er.

For at kunne beskrive hvad en sandsynlighed mere matematisk præcis, skal begreber som \textit{tilfældighed}, \textit{udfaldsrummet} og \textit{hændelser} først beskrives.

I en situation hvor der sker noget, der ikke er muligt at forudse det præcise udfald kaldes \textit{tilfældig}. Et eksempel på dette kan være, hvis man kaster med en terning, hvor man ikke er sikker på om man slår $1, 2, \cdot$ eller $6$. Dette kaldes for et \textit{tilfældigt eksperiment}. For et tilfældigt eksperiment kan begreberne udfaldsrum og hændelser defineres som følgende

\begin{minipage}\textwidth
\begin{defn}\textbf{Udfaldsrum} %Ny definition
\newline
Mængden af mulige udfald i et tilfældigt eksperiment kaldes et udfaldsrum og bliver noteret, $S$.
\end{defn}
\end{minipage}

Et udfaldsrum kan være \textit{endeligt}, \textit{tællelig uendeligt} og \textit{ikke-tællelig uendeligt}. Endeligt betyder, at udfaldsrummet består af et endeligt antal udfald, hvor der gælder for de to andre, at udfaldsrummet består at et uendeligt antal udfald. Hvis udfaldene er adskilt i punkter som foreksempel $\{1,2,3, \cdots\}$ er udfaldsrummet tællelig, og hvis det derimod er et kontinuum af punkter er det ikke-tællelig. Et eksempel på dette er udfaldsrummet $[0, \infty)$.

\begin{minipage}\textwidth
\begin{defn}\textbf{Hændelse} %Ny definition
\newline
En delmængde af udfaldsrummet, $A \subseteq S$, kaldes en hændelse.
\end{defn}
\end{minipage}

Hvis der for to hændelser $A$ og $B$ gælder at $A \cap B = \varnothing$ kaldes de for \textbf{\textit{disjunkte}}.

Følgende beskriver et eksemplet for udfaldsrummet og hændelser for det tilfældige eksperiment, hvor der kastes med en terning.

\begin{eks} \textbf{} %Nyt eksempel
\newline
Det tilfældige eksperiment går ud på, at man kaster en terning og observerer nummeret.

For dette tilfældige eksperiment vil udfaldene kunne være tallene fra 1 til 6. Dermed vil udfaldsrummet være
\begin{align*}
    S = \{1, 2, 3, 4, 5, 6\}
\end{align*}

Det er også muligt at have to hændelser, når man kaster en terning, hvor den ene hændelse er at få ulige numre og den anden er at få lige numre. Altså er delmængderne at udfaldsrummet følgende
\begin{align*}
    A = \{1, 3, 5\} \quad B = \{2, 4, 6\}
\end{align*}
I dette tilfælde er $A$ og $B$  \textbf{disjunkte}, da $A \cap B = \varnothing$. Altså er det ikke muligt at både $A$ og $B$ kan hænde på samme tid.
\end{eks}

Det er hermed muligt at lave en matematisk definition af sandsynlighed.

\begin{defn}\textbf{Sandsynlighedsregningens grundsætninger} \label{def:sandsynlighedsregningens_grundsætninger}%Ny definition
\newline
Et sandsynlighedsmål er en funktion $P$, 
der tildeler hver hændelse $A$ et tal $P(A)$, der opfylder at
\begin{enumerate}
    \item[(a)] $0 \leq P(A) \leq 1$
    \item[(b)]$P(S) = 1$ og $P(\emptyset)=0$
    \item[(c)] Hvis $A_1, A_2, \cdots$ er en sekvens af parvise \textbf{disjunkte} hændelser, altså at hvis $i \neq j$, så er $A_i \cap A_j = \varnothing$, så gælder det at
\end{enumerate}
\begin{align*}
    P\left(\bigcap_{k=1}^\infty A_k \right) = \sum_{k=1}^\infty P(A_k)
\end{align*}
\end{defn}

Ud fra det sidste aksiom i \autoref{def:sandsynlighedsregningens_grundsætninger} følger det naturligt, at hvis $A_1, A_2, \cdots , A_n$ er en sekvens af parvise \textbf{disjunkte} hændelser så gælder det at
\begin{align*}
    P\left(\bigcap_{k=1}^n A_k \right) = \sum_{k=1}^n P(A_k)
\end{align*}

\textbf{Tænker at vi skriver de regneregler(inklusiv lim) ind vi bruger hen ad vejen. gider ikke lige skrive dem alle sammen ind og bevise dem hvis vi kun anvender 2,3 stykker. samme med regrad to order/replacement}













