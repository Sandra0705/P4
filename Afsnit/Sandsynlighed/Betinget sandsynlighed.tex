I et tilfældigt eksperiment kan sandsynligheden for én hændelse være \textit{betinget} af en anden hændelse. Dette kaldes \textit{betinget sandsynlighed}, hvilket defineres som følgende

%Hvis der er givet ekstra information om en given hændelse, så vil udfaldsrummet indskrænkes - mere formelt:\\
\begin{minipage}\textwidth
\begin{defn}\textbf{Betinget sandsynlighed} \label{def:betinget_sandsynlighed}%Ny definition
\newline
Lad $(\Omega, \F, P)$ være et sandsynlighedsrum, $A,B\in \F$ og $P(B)>0$. Så gælder det, at den \textit{betingede sandsynlighed}, af $A$ givet $B$, er
\begin{align*}
      \displaystyle P(A|B)=\frac{P(A\cap B)}{P(B)}.
\end{align*}
\end{defn}
\end{minipage}

Eksempelvis, kastes en mønt to gange. Det oplyses, at det andet udfald er det samme som det første. Dermed er sandsynligheden af andet kast betinget af første kast.

Hvis to hændelser, $A$ og $B$, er \textit{uafhængige}, vil en hændelse $B$ ikke påvirke sandsynligheden for $A$. Dermed gælder det, at den betingede og ubetingede sandsynlighed er den samme, altså $P(A)=P(A|B)\Leftrightarrow P(A\cap B)=P(A)P(B)$. Dette generaliseres til følgende definition

\begin{minipage}\textwidth
\begin{defn}\textbf{Uafhængighed} %Ny definition
\newline
Lad $B\subseteq \F$. Hændelserne i $B$ siges at være uafhængige, hvis der for alle hændelser i $B$ gælder, at
\begin{align}\label{eq:uafhængighed}
    P\left(\bigcap_{A\in B} A\right)=\prod_{A\in B} P(A).
\end{align}
Hvis de ikke er uafhængige, siges de at være \textit{afhængige}.
\end{defn}
\end{minipage}

Hvis $|B|=2$ og \eqref{eq:uafhængighed} er sand, siges hændelserne i $B$ at være \textit{parvist uafhængige}.

\subsection{Total sandsynlighed}
%I tilfælde af, at sandsynligheden for en given hændelse er svær at bestemme direkte kan det i nogle tilfælde være hensigtsmæssigt at opdele problemet i termer af betinget sandsynlighed.

Betinget sandsynlighed kan i nogle tilfælde simplificere beregningen af sandsynligheden for en given hændelse. 

\begin{minipage}\textwidth
\begin{thmx} \textbf{Loven om total sandsynlighed} \label{sæt:loven_om_total_sandsynlighed} %Ny sætning
\newline
Lad $A$ være en hændelse og $B_1,B_2,\dots$ være en sekvens af parvist disjunkte hændelser som opfylder, at
\begin{enumerate}
\item $P(B_k)>0$ for $k=1,2,\dots$,
\item $\Omega=\displaystyle\bigcup_{k=1}^\infty B_k$.
\end{enumerate}
Så gælder der for $A$, at
\begin{align*}
    P(A)=\sum_{k=1}^\infty P(A|B_k)P(B_k).
\end{align*}
\end{thmx}
\end{minipage}
\begin{bev} \textbf{} %Nyt bevis
\newline
Lad $A\subseteq \Omega$ og $B_k \subseteq \Omega $ være en sekvens af parvist disjunkte hændelser og $P(B_k)>0$ for $k=1,2,\dots$. Bemærk, at
\begin{align*}
    A=A\cap \Omega=A\cap\bigcup_{k=1}^\infty B_k=\bigcup_{k=1}^\infty(A\cap B_k),
\end{align*}
som følger af den distributive lov for uendelige foreningsmængder (se \autoref{Distributive love}). 
Eftersom $A\cap B_1,A\cap B_2,\dots$ er parvis disjunkte følger det af \autoref{def:sandsynlighedsregningens_grundsætninger} punkt 3 og \autoref{def:betinget_sandsynlighed}, at
\begin{align*}
    P(A)=P\left(\bigcup_{k=1}^\infty(A\cap B_k)\right) =\sum_{k=1}^\infty P(A\cap B_k)=\sum_{k=1}^\infty P(A|B_k)P(B_k).
\end{align*}
Dermed er \autoref{sæt:loven_om_total_sandsynlighed} bevist.
\end{bev}

% \begin{eks} \textbf{} %Nyt eksempel
% \newline
% Der er to poser. De har begge 10 kugler. Den ene har 5 røde og 5 blå. Den anden har 3 røde og 7 blå. 
% Én pose vælges tilfældigt, hvorefter der trækkes en kugle tilfældigt. Hvad er sandsynligheden for, at der trækkes en rød?\\

% Lad 
% \begin{align*}
%     R&=\{\text{Kuglen er rød}\}=A\\
%     P_1&=\{\text{Trækker fra pose ét}\}=B_1\\
% \end{align*}
% Enten trækker man af den pose eller den anden. Altså er $B_2=P_1^c$.
% Derfor er $S=\bigcup_{k=1}^\infty B_k=P_1\cup P_2$.
% Der gælder desuden, at
% \begin{align*}
%     P(P_1)=\frac{1}{2}>0\\
%     \text{ og }
%     P(p_1^c)=\frac{1}{2}>0
% \end{align*}

% Så følger det af loven om Total Sandsynlighed, at
% \begin{align*}
%     P(R)&=P(R|P_1)P(P_1)+P(R^c|P_1^c)P(P_1^c)\\
%     &=\frac{5}{10}\cdot\frac{1}{2}+\frac{3}{10}\cdot\frac{1}{2}\\
%     &=\frac{2}{5}
% \end{align*}

%\end{eks}
%By virtue of Proposition 1.2, we realize that the law of total probability is also true for a finite union of events
%Hvis mængden af hændelser er endelig og givet ved $n$, kan man konstruere en uendelig følge af hændelser, sådan at P(\bigcup_{k=n}^\infty B_k)=0.
%For $n=2$ gælder der, at hvis $B_1=B$, så følger det at $B_2=B^c$ og følgende korollar fås.
%\begin{minipage}\textwidth
%\begin{kor} \textbf{} %Nyt korollar
%\newline
%Hvis $0<P(B)<1$, så er
%\begin{align*}
%    P(A)=P(A|B)P(B)+P(A|B^c)P(B^c)
%\end{align*}
%\end{kor}
%\end{minipage}

\iffalse
Idét foreningsmængder er kommutative, gælder der, at
$$P(B_j|A)P(A)=P(A|B_j)P(B_j)$$
som må betyde, at
$$P(B_j|A)=\frac{P(A|B_j)P(B_j)}{P(A)}$$

Dette leder op til følgende proposition:\\
\begin{minipage}\textwidth
\begin{pro} \textbf{(Bayes' Formel).} %Ny proposition
\newline
Med samme antagelser som i Loven om Total Sandsynlighed og hvis $P(A)>0$, så gælder der for enhver hændelse, $B_j$, at
\begin{align*}
    P(B_j|A)=\frac{P(A|B_j)P(B_j)}{\sum_{k=1}^\infty P(A|B_k)P(B_k)}
\end{align*}
\end{pro}
\end{minipage}
\begin{bev}
Beviset følger af foregående udregninger.
\end{bev}

Analogt, følger denne korrollar:\\
\begin{minipage}\textwidth
\begin{kor} \textbf{} %Nyt korollar
\newline
Hvis $0<P(B)<1$ og $P(A)>0$, så er
\begin{align*}
    P(B|A)=\frac{P(A|B)P(B)}{P(A|B)P(B)+P(A|B^c)P(B^c)}
\end{align*}
\end{kor}
\end{minipage}
\fi