I et tilfældigt eksperiment kan sandsynligheden for en hændelse være \textit{betinget} af en anden hændelse. Dette kaldes \textit{betinget sandsynlighed} og defineres som følgende

%Hvis der er givet ekstra information om en given hændelse, så vil udfaldsrummet indskrænkes - mere formelt:\\
\begin{minipage}\textwidth
\begin{defn}\textbf{Betinget sandsynlighed} \label{def:betinget_sandsynlighed}%Ny definition
\newline
Lad $(\Omega, \F, P)$ være et sandsynlighedsrum, $A,B\in \F$ være hændelser og $P(B)>0$. Så gælder det, at den \textit{betingede sandsynlighed} af $A$ givet $B$ er
\begin{align*}
      \displaystyle P(A|B)=\frac{P(A\cap B)}{P(B)}.
\end{align*}
\end{defn}
\end{minipage}

Hvis to hændelser, $A,B\in \F$, er \textit{uafhængige}, vil sandsynligheden for $B$ ikke påvirke sandsynligheden for $A$. Dermed gælder det, at den betingede og ubetingede sandsynlighed er den samme, altså
\begin{align*}
    P(A)=P(A|B)\Leftrightarrow P(A\cap B)=P(A)P(B).
\end{align*}
Dette generaliseres til følgende definition.

\begin{minipage}\textwidth
\begin{defn}\textbf{Uafhængighed}\label{def:uafhængighed} %Ny definition
\newline
Lad $(\Omega, \F, P)$ være et sandsynlighedsrum og $B\subseteq \F$. Hændelserne i $B$ siges at være uafhængige, hvis der for alle hændelser i $B$ gælder, at
\begin{align}\label{eq:uafhængighed}
    P\left(\bigcap_{A\in B} A\right)=\prod_{A\in B} P(A).
\end{align}
Hvis hændelserne ikke er uafhængige, siges de at være \textit{afhængige}.
\end{defn}
\end{minipage}

Lad $B \subseteq \F$ og \eqref{eq:uafhængighed} være opfyldt. Hvis $|B| = 2$, siges hændelserne i $B$ at være \textit{parvist uafhængige}.

\subsection{Total sandsynlighed}
%I tilfælde af, at sandsynligheden for en given hændelse er svær at bestemme direkte kan det i nogle tilfælde være hensigtsmæssigt at opdele problemet i termer af betinget sandsynlighed.

Betinget sandsynlighed kan i nogle tilfælde simplificere beregningen af sandsynligheden for en given hændelse. 

\begin{minipage}\textwidth
\begin{thmx} \textbf{Loven om total sandsynlighed} \label{sæt:loven_om_total_sandsynlighed} %Ny sætning
\newline
Lad $(\Omega, \F, P)$ være et sandsynlighedsrum, $A$ være en hændelse og $(B_1,B_2,\ldots)$ være en sekvens af parvist disjunkte hændelser som opfylder, at
\begin{enumerate}
\item $P(B_k)>0$ for $k = 1, 2, \ldots$
\item $\Omega=\displaystyle\bigcup_{k=1}^\infty B_k$.
\end{enumerate}
Så gælder det, at
\begin{align*}
    P(A)=\sum_{k = 1}^\infty P(A|B_k)P(B_k).
\end{align*}
\end{thmx}
\end{minipage}
\begin{bev} \textbf{} %Nyt bevis
\newline
Lad $A$ være en hændelse og $(B_1,B_2,\ldots)$ være en sekvens af parvist disjunkte hændelser som opfylder punkt 1 og 2 i \autoref{sæt:loven_om_total_sandsynlighed}. Bemærk, at
\begin{align*}
    A=A\cap \Omega=A\cap\bigcup_{k=1}^\infty B_k=\bigcup_{k=1}^\infty(A\cap B_k),
\end{align*}
som følger af den distributive lov for uendelige foreningsmængder (se \autoref{Distributive love}). 
Eftersom $A\cap B_1,A\cap B_2,\dots$ er parvist disjunkte, følger det af \autoref{def:sandsynlighedsregningens_grundsætninger} punkt 3 og \autoref{def:betinget_sandsynlighed}, at
\begin{align*}
    P(A)=P\left(\bigcup_{k=1}^\infty(A\cap B_k)\right) =\sum_{k=1}^\infty P(A\cap B_k)=\sum_{k=1}^\infty P(A|B_k)P(B_k).
\end{align*}
Dermed er \autoref{sæt:loven_om_total_sandsynlighed} bevist.
\end{bev}

