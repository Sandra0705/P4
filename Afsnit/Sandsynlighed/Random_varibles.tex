En \textit{tilfældig variabel}, $X$, er en variabel, som antager sin værdi fra et tilfældigt eksperiment, $\E$. I dette projekt antages det, at tilfældige variabler er reelle og diskrete. 
Tilfældige variabler kan udtrykkes som reelle funktioner, der tildeler værdier til eksperimentets mulige udfald.


Da $X$ først antager en værdi efter $\E$, er det kun muligt at beskrive \textit{billedet} af $X$ samt de tilhørende sandsynligheder. Billedet af $X$ betegnes $\text{Range}(X)$ og er mængden af mulige værdier, $X$ kan antage. Mængden af de tilhørende sandsynligheder kaldes for \textit{sandsynlighedsfordelingen} af $X$.

% Før eksperimentet er det muligt at beskrive mængden af mulige værdier, $range(X)$, og de tilhørende sandsynligheder, \textit{fordelingen} af $X$.
% Det betyder, at det er muligt at beskrive billedet af $X$, men ikke hvilken værdi $X$ antager.

\begin{minipage}\textwidth
\begin{defn}\label{def:Diskret_tilfældig_variabel}\textbf{Diskret tilfældig variabel} %Ny definition
\newline
Lad $(\Omega, \F, P)$ være et sandsynlighedsrum. En diskret tilfældig variabel, $X:\Omega \to \R$, er en reel funktion på sandsynlighedsrummet således, at
%
\begin{align}
    &\text{$\text{Range}(X)$ er en tællelig delmængde af $\R$ } \quad \text{og}\label{eq:def_diskretitet_betingelsen}\\
    &\forall x\in \R: \{\omega\in \Omega: X(\omega)=x\}\in % S\cup \{\emptyset\} \subseteq 
    \F. \label{eq:def_tilfældig_variabel}
\end{align}

\end{defn}
\end{minipage}

En funktions billede kan være endeligt, tællelig uendeligt og ikke-tællelig uendeligt. I \autoref{def:Diskret_tilfældig_variabel} er \eqref{eq:def_diskretitet_betingelsen} betingelsen, der sikrer, at $X$ er en diskret funktion. Da funktionen er diskret, er dens billede tællelig.

Det bemærkes, at $X$ antager værdien $x$, hvis og kun hvis resultatet af $\E$ er et element af delmængden af $\Omega$, som afbildes over i $x$. Derved ligger resultatet af $\E$ i delmængden $X^{-1}(x) = \{\omega\in \Omega: X(\omega)=x\} \subseteq \Omega$, hvor det bemærkes, at  $X^{-1}(x)\subseteq \Omega \in \F$.
Betingelse \eqref{eq:def_tilfældig_variabel} medfører altså, at urbilledet, $X^{-1}(x)$ til ethvert $x$, er element af hændelsesrummet, $\F$, og har derfor den tilsvarende sandsynlighed, $P(\omega\in \Omega: X(\omega)=x)$.

Fremadrettet vil $\{\omega\in \Omega: X(\omega)=x\}$ betegnes ved $\{X=x\}$. Da den diskrete tilfældige variabel kan antage flere værdier, defineres en funktion, der bestemmer sandsynligheden for hvert udfald.

\begin{minipage}\textwidth
\begin{defn}\label{def:Frekvensfunktionen}\textbf{Frekvensfunktion} %Ny definition
\newline
    Lad $X$ være en diskret tilfældig variabel. % med range $\{x_k| k\in N \subseteq \N\}$. 
    Funktionen $p_X:\R\to [0,1]$ defineret ved
%    
\begin{align}\label{eq:p_x}
    p_X(x) = P(X = x),
\end{align}
%    
    kaldes for en \textit{frekvensfunktion} til $X$.
\end{defn}
\end{minipage}

Frekvensfunktionen er derved sandsynligheden for, at $X$ antager værdien $x$. Hvis $x\not\in \text{Range}(X)$, gælder det, at $p_X(x) = 0$.  

%$p_X(x)=0$ for $x\notin range(X)$.
En frekvensfunktion har følgende egenskaber
\begin{pro}\label{prop:frekvensfunktion}\textbf{}
\newline
    Lad $X$ være en diskret tilfældig variabel og lad $p_X$ være den tilhørende frekvensfunktion. Så gælder det, at
    %
   % En funktion $p_X$ er en frekvensfunktion for en diskret tilfældig variable $X$, %  med $range(X)=\{x_k | k\in N \subseteq \N\}$
    %hvor det gælder, at
    \begin{enumerate}
        \item $\displaystyle p_X(x)\geq 0$,
        %\item[(b)] $\displaystyle \sum_{k=1}^\infty p(x_k)=1$
        \item $\displaystyle \sum_{x\in \R} p_X(x)=1$
    \end{enumerate}
    %hvor summen er endelig, hvis $range(X)$ er endelig.
    for alle $x\in \R$.
\end{pro}


\begin{bev} \textbf{} %Nyt bevis
\newline
Lad $X$ være en diskret tilfældig variabel og $I=\text{Range}(X)$. Lad derudover $p_X$ være frekvensfunktionen til $X$.

\textbf{Bevis for punkt 1}\\
    % Antag først at, $x\in Range(X)$, så vil frekvensfunktionen være sandsynligheden for at $X$ giver hændelsen $x$. Fra \autoref{def:sandsynlighedsregningens_grundsætninger} punkt (a) fås, at enhver hændelse har en ikke-negative sandsynlighed, og derfor må $p_X(x) = P(X=x) \geq 0$. \\
    % Antag omvendt, at $x\notin Range(X)$, så vil det være en umulig hændelse, at $X$ giver hændelsen $x$. Da sandsynligheden af en umulig hændelse altid er 0, fås $p_X(x)=P(\emptyset) = 0$.\\
    % Derved er det bevist, at $p_X(x) \geq 0$
    % 
    Frekvensfunktionen er defineret for alle $x\in \R$ og angiver sandsynligheden for, at $X=x$.
    Fra \autoref{def:sandsynlighedsregningens_grundsætninger} punkt 1 fås, at enhver hændelse har en ikke-negativ sandsynlighed, og derfor må $p_X(x) = P(X=x) \geq 0$.

\textbf{Bevis for punkt 2}\\
    Lad $\displaystyle\sum_{x\in \R} p_X(x)$ være givet.
    Summen omskrives ved brug af \eqref{eq:p_x}.
    \vspace{-0.4cm}
    \begin{align*}
    \sum_{x\in \R} p_X(x) &=  \sum_{x\in \R} P(X=x).
    \intertext{Dette omskrives ved brug af  \autoref{def:sandsynlighedsregningens_grundsætninger} punkt 3}
     \sum_{x\in \R} P(X=x)  &= P\left( \bigcup_{x\in \R} \{X=x\}\right)\\
    &= P\left( \bigcup_{x\in I} \{X=x\}\right) + P\left( \bigcup_{x\notin I} \{X=x\}\right)\\
     %P\left( S\cup \emptyset \right) = P(S) + P(\emptyset) 
    &= P(\Omega) + P(\emptyset).
\end{align*} 
    Fra \autoref{def:sandsynlighedsregningens_grundsætninger} punkt 2 følger det, at $P(\Omega)=1$ og $P(\emptyset)=0$. 
    
Dermed er \autoref{prop:frekvensfunktion} bevist.
\end{bev}

Hvis frekvensfunktionen har den samme værdi for alle $x \in \text{Range}(X)$, siges mængden af sandsynligheder at være \textit{uniform fordelt}.\\
Frekvensfunktionen kan illustreres i et pindediagram, hvor søjlerne repræsenterer de forskellige udfald, og højden af en given søjle repræsenterer sandsynligheden for udfaldet. Dette illustreres i \autoref{eks:frekvensfunktion_med_terninger}.

\begin{eks} \textbf{}\label{eks:frekvensfunktion_med_terninger} %Nyt eksempel
\newline
Lad $\E$ være et tilfældigt eksperiment hvor der kastes to fair terninger. Lad udfaldsrummet være givet ved %$S=\left\{\{1,1\},\{1,2\},\cdots,\{6,6\}\right\}$
$\Omega=\left\{(1,1),(1,2),\dots,(6,6)\right\}$ med kardinalitet $|\Omega|=6^2 = 36$, som er antallet af de forskellige udfald.
%$S = \{2, 3, 4, 5, 6, 7, 8, 9, 10, 11, 12\},$
Lad $X$ være summen af de to terningkast således, at $\text{Range}(X)=\{2, 3, \dots , 12\}$. Først bestemmes frekvensfunktionen for hvert $x \in \text{Range}(X)$.
%
\begin{align*}
    p_X(2)=\frac{1}{36},\  
    p_X(3)=\frac{2}{36},\  
    \dots,\  
    p_X(12)=\frac{1}{36}.
\end{align*}

I nedenstående pindediagram, \autoref{fig:eksempel_med_frekvensfunktion}, illustreres frekvensfunktionen for $X$.  

\begin{figure}[H] \centering
\pgfplotsset{
      standard/.style={
         %width=0.85\linewidth, %størrelse på graf
         axis x line=middle,
         axis y line=middle,
         enlarge x limits=0.15,
         every axis x label/.style={at={(1,-0.01)},anchor=north west},
         every axis y label/.style={at={(0,1.12)},anchor=north},
         every axis plot post/.style={mark options={fill=white}}
      }
   }
   \begin{tikzpicture}
      \begin{axis}[
      standard,
      domain = 2:12,
      samples = 11,
      xlabel={$x$},
      ylabel={$p_X(x)$},
      ymin=0,
      yticklabels={$1/36$, $2/36$, $3/36$, $4/36$, $5/36$, $6/36$},
      ytick = {1, 2, 3, 4, 5, 6},
      ymax=6.8]
      \addplot+[ycomb,black,thick] {(6-abs(x-7))};
      \end{axis}
   \end{tikzpicture}
\caption{Frekvensfunktion for summen af to terningkast.}\label{fig:eksempel_med_frekvensfunktion}
\end{figure}
%
I pindediagrammet repræsenterer søjlerne de enkelte udfald, $x\in \text{Range}(X)$, og højden repræsenterer den tilhørende sandsynlighed, $p_X(x)$.
%I \autoref{fig:eksempel_med_frekvensfunktion} illustrerer højden af søjlerne de enkelte sandsynligheder for at $X$ antager værdien $x$.
\end{eks}

I den forrige teori har hændelsen været på formen $\{X=x\}$, hvor $X$ har antaget én bestemt værdi, $x$. Det er herudover muligt at bestemme sandsynligheden for hændelser på formen $\{X\leq x\}$, altså hvor $X$ antager værdier mindre end eller lig $x$. Da $X$ er en diskret tilfældig variabel, er det kun nødvendigt at analysere elementer i $\text{Range}(X)$.

Hvis $\mathcal{Y}_X = \{y \in \text{Range}(X) \ | \ y\leq x\}$, kan mængden, hvor $X$ antager værdier mindre end eller lig $x$ udtrykkes ved
\begin{align*}
    \{X \leq x\} = \bigcup_{y \in \mathcal{Y}_X} \{X = y\}. 
\end{align*}

Funktionen, der bestemmer sandsynligheden $P(X \leq x)$, defineres som følgende

\begin{minipage}\textwidth
\begin{defn}\textbf{Fordelingsfunktion}\label{Def:Kumulative_fordelingsfunktion} %Ny definition
\newline
Lad $X$ være en diskret tilfældig variabel. Funktionen $F_X:\R \to [0,1]$, defineret ved
$$F_X(x)=P(X\leq x) \text{ for } x\in\R,$$
kaldes for \textit{fordelingsfunktionen} til $X$.
\end{defn}
\end{minipage}

Fra \autoref{Def:Kumulative_fordelingsfunktion} og \autoref{def:sandsynlighedsregningens_grundsætninger} punkt 3 følger det, at
%
\begin{align}
    F_X(x) &= P\left(X\leq x\right)=P\left(\bigcup_{y\in \mathcal{Y}_X} \{X=y\}\right) = \sum_{y\in \mathcal{Y}_X} P(X=y) = \sum_{y\in \mathcal{Y}_X} p_X(y), \label{eq:fordelingsfunktion_som_sum_af_frekvensfunktioner}
\end{align}
hvor $\mathcal{Y}_X = \{y\in \text{Range}(X) \ | \ y\leq x\}$.
%
Altså kan fordelingsfunktionen udtrykkes som en sum af frekvensfunktioner, hvilket illustreres i \autoref{eks:fordelingsfunktionen}.


\begin{eks}\textbf{}\label{eks:fordelingsfunktionen}\\
Lad $\E$ være givet som i \autoref{eks:frekvensfunktion_med_terninger}, hvor frekvensfunktionen for alle $x \in \text{Range}(X)$ blev bestemt.

Fordelingsfunktionen for alle $x \in \text{Range}(X)$ bestemmes ved brug af \eqref{eq:fordelingsfunktion_som_sum_af_frekvensfunktioner}. 
\begin{align*}
    F_X(2)&=\sum_{k=2}^2 p_X(k)=p_X(2)=\frac{1}{36},\\
    F_X(3)&=\sum_{k=2}^3 p_X(k)=p_X(2)+p_X(3)=\frac{1}{36}+\frac{2}{36}=\frac{3}{36},\\
    &\vdots\\
    F_X(12)&=\sum_{k=2}^{12}p_X(k)=p_X(2)+p_X(3)+\cdots+ p_X(12)=\frac{1}{36}+\frac{2}{36}+\cdots+\frac{1}{36}=1.
\end{align*}

I nedenstående, \autoref{fig:eksempel_fordelingsfunktion_terningkast}, illustreres fordelingsfunktionen for $X$. 

\begin{figure}[H]
    \centering
    
\begin{tikzpicture}
\begin{axis}[axis y line=middle,
    axis x line=middle,
	%title=Fordelingsfunktion for summen af to terningkast,
	unbounded coords=jump,
	 xlabel={$x$},
     ylabel={$F_X(x)$},
     xmax=13.4,
     ymax=1.15,
     every axis x label/.style={at={(1,-0.01)},anchor=north west},
     every axis y label/.style={at={(0,1.12)},anchor=north},
    yticklabels={$0$,$9/36$, $18/36$,$27/36$, $1$},
    ytick = {0.001, 9/36, 18/36, 27/36, 1}
    ]
	\addplot[color=black,mark=none, line width=0.3mm] coordinates {
        (0,inf) (0,0) (2,0)
        (2, inf) (2, 1/36) (3, 1/36)
 		(3,inf) (3, 3/36) (4,3/36)
 		(4,inf) (4, 6/36) (5,6/36)
 		(5,inf) (5, 10/36) (6,10/36)
 		(6,inf) (6, 15/36) (7,15/36)
 		(7,inf) (7, 21/36) (8, 21/36)
        (8,inf) (8, 26/36) (9, 26/36)
        (9,inf) (9, 30/36) (10,30/36)
       (10,inf) (10, 33/36) (11, 33/36)
       (11,inf) (11, 35/36) (12, 35/36)
       (12, inf) (12, 1) (14, 1)
    };
    
    \addplot[color=black,mark=*, only marks] coordinates {
        (2, 1/36)
        (3, 3/36)
        (4, 6/36)
        (5, 10/36)
        (6, 15/36)
        (7, 21/36)
        (8, 26/36)
        (9, 30/36)
        (10, 33/36)
        (11, 35/36)
        (12, 36/36)
    };
    \addplot[color=black,mark=*, fill=white, only marks] coordinates {
        (2, 0)
        (3,1/36)
 		(4,3/36)
        (5,6/36)
 		(6,10/36)
 		(7,15/36)
 		(8, 21/36)
        (9, 26/36)
        (10,30/36)
        (11, 33/36)
        (12, 35/36)
    };
\end{axis}
\end{tikzpicture}
\caption{Fordelingsfunktion for summen af to terningkast.}\label{fig:eksempel_fordelingsfunktion_terningkast}
\end{figure}

På grafen ses det, at funktionen springer ved hver værdi, $X$ kan antage. Ethvert spring i fordelingsfunktionen repræsenterer værdien for frekvensfunktionen. 
\end{eks}

Det er en generel egenskab for fordelingsfunktionen, at $F_X(x_1)\leq F_X(x_2)$ for $x_1\leq x_2$. Dette gør sig gældende, da $\{X\leq x_1\}\subseteq \{X\leq x_2\}$ for $x_1\leq x_2$.

