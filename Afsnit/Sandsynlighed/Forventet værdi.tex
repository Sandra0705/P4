


%\textit{Den forventede værdi} afhænger af de gennemsnitlige værdier af et eksperiment. 
Ud fra frekvensfunktionen, er det muligt at bestemme den \textit{forventede værdi} for en diskret tilfældig variabel. Den forventede værdi afhænger af de gennemsnitlige værdier for de diskrete tilfældige variabler i et tilfældigt eksperiment, $\E$. Den forventede værdi defineres som følgende.

\begin{minipage}\textwidth
\begin{defn}\textbf{Forventet værdi} \label{def:Forventetværdi} %Ny definition 
\newline
Lad $X$ være en diskret tilfældig variabel med den tilhørende frekvensfunktion $p_X$. Den forventede værdi af $X$ er defineret ved
\begin{align}
E[X]=\sum_{x\in \text{Range}(X)} x p_X(x),
\end{align}
hvis summen er absolut konvergent.
\end{defn}
\end{minipage}

Den forventede værdi er et vægtet gennemsnit af de mulige værdier af $X$ med tilsvarende sandsynligheder som vægte. 

Den forventede værdi af en funktion, som afhænger af $X$, kan opstilles som følgende

\begin{minipage}\textwidth
\begin{pro} \textbf{} \label{prop:forventet_værdi_af_funktion} %Ny proposition
\newline
Lad $X$ være en diskret tilfældig variabel med den tilhørende frekvensfunktion $p_X$. Lad $g: \R \to \R$ være en reel funktion. Så gælder det, at
\begin{align}           
    E\left[g(X)\right]=\sum_{x \in \text{Range}(X)} g(x)p_X(x),
\end{align}
hvis summen er absolut konvergent.
\end{pro}
\end{minipage}

\begin{bev}\textbf{} %Nyt bevis
\newline
Lad $(\Omega, \F, P)$ være et sandsynlighedsrum og $X: \Omega \to \R$ være en diskret tilfældig variabel med frekvensfunktion, $p_X$. Lad $I= \text{Range}(X)$ og $Y=g(X)$, så er $Y$ en diskret tilfældig variabel med billede $\text{Range}\hspace{-2pt}\left(g(I)\right)$ og den tilhørende frekvensfunktion $p_Y$. Fra \autoref{def:Frekvensfunktionen} og \autoref{def:Diskret_tilfældig_variabel} gælder det, at
%
\begin{align*}
    p_Y(y) = P(Y = y) = P\left(\{\omega \in \Omega: Y(\omega) = y\}\right).
\end{align*}
%
Af \autoref{def:Forventetværdi} følger det, at
%
\begin{align}\label{eq:E[Y]}
    E[Y] = \sum_{y \in \text{Range}\left(g(I)\right)} yp_Y(y).
\end{align}
%
For ethvert $y\in \text{Range}\hspace{-2pt}\left(g(I)\right)$ eksisterer der mindst et $x\in I$ således, at $g(x)=y$. Dette medfører, at
% 
\begin{align*}
    P\left(\{\omega \in \Omega: Y(\omega) = y\}\right) = P\left(\left\{\omega \in \Omega: X(\omega) \in \{x\in I| g(x)=y\}\right\}\right).
\end{align*}
Derudover gælder det, at
\begin{align*}
    \left\{\omega \in \Omega: X(\omega) \in \{x\in I| g(x)=y\}\right\} = \bigcup_{\substack{x \in I:\\ g(x) =y}} \{\omega \in \Omega |X(\omega) = x\}.
\end{align*}
Hvorom det gælder, at
\begin{align*}
    \{\omega \in \Omega | X(\omega)=x_j\} \cap \{\omega \in \Omega | X(\omega)=x_i\} = \emptyset \text{ for alle } i,j=1,2, \ldots \Rightarrow x_j\neq x_i.
\end{align*}
Det følger da af \autoref{def:ovegangsmatrice} punkt 3 og \autoref{def:Frekvensfunktionen}, at
\begin{align}
    P\left(\left\{\omega \in \Omega: X(\omega) \in \{x\in I| g(x)=y\}\right\}\right)&=\sum_{\substack{x \in I:\\ g(x) =y}} P\left(\{\omega \in \Omega |X(\omega) = x\}\right)\nonumber\\
    &=\sum_{\substack{x \in I:\\ g(x) =y}}p_X(x).\label{eq:bevis_for_forventet_værdi_som_funktion_2}
\intertext{Ved at indsætte \eqref{eq:bevis_for_forventet_værdi_som_funktion_2} i \eqref{eq:E[Y]} fås følgende}
    E[Y] = \sum_{y \in \text{Range}\left(g(I)\right)} y\left( \sum_{\substack{x \in I:\\ g(x) =y}}p_X(x) \right)
    &=\sum_{y \in \text{Range}\left(g(I)\right)}\sum_{\substack{x \in I:\\ g(x) =y}} g(x) p_X(x). \nonumber\\
    \intertext{Da}
    \bigcup_{y\in \text{Range}\left(g(I)\right)}\bigcup_{\substack{x\in I\\g(x)=y}}\{x\}&=\bigcup_{y\in\text{Range}\left(g(I)\right)}\{x\in I|g(x)=y\} = I,\nonumber\\
    \intertext{følger det, at}
    \sum_{y \in \text{Range}\left(g(I)\right)}\sum_{\substack{x \in I:\\ g(x) =y}} g(x) p_X(x)&=\sum_{x\in I}g(x)p_X(x). \nonumber
\end{align}

Da $Y=g(X)$ og $I = \text{Range}(X)$, gælder det, at
\begin{align*}
    E\left[g(X)\right]=\sum_{x\in \text{Range}(X)}g(x)p_X(x).
\end{align*}
Dermed er \autoref{prop:forventet_værdi_af_funktion} bevist.
\end{bev}

Hvis sandsynligheden for en diskret tilfældig variabel, $X$, er betinget af en hændelse, $B\in\F$, påvirker det sandsynlighedsfordelingen af $X$. Sandsynligheden $P(X=x)$ erstattes derfor af $P(X=x|B)$.

\begin{minipage}\textwidth
\begin{defn}\textbf{Betingede forventede værdi}\label{def:betinget_forventet_værdi} %Ny definition
\newline
Lad $(\Omega, \F, P)$ være et sandsynlighedsrum og $B\in\F$ en hændelse. Lad $X$ være en diskret tilfældig variabel og $P(B)>0$. Så er den betingede forventede værdi af $X$ givet $B$ defineret ved 
\begin{align*}
    E[X|B]&=\sum_{x\in Range(X)}xP(X=x|B),
\end{align*}
hvis summen er absolut konvergent.
\end{defn}
\end{minipage}

I ovenstående definition er den forventede værdi af den diskrete tilfældige variabel, $X$, betinget af en hændelse, $B$. Hvis den forventede værdi af en diskret tilfældig variabel $X$ derimod er betinget af udfaldet af en anden diskret tilfældig variabel $Y$, defineres den betingede forventede værdi som følgende

\begin{minipage}\textwidth
\begin{defn}\textbf{Betingede forventede værdi af en diskret tilfældig variabel}\label{def:betinget_forventet_værdi_af_diskrete_tilfældige_variabler} %Ny definition
\newline
Lad $X, Y$ være diskrete tilfældige variabler. Så er den betingede forventede værdi af $X$ givet $Y$, givet ved
\begin{align*}
    E[X|Y=y] = \sum_{x\in Range(X)}xP(X=x|Y=y),
\end{align*}
hvis summen er absolut konvergent.
\end{defn}
\end{minipage}

Den forventede værdi giver ikke information om variationen for $X$. Derfor introduceres begrebet \textit{varians}, hvilket defineres som følgende

\begin{minipage}\textwidth
\begin{defn}\textbf{Varians}\label{def:varians} %Ny definition
\newline
Lad $X$ være en diskret tilfældig variabel med den forventede værdi, $E[X]$. Så er variansen af $X$ givet ved
\begin{align*}
    \text{Var}[X]=E\left[\left(X-E[X]\right)^2\right].
\end{align*}
\end{defn}
\end{minipage}

%Fremadrettet beskrives varians ved $\sigma^2$. 

%Jo større variansen er jo mere varierer X omkring den forventede værdi.

Bemærk desuden, at variansen er større end eller lig $0$. En højere varians medfører en større gennemsnitlig afvigelse fra den forventede værdi. Dette kaldes også for \textit{standardafvigelse}, som præsenteres i følgende definition.

\begin{minipage}\textwidth
\begin{defn}\label{def:standardafvigelse}\textbf{Standardafvigelse} %Ny definition
\newline
Lad $X$ være en diskret tilfældig variabel med varians, $\text{Var}[X]$. Standardafvigelsen for $X$ er defineret ved
$$\sigma = \sqrt{\text{Var}[X]}.$$
\end{defn}
\end{minipage}

Følgende korollar kan i nogle tilfælde simplificere beregningen af variansen.

\begin{minipage}\textwidth
\begin{kor} \textbf{}\label{kor:varians} %Nyt korollar
\newline
Lad $X$ være en diskret tilfældig variabel med den forventede værdi, $E[X]$. Så er variansen givet ved
\begin{align*}
    \text{Var}[X]=E[X^2] - \left(E[X]\right)^2.
\end{align*}
\end{kor}
\end{minipage}

\begin{bev} \textbf{} %Nyt bevis
\newline
Lad $X$ være en diskret tilfældig variabel med den tilhørende frekvensfunktion, $p_X$, og lad $I=\text{Range}(X)$. Det følger af \autoref{def:varians} og \autoref{prop:forventet_værdi_af_funktion}, at
\begin{align*}
    \text{Var}[X] &=E\left[\left(X-E[X]\right)^2\right] \\
    &=\sum_{x\in I}\left(x-E[X]\right)^2 p_X(x)\\
    &= \sum_{x\in I} \left(x^2-2xE[X]+\left(E[X]\right)^2\right)p_X(x)\\
    &= \sum_{x\in I} x^2p_X(x) - 2E[X] \sum_{x\in I} x p_X(x) + \left(E[X]\right)^2 \sum_{x\in I} p_X(x)\\
    &= E[X^2]-2\left(E[X]\right)^2+\left(E[X]\right)^2
    = E[X^2] - \left(E[X]\right)^2.
\end{align*}
Dermed er \autoref{kor:varians} bevist.
\end{bev}

Den forventede værdi og varians af et terningkast beregnes i \autoref{eks:forv_og_var}.

\begin{eks} \textbf{} \label{eks:forv_og_var} %Nyt eksempel
\newline
Lad $\E$ være et tilfældigt eksperiment hvor der kastes en fair terning. Lad $X$ være en tilfældigt variabel til $\E$ med tilhørende frekvensfunktion $p_X$.

%Hvis frekvensfunktionen har den samme værdi for alle mulige udfald, siges mængden af sandsynligheder at væreuniform fordelt.

Frekvensfunktionen er uniformt fordelt med $\text{Range}(X)=\{1,2,3,4,5,6\}$ og $p_X=\frac{1}{6}$.
Dermed er den forventede værdi givet ved
\begin{align*}
    E[X]=\sum_{x=1}^6 xp_X(x) = \dfrac{1}{6}\sum_{x=1}^6 x=\dfrac{7}{2}.
\end{align*}
Ved brug af \autoref{prop:forventet_værdi_af_funktion} fås
\begin{align*}
    E[X^2]=\sum_{x=1}^6  x^2 p_X(x) = \dfrac{1}{6} \sum_{x=1}^6 x^2 = \dfrac{91}{6}.
\end{align*}
Ud fra \autoref{kor:varians} bestemmes variansen
\begin{align*}
    \text{Var}[X]  &=E[X^2]-\left(E[X]\right)^2\\
            &=\frac{91}{6} - \left(\frac{7}{2}\right)^2 = \dfrac{35}{12}.
\end{align*}
Dermed er variansen for et terningkast bestemt til at være $\text{Var}[X] = \displaystyle \frac{35}{12}$ og den forventede værdi er $\frac{7}{2}$.
\end{eks}

