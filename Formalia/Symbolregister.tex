%Symbolregister

% Symbolregisteret kan opdeles som set nedenfor, ved at indsætte \nomenclature[A]{$mit-symbol$}{forklaring} indsættes det under kategori A, kategori kan laves i preamble under symbolregister, hvis ikke et symbol får en kategori gives den "other".

% Eksempel på et symbolregister:

% A = Indledning/Problemanalyse/Problemformulering


% B = Sandsynlighed
%Mængder
\nomenclature[B,01]{$\Omega$}{Et udfaldsrum}
\nomenclature[B,02]{$\omega$}{Element af udfaldsrummet}
\nomenclature[B,04]{$A, B$}{Hændelser}
\nomenclature[B,03]{$\F$}{Hændelsesrum}
\nomenclature[B,05]{$\emptyset$}{Den tomme mængde}
\nomenclature[B,08]{$\R$}{De reelle tal}
\nomenclature[B,06]{$P(\cdot)$}{Sandsynlighedsmål}
\nomenclature[B,09]{$A_k, B_k$}{Den $k$'te hændelse}
\nomenclature[B,11]{$P(A \mid B)$}{Sandsynligheden for $A$ givet $B$}
\nomenclature[B,10]{$k,n,i,j$}{Tællevariabler}
\nomenclature[B,00]{$\E$}{Tilfældigt eksperiment}
\nomenclature[B,12]{$X, Y$}{Diskrete tilfældige variabler}
\nomenclature[B,13]{Range$(\cdot)$}{Billedet af en diskret tilfældig variabel}
\nomenclature[B,07]{$(\Omega,\F, P)$}{Sandsynlighedsrum}
\nomenclature[B,14]{$X^{-1}$}{Urbilledet af $X$}
\nomenclature[B,15]{$p_X(\cdot)$}{Frekvensfunktionen til $X$}
\nomenclature[B,16]{$\mathcal{Y}$}{En mængde}
\nomenclature[B,16]{$F_X(\cdot)$}{Fordelingsfunktion til $X$}
\nomenclature[B,18]{$g(\cdot)$}{Vilkårlig reel funktion funktion}
\nomenclature[B,17]{$E[\cdot]$}{Den forventede værdi}
\nomenclature[B,19]{$E[X \mid Y]$}{Den forventede værdi for $X$ givet $Y$}
\nomenclature[B,20]{Var$[\cdot]$}{Varians}
\nomenclature[B,21]{$\sigma$}{Standardafvigelsen}

%Diverse


% C = Markov-kæder
\nomenclature[C,01]{$\bm X$}{En sekvens af diskrete tilfældige variabler }
\nomenclature[C,02]{$S$}{Tilstandsrummet}
\nomenclature[C,03]{$T$}{Indeksmængden}
\nomenclature[C,04]{$j, i$}{Tilstande i tilstandsrummet}
\nomenclature[C,05]{$t, m, r$}{Indicer i indeksmængden}
\nomenclature[C,06]{$p_{ij}$}{Overgangssandsynligheden for en Markov-kæde}
\nomenclature[C,07]{$\P$}{Overgangsmatricen for en Markov-kæde}
\nomenclature[C,08]{$P(X_0 = i_0)$}{Begyndelses -\\sandsynligheden}
\nomenclature[C,09]{$p_{ij}^{(t)}$}{$t$'te trins overgangssandsynligheden for en Markov-kæde}
\nomenclature[C,10]{$\P^{(t)}$}{$t$'te trins overgangsmatricen for en Markov-kæde}
\nomenclature[C,11]{$i \to j$}{$j$ er tilgængelig fra $i$}
\nomenclature[C,12]{$i \leftrightarrow j$}{$j$ og $i$ kommunikerer}
\nomenclature[C,13]{$\tau_i$}{Antal trin før Markov-kæden besøger $i$}
\nomenclature[C,14]{$d(i)$}{Største fælles divisor af antallet af trin for at vende tilbage til $i$.}
\nomenclature[C,15]{$\bm \pi$}{Invariant fordeling}
\nomenclature[C,16]{$\pi_i$}{Den $i$'te indgang i den invariante fordeling}
\nomenclature[C,17]{$\bm q$}{En sandsynlighedsfordeling}
\nomenclature[C,18]{$q_i$}{Den $i$'te indgang i en sandsynlighedsfordeling}



% \noncleture[BOGSTAV,VÆGT]{symbol}{symbolforklaring}
% Hvor BOGSTAV betegner afsnittet den skal i og VÆGT betegner hvor den skal hen (01 er først og 0n er sidst)

% D =  Markov beslutningsprocesser
\nomenclature[D,01]{$T$}{Mængden af beslutningstidspunkter}
\nomenclature[D,02]{$t$}{Beslutningstidspunkter}
\nomenclature[D,03]{$S$}{Tilstandsrummet}
\nomenclature[D,04]{$s$}{Tilstande}
\nomenclature[D,05]{$S_t$}{Mængden af tilstande til $t$}
\nomenclature[D,06]{$A$}{Beslutningsrummet}
\nomenclature[D,07]{$a$}{Beslutninger}
\nomenclature[D,08]{$A_s$}{Mængden af mulige beslutninger til $s$}
\nomenclature[D,09]{$A_{s,t}$}{Mængden af mulige beslutninger til $s$ og $t$}
\nomenclature[D,10]{$r_t(s,a)$}{Belønning}
\nomenclature[D,11]{$r_t(s,a,s')$}{Belønning afhængig af tilstanden til $t+1$}
\nomenclature[D,12]{$s'$}{Tilstanden til beslutningstidspunktet $t+1$}
\nomenclature[D,13]{$p_t(\cdot \mid s,a)$}{Overgangssandsynlig - \\hedsfunktionen}
\nomenclature[D,14]{$N$}{Sidste beslutningstidspunkt}
\nomenclature[D,15]{$d_t$}{Beslutningsregel}
\nomenclature[D,16]{$h_t$}{Historik}
\nomenclature[D,17]{$H_t$}{Mængden af historikker}
\nomenclature[D,18]{$q_{d_t}(\cdot)$}{Sandsynlighedsfordelingen for mængden af beslutningsregler}
\nomenclature[D,19]{$D_t^K$}{Mængden af beslutningsregler}
\nomenclature[D,20]{$K$}{Klassen af beslutningsregler}
\nomenclature[D,21]{$\Psi$}{Strategi}
\nomenclature[D,22]{$\Pi$}{Mængden af alle strategier}
\nomenclature[D,23]{$\omega$}{Udfaldsvej}
\nomenclature[D,24]{$Z_t, W$}{Diskret tilfældig variabel}
\nomenclature[D,25]{$P^\Psi$}{Sandsynlighedsmål givet en strategi}
\nomenclature[D,26]{$E^\Psi$}{Den forventede værdi givet en strategi}
\nomenclature[D,27]{$v_N^{\Psi}$}{Den forventede totale belønning givet en strategi}
\nomenclature[D,28]{$\Psi^*$}{Den optimale strategi}
\nomenclature[D,29]{$v_N^{\Psi^*}$}{Den forventede totale belønning givet en optimal strategi}
\nomenclature[D,30]{$\Psi^*_\varepsilon$}{Den $\varepsilon$-optimale strategi}
\nomenclature[D,31]{$v_N^{\Psi^*_\varepsilon}$}{Den forventede totale belønning givet en $\varepsilon$-optimal strategi}
\nomenclature[D,32]{$v_N^{*}$}{Den maksimale forventede totale belønning}
\nomenclature[D,33]{$u_t^{\Psi}$}{Den forventede totale belønning givet en strategi fra beslutningstidspunkt $t$}
\nomenclature[D,34]{$u_t^{*}=u_t$}{Den maksimale forventede totale belønning fra beslutningstidspunkt $t$}
\nomenclature[D,35]{$G$}{En vilkårlig diskret mængde}
\nomenclature[D,36]{$z$}{Element i $G$.}
\nomenclature[D,37]{$a^\circ$}{En beslutning}
\nomenclature[D,38]{$\lambda$}{Diskonteringsfaktor}
\nomenclature[D,39]{$v_{N,\lambda}^{\Psi}$}{Den forventede totale diskonterede belønning givet en strategi}


% E =Investeringsproblemet
\nomenclature[E,01]{$T$}{Mængden af beslutningstidspunkter}
\nomenclature[E,02]{$t$}{Beslutningstidspunkter}
\nomenclature[E,03]{$S$}{En opsparing}
\nomenclature[E,04]{$S_t$}{En opsparing til beslutningstidspunkt $t$}
\nomenclature[E,05]{$a_t$}{En beslutning der angiver et forbrug til beslutningstidspunkt $t$}
\nomenclature[E,06]{$A_t$}{Mængden af beslutninger $a_t$}
\nomenclature[E,07]{$\gamma_i$}{Indlånsrenten}
\nomenclature[E,07]{$\gamma_u$}{Udlånsrenten}
\nomenclature[E,08]{$I$}{Indkomst}
\nomenclature[E,09]{$\bm \xi$}{Stokastisk proces for arbejdsstatus}
\nomenclature[E,10]{$\xi$}{Arbejdsstatus til beslutningstidspunkt $t$}
\nomenclature[E,11]{$\alpha$}{Sandsynligheden for at investoren mister sit arbejde i næste beslutningstidspunkt givet, at han er i arbejde i det nuværende beslutningstidspunkt}
\nomenclature[E,12]{$\beta$}{Sandsynligheden for at investoren er arbejdsløs i næste beslutningstidspunkt givet, at han ikke er i arbejde til nuværende arbejdstidspunkt}
\nomenclature[E,13]{$\Delta$}{En diskretseringfaktor}
\nomenclature[E,14]{$L_t$}{Samlede lån indtil $t$}
\nomenclature[E,15]{$L$}{Lånebegrænsning}



% F = Problemløsning

% P = Diverse


\printnomenclature[2.5cm]

% Ja præcis den her :))))))))))))