This paper examines how to determine the optimal consumption and maximal reward for a consumer in a given period. Furthermore, it studies how the probability of unemployment, interest rate and loan affect the optimal consumption and maximal reward. The Consumer Problem is a Markov Decision Process and therefore, the optimal consumption and maximal reward can be determined using backward induction and the Bellman equation. To arrive at this conclusion and to implement the probability of unemployment, it is necessary to introduce basic concepts within probability theory such as conditional probability, expected value and Markov-chains. Moreover, Markov Decision Processes will be examined and implemented within the Consumption Problem using Python.
From this, an inverse relationship between unemployment and total reward has been found, while the reasonable consumer has a positive relationship between consumption and interest rate. Furthermore, the loan restriction and ratio between the lendingrate and depositrate affect the reward and the consumption.




